Come già sintetizzato nell'OdG, il gruppo \Gruppo{} ha posto numerosi quesiti all'azienda proponente. Qui di seguito verranno riportati, in maniera molto sintetizzata, i principali punti di discussione con il Dott. Patera.
\begin{enumerate}
	\item \textbf{Discussione e confronto di tecnologie e scelte progettuali del capitolato}: durante la riunione, il gruppo ha richiesto dei chiarimenti al Dott. Patera riguardo i seguenti punti:
	\begin{itemize}
		\item \textbf{marcatura igienizzazione stanza}: il gruppo e il proponente hanno concordato di marcare una intera stanza come igienizzata solamente dopo che un apposito addetto alle pulizie, ovvero l'igienizatore, ha terminato la sanificazione delle postazioni, indipendentemente dal fatto che queste ultime siano state già segnate come igienizzate da un utente precedente;
		\item \textbf{tipo di applicazione}: il Dott. Patera ha consigliato di sviluppare un'applicazione che non sia monolitica, ma sviluppata a micro-servizi, per esigenze di scalabilità e di facilità di manutenzione futura;
		\item \textbf{risorse e requisiti hardware dell'applicazione mobile}: l'attenzione al risparmio energetico dell'applicazione mobile che verrà sviluppata dal gruppo \Gruppo{} è cruciale ai fini di una valutazione positiva del capitolato in questione. Nello specifico, il gruppo dovrà produrre dei test sufficientemente esaustivi riguardanti il consumo di batteria durante un uso continuativo dell'applicazione.
		\item \textbf{tecnologie fissate}: Imola Informatica non obbliga gli studenti del gruppo \Gruppo{} all'utilizzo di una determinata tecnologia, preferendo concedere libertà di scelta. Il repentino cambio di una determinata tecnologia, come quella dei tag RFID a favore di un eventuale uso di QR-Code, dovrà essere sufficientemente documentata dal gruppo e comunicata in maniera repentina all'azienda;
		\item \textbf{accessibilità del sito}: rendere il sito dell'applicativo che il gruppo \Gruppo{} dovrà produrre accessibile ad eventuali utenti con disabilità visive, uditive o motorie sarà un valore aggiunto al capitolato stesso, ma non sarà obbligatorio al fine del soddisfacimento di tutti i vincoli del capitolato.
	\end{itemize}
	\item \textbf{Mezzi di comunicazione preferenziali con il proponente}:
	\begin{itemize}
		\item \textbf{gruppo Telegram}: il gruppo, in accordo con il Dott. Patera, si è reso disponibile alla futura creazione di un gruppo Telegram tra l'azienda ed i componenti del gruppo \Gruppo{}, favorendo quindi una comunicazione più veloce e diretta tra le due parti;
		\item \textbf{cadenza dei meeting}: il proponente lascia la massima disponibilità organizzativa al gruppo \Gruppo{} per stabilire le tempistiche di ritrovo con l'azienda proponente, favorendo quindi un tipo di incontro non regolarmente cadenziato, ma ondemand.
	\end{itemize}
\end{enumerate}