\begin{enumerate}
    \item è stata fissata una data entro cui la correzione della documentazione deve essere completata ovvero il: 2021-02-16;
    \item è stato contattato il committente per fissare un incontro;
    \item è stato deciso di spostare la documentazione della repo da Github a Gitlab per usufruire di una migliore gestione delle pipelines;
    \item è stata fatto notare che il gruppo in questa fase ha utilizzato Jira in modo troppo sporadico e che il livello di 
    comunicazione che deve essere maggiore.
\end{enumerate}

    La pianificazione sarà la seguente:
    \begin{itemize}
    \item correzione della documentazione;
    \item analisi di eventuali criticità nell'utilizzo della tecnologia NFC con discussione su possibili soluzioni alternative;
    \item raffinamento dell'analisi;
    \item inizializzazione dell'infrastruttura;
    \item progettazione architetturale dell'applicazione;
    \item implementazione delle componenti più critiche ed essenziali ai fini del \glo{PoC}, ovvero:
        \begin{itemize}
            \item tracciamento dell'utente via smartphone;
            \item comunicazione dati tra smartphone e database;
            \item salvataggio dell'hash generato in blockchain e della relativa transazione.
        \end{itemize}
    \end{itemize}
    Lo stack tecnologico proposto è il seguente:
    \rowcolors{2}{\evenRowColor}{\oddRowColor}
	\renewcommand{\arraystretch}{2}
    \begin{longtable}{ C{5cm} C{4cm} C{5cm} }
        \caption{Tabella stack infrastrutturale}                                                                                 \\
        \rowcolor{\primaryColor}
        \textcolor{\secondaryColor}{\textbf{Nome}} & \textcolor{\secondaryColor}{\textbf{Contesto di utilizzo}} & \textcolor{\secondaryColor}{\textbf{Motivazione}}\\ \endhead
        {Kubernets} & {Per la gestione del container}   & {È stato consigliato sia dal committente che da un membro del team con esperienza}\\
    \end{longtable}

    \begin{longtable}{ C{5cm} C{4cm} C{5cm} }
        \caption{Tabella stack tecnologico di sviluppo}                                                                                 \\
        \rowcolor{\primaryColor}
        \textcolor{\secondaryColor}{\textbf{Nome}} & \textcolor{\secondaryColor}{\textbf{Contesto di Utilizzo}} & \textcolor{\secondaryColor}{\textbf{Motivazione}}\\ \endhead
        {Android} & {Per lo sviluppo dell'applicazione mobile}   & {La maggior parte dei componenti ha un dispositivo con OS Android e la relativa libreria è molto ben documentata}\\
        {Kotlin} & {Per lo sviluppo nativo dell app mobile (utente)} & {Facile leggibilità ed è consigliato da Google} \\
        {Typescript} & {Per lo sviluppo relativo all'applicazione Web (Admin)} & { Facilita la scrittura di codice JavaScript } \\
        {MongoDB} & {Per il database} & { Permette una facile e veloce integrazione di applicazioni con mole di dati alta } \\
        {RabbitMQ} & {Per garantire la continuità del servizio} & {Per rendere l'applicazione quasi sempre disponibile e con grande affidabilità} \\
        {Kong} & {Per la comunicazione via API tra componenti} & {Facilita lo sviluppo di API e microservizi} \\
        {Angular} & {Grafica applicazione web} & {Framework che facilita lo sviluppo di UI scitte in Javascript} \\
        {Keycloak} & {Logica di Login} & {Facilita l'implementazione di logiche relative all'accesso autorizzato} \\
        {Truffle framework} & {Testing comunicazione blockchain} & {Facilita lo sviluppo e il testing di applicazioni dove è richiesto l'utilizzo di blockchain basate su Ethereum} \\
        {Solidity} & {Comunicazione blockchain per salvare l'hash} & {Consente di creare contratti facili da sviluppare e sicuri (curva di apprendimento esponenziale)} \\
        {NestJS} & {Backend logica applicazione} & {Facilita l'implementazione di codice JavaScript lato backend} \\
        {NodeJS} & {Per lo sviluppo della parte backend e frontend} & {Facilita la scrittura di codice Javascript includendo diversi moduli} \\
    \end{longtable}
    
