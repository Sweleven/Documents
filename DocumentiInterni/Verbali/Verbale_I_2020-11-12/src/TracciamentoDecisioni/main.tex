Dopo una attenta analisi e discussione, il gruppo ha scelto di dividere il lavoro di
creazione della documentazione iniziale. Nello specifico, il gruppo ha concordato
di poter già iniziare il lavoro sulle seguenti documentazioni relative ad ogni
capitolato scelto:
\begin{itemize}
    \item studio di Fattibilità
    \item norme di Progetto;
    \item glossario.
\end{itemize}

Data la mole di lavoro impegnativa, il gruppo ha deciso di suddividere il lavoro
creando due sotto-gruppi, con lo specifico compito di redigere un determinato
documento. I due sotto gruppi, composti relativamente da tre e quattro persone,
sono stati così formati:
\begin{itemize}
    \item sotto-gruppo 1: Filippo Pinton, Alessio Trevisan , Edoardo Caregnato;
    \item sotto-gruppo 2: Elvis Murtezan, De Tomasi Andrea, Grigoletto Giovanni,
    Kandoul Abdelwahad.
\end{itemize}

I relativi compiti dei due sotto-gruppi sono i seguenti:
\begin{itemize}
    \item sotto-gruppo 1: Redigere norme di progetto per ciascuno dei capitolati scelti;
    \item sotto-gruppo 2: Redigere lo studio di fattibilit`a per ciascuno dei capitolati
    scelti.
\end{itemize}
Al termine dei compiti assegnati ad ogni sotto-gruppo, ciascuno farà da verificatore della documentazione prodotta al gruppo opposto. La consegna di tale
materiale è fissata entro e non oltre Giovedì 3 Dicembre 2020.
\subsection{Programmazione prossimo incontro}
Il gruppo ha scelto di trovarsi sempre tramite canale Discord nella metodologia
”ondemand”, ovvero in caso di necessità da parte di uno o più membri del
gruppo. In caso però di adozione della metodologia di lavoro ”Scrum”, il gruppo
intende fissare meeting ricorrenti di circa 1/2 settimane, per verificare il lavoro
svolto da ogni singolo componente ed evitare carichi di lavoro eccessivi nel caso in
cui un componente del gruppo non sia riuscito a portare a termine il quantitativo
di lavoro a lui assegnato.
Il prossimo meeting di gruppo, sempre tramite piattaforma ”Discord” è fissato
per il giorno Lunedì 7 Dicembre 2020.