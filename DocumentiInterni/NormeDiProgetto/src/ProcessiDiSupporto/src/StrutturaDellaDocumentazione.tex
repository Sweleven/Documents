\subsection{Struttura della documentazione}
Ogni documento prodotto che verrà consegnato assieme al software \NomeProgetto{} verrà realizzato secondo la struttura spiegata nel corso di questa documentazione.


\subsubsection{Struttura prima pagina}
La prima pagina del documento è così formata:
\begin{itemize}
	\item logo del gruppo: il logo del gruppo {\Gruppo}, comprensivo di nome, è visibile sulla prima pagina ed è posizionato al centro della stessa;
	\item lista dei contributori: posizionata sotto il logo del gruppo, indica i nomi dei contributori dello specifico documento;

	\item tabella: presente sotto il titolo del documento, racchiude le seguenti informazioni:
	\begin{itemize}
		\item versione: indica la versione del documento;
		\item redattori: nome, cognome e matricola degli studenti incaricati della scrittura del documento in oggetto;
		\item approvatori: nome, cognome e matricola degli studenti incaricati all’approvazione del documento in oggetto;
		\item verificatori: nome, cognome e matricola degli studenti incaricati alla verifica del documento in oggetto;
		\item uso: marcatore d’uso del documento, che può essere interno o esterno;
		\item destinazione: indica i destinatari del documento in oggetto.
		\end{itemize}
	\end{itemize}
	
\subsubsection{Registro delle modifiche}
Ogni documento riporta una sezione denominata come sopracitato, recante una descrizione sommaria delle modifiche effettuate al documento. Il presente registro è formattato in modalità tabellare, dove sono riportate le seguenti voci:	

\begin{itemize}
	\item versione: identifica la versione del documento dopo la modifica;
	\item data di modifica: identifica la data di ultima modifica del documento;
	\item descrizione: sintetizza una descrizione della modifica apportata al documento;
	\item nominativo: identifica lo studente, tramite nome e cognome, che ha apportato la modifica;
	\item ruolo: identifica il ruolo dello studente che ha apportato la modifica.
\end{itemize}

\subsubsection{Indice}
Ogni documento prodotto dal gruppo {\Gruppo} contiene, all’inizio dello stesso, un indice cui scopo è quello di esplicitare la struttura e le gerarchie del documento in oggetto. 

\subsubsection{Elenco immagini}
I documenti muniti di figure saranno provvisti di un indice delle figure contenute in essi, con una breve descrizione della stessa ed il numero della pagina dove è collocata.


\subsubsection{Elenco tabelle}
I documenti muniti di tabelle, saranno provvisti di un indice delle tabelle contenente, per
ciascuna, una breve descrizione ed il numero della pagina dove è collocata.

\subsubsection{Struttura pagine di contenuto}
La struttura delle pagine di contenuto è la seguente:
\begin{itemize}
	\item logo: il logo del gruppo è posizionato in alto a sinistra;
	\item sezione: il nome ed il numero della sezione sono posizionati in centro alla pagina;
	\item intestazione: è l’intestazione del documento;
	\item contenuto: il contenuto principale del documento è posizionato dopo l’intestazione dello stesso;
	\item nome e versione del documento: presente in basso a sinistra, indica appunto il nome del documento in oggetto e la sua attuale versione di scrittura;

	\item numero pagina: presente in basso a destra, su ogni pagina ad esclusione della prima pagina.
\end{itemize}



