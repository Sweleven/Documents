\subsection{Struttura della documentazione}
Ogni documento prodotto che verrà consegnato assieme al software "nomesoftware" verrà realizzato secondo la struttura spiegata nel corso di questa documentazione.

\subsubsection{Struttura prima pagina}
La prima pagina del documento è così formata:
\begin{itemize}
	\item logo del gruppo: Il logo del gruppo SWEleven è visibile sulla prima pagina ed è posizionato al centro della stessa;
	\item gruppo e progetto: Il nome del gruppo è visibile immediatamente sotto il logo. In aggiunta, è presente anche la lista dei nomi dei contributori dello specifico documento;
	\item tabella: presente sotto il titolo del documento, racchiude le seguenti informazioni:
	\begin{itemize}
		\item versione: indica la versione del documento;
		\item redattori: nome, cognome e matricola degli studenti incaricati della scrittura del documento in oggetto;
		\item approvatori: nome, cognome e matricola degli studenti incaricati all’approvazione del documento in oggetto;
		\item verificatori: nome, cognome e matricola degli studenti incaricati della verifica del documento in oggetto;
		\item uso: marcatore d’ uso del documento, che può essere interno o esterno;
		\item destinazione: indica i destinatari del documento in oggetto;
		\end{itemize}
	\end{itemize}
	
\subsubsection{Registro delle modifiche}
Ogni documento riporta una sezione denominata con il nome sopracitato, recante una descrizione sommaria sulle modifiche effettuate al documento. Il presente registro è formattato in modalità tabellare, dove sono riportate le seguenti voci:	
\begin{itemize}
	\item versione: identifica la versione del documento dopo la modifica;
	\item data di modifica: identifica la data di ultima modifica del documento;
	\item descrizione: sintetizza una descrizione della modifica apportata al documento;
	\item nominativo: identifica lo studente, tramite nome e cognome, che ha apportato la modifica;
	\item ruolo: identifica il ruolo dello studente che ha apportato la modifica;
\end{itemize}

\subsubsection{Indice}
Ogni documento prodotto dal gruppo SWEleven contiene, all’inizio dello stesso, un indice, cui scopo è quello di esplicitare le parti gerarchiche che compongono il documento in oggetto. 

\subsubsection{Elenco immagini}
I documenti muniti di figure saranno provvisti di un indice delle figure contenute, con una breve descrizione della stessa ed il numero della pagina dove è collocata.

\subsubsection{Elenco tabelle}
I documenti muniti di tabelle, saranno provvisti di un indice delle tabelle contenente, per
ciascuna, una breve descrizione ed il numero della pagina dove è collocata.

\subsubsection{Struttura pagine di contenuto}
La struttura delle pagine di contenuto è la seguente:
\begin{itemize}
	\item logo: il logo del gruppo è posizionato in alto a sinistra;
	\item sezione: il nome e numero della sezione sono posizionati in centro alla pagina;
	\item intestazione: è l’intestazione del documento;
	\item contenuto: il contenuto principale del documento è posizionato dopo l’intestazione dello stesso;
	\item numero pagina: presente in basso a destra, su ogni pagina ad esclusione della prima pagina;
\end{itemize}



