\subsection{Norme tipografiche}
Il gruppo ha scelto di utilizzare la convenzione CamelCase con, in aggiunta, le seguenti regole:
\begin{itemize}
	\item nomi di file o cartelle dovranno necessariamente avere la prima lettera maiuscola;
	\item i nomi composti di file o cartelle verranno scritti uniti, lasciando le iniziali delle relative parole maiuscole;
	\item i nomi dei documenti utilizzeranno la seguente nomenclatura:
	\begin{center}
	\textbf{NomeDelFile\_v.X.Y.Z}
	\end{center}
	dove X,Y e Z indicano la versione del documento;
\end{itemize}

\subsubsection{Elenchi puntati}
Tutti gli elenchi puntati nelle documentazioni prodotte dal gruppo {\Gruppo} rispetteranno le seguenti norme sintattiche:
\begin{itemize}
	\item la frase che precede il primo elemento dell’elenco puntato, o se l’elemento precede a sua volta un sottoelenco, dovrà terminare con il carattere “:”;
	\item ogni elemento dell’elenco deve cominciare con la lettera minuscola, ad eccezione dei nomi propri;
	\item ogni elemento dell’elenco deve terminare con il carattere “;”;
	\item l’ultimo elemento dell’elenco terminerà con il carattere “.”.
\end{itemize}

\subsubsection{Glossario}
Per eliminare ambiguità linguistiche e formattare automaticamente le parole chiavi più importanti, è stato creato un glossario in \LaTeX{} con nomenclatura nomegruppo.sty.

\subsubsection{Sigle}
In tutti i documenti prodotti dal gruppo {\Gruppo}, si adotteranno le seguenti sigle per abbreviazione dei nomi più ricorrenti, quali:
\begin{itemize}
	\item AdR: indica l’analisi dei requisiti;
	\item NdP: indica le norme di progetto;
	\item PdP: indica il piano di progetto;
	\item PdQ: indica il piano di qualifica;
	\item SdF: indica lo studio di fattibilità;
	\item RR: indica la Revisione dei Requisiti;
	\item RP: indica la Revisione di Progettazione;
	\item RQ: indica la Revisione di Qualifica;
	\item RA: indica la Revisione di Accettazione.
\end{itemize}

\subsubsection{Tabelle}{\Gruppo}
Il gruppo {\Gruppo}, durante la stesura della documentazione necessaria al progetto, potrebbe far ampio uso di tabelle riassuntive, per dare un carattere meno prolisso alla documentazione e facilitare la lettura della stessa.
Tutte le tabelle saranno costruite secondo i seguenti standard:
\begin{itemize}
	\item l’header della tabella, avrà sfondo blu e colore di background bianco;
	\item la prima parola di ogni cella avrà la lettera maiuscola;
	\item per facilitare la lettura tra più righe, si differenzieranno le righe con indice pari da quelle dispari, alternando ciclicamente il colore bianco e grigio;
	\item la tabella sarà in posizione centrale rispetto al documento.
\end{itemize}

L’esempio sotto riportato riporta una tabella costruita secondo le disposizioni sopra-descritte:

    \rowcolors{2}{\evenRowColor}{\oddRowColor}
    \renewcommand{\arraystretch}{1.5}
    \centering
    \begin{longtable}{ c c  C{4cm}  c  c }
        \rowcolor{\primaryColor}
        \textcolor{\secondaryColor}{
        \textbf{Header 1}}     & \textcolor{\secondaryColor}{\textbf{Header 2}}       & \textcolor{\secondaryColor}
        {\textbf{Header 3}} & \textcolor{\secondaryColor}{\textbf{Header 4}} & \textcolor{\secondaryColor}{\textbf{Header 5}}                          \\
        Testo esempio                 & Testo esempio                                    & Testo esempio                                & Testo esempio & Testo esempio{} \\
        Testo esempio                & Testo esempio                                    & Testo esempio                                   & Testo esempio & Testo esempio{} \\
        Testo esempio                & Testo esempio                                    & Testo esempio                          & Testo esempio & Testo esempio{}    \\
    \end{longtable}


\subsubsection{Immagini}
Tutte le immagini utilizzate durante la stesura della documentazione, dovranno essere collocate in una organizzazione gerarchica all’interno delle cartelle e sottocartelle della documentazione. In particolare, ogni immagine sarà salvata dentro la specifica cartella del documento a cui appartiene. Un esempio di nomenclatura utilizzata per le immagini sarà la seguente:
\textbf{N\_nn\_\#\_nomeimg.estensione}\\
dove:
\begin{itemize}
	\item N indicherà la sezione dove verrà inserita l’immagine;
	\item n indicherà la sottosezione dove verrà inserita l’immagine;
	\item # indicherà un numero intero naturale progressivo, che sarà l’indice dell’immagine;
	\item nomeimmagine indicherà il nome proprio dell’immagine;
	\item estensione indicherà il formato dell’immagine (ex: jpg, png…).
\end{itemize}
Ogni immagine, in aggiunta, sarà correlata da una breve descrizione della stessa, posizionata in centro pagina.

\subsubsection{Diagrammi UML}
Ogni diagramma uml inserito all’interno della documentazione avrà estensione png. Il programma atto alla creazione dei diagrammi uml sarà starUML





