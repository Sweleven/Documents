\subsection{Scopo del documento}
Questo documento ha lo scopo di essere utilizzato come linea guida per svolgere le attività nell'intero ciclo di vita del progetto.
Al suo interno vengono quindi dichiarate le norme, le tecnologie e gli strumenti che il gruppo \Gruppo{} intende utilizzare.
Ogni membro del gruppo è obbligato a tenere in considerazione questo documento al fine di garantire la massima coerenza del materiale prodotto.
	
\subsection{Scopo del prodotto}
Lo scopo del prodotto è quello di realizzare un’applicazione in grado di segnalare ad un server
dedicato la presenza di un utente su una determinata postazione e controllare in ogni momento
se questa è libera, disabilitata, occupata, prenotata oppure da pulire. L’applicazione dovrà essere
installata in uno smartphone \glo{Android} e dovrà poter essere utilizzata impiegando
dei tag \glo{RFID}.

\subsection{Glossario}
Al fine di evitare ambiguità fra i termini, e per avere chiare fra tutti gli stakeholder le terminologie utilizzate per la realizzazione del presente documento, il gruppo \Gruppo{} ha redatto un documento denominato \Gv{}.
In tale documento, sono presenti tutti i termini tecnici, ambigui, specifici del progetto e scelti dai membri del gruppo con le loro relative definizioni.
Un termine presente nel \Gv{} e utilizzato in questo documento viene indicato con un pedice \glo{} alla fine della parola.

\subsection{Riferimenti} 
\subsubsection{Normativi}
\begin{itemize}
	\item \textbf{Capitolato d'appalto C1:} \url{https://www.math.unipd.it/~tullio/IS-1/2020/Progetto/C1.pdf}
	\item \textbf{ISO\_1227--1995:} \url{https://www.math.unipd.it/~tullio/IS-1/2009/Approfondimenti/ISO_12207-1995.pdf}
\end{itemize}

\subsubsection{Informativi}
\begin{itemize}
	\item \textbf{\LaTeX:} \url{https://www.latex-project.org/help/documentation/}
	\item \textbf{Jira:} \url{https://www.atlassian.com/it/software/jira}
	\item \textbf{GitHub:} \url{https://github.com/}
	\item \textbf{GitKraken:} \url{https://www.gitkraken.com/}
\end{itemize}