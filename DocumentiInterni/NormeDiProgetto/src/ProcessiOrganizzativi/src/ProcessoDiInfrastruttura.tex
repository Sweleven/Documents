\subsection{Processo di infrastruttura}
Il processo di infrastruttura comprende tutte le indicazioni per stabilire e mantenere l'infrastruttura richiesta per ogni altro processo. L'infrastruttura può riguardare hardware, software, strumentistica, tecniche, standard, strutture per lo sviluppo, esecuzione o manutenzione.
Tale processo prevede i seguenti step:

\begin{enumerate}
    \item Implementazione del processo
    \item Istituzione dell'infrastruttura
    \item Manutenzione dell'infrastruttura
\end{enumerate}

\subsubsection{Implementazione del processo}

\myparagraph{}L'infrastruttura deve essere definita e documentata per coniugarsi con i requisiti del processo, considerando le procedure coinvolte, gli standard, la sistemistica e le tecniche di sviluppo e e manutenzione.

\paragraph{} L'istituzione dell'infrastruttura deve essere documentata.

\subsubsection{Istituzione dell'infrastruttura}

\myparagraph{}
La configurazione dell'infrastruttura deve essere pianificata e documentata. I criteri per la scelta della configurazione devono includere le funzionalità a disposizione, le performance, la sicurezza, la disponibilità (in particolare di spazio e di risorse), l'equipaggiamento correlato, i costi. Vanno inoltre considerati i requisiti legati alle tempistiche (correlati alle performance).

In particolare l'infrastruttura indicata per il progetto corrente non riguarda prodotti hardware ma si limita ai software associati alle attività in seguito riportate:

\textbf{Comunicazione e meeting}

Di seguito  vengono illustrati gli strumenti software per lo svolgimento di riunioni telematiche e le comunicazioni rapide. Nello specifico, le comunicazioni vengono divise in due macrocategorie, ovvero le comunicazioni interne al gruppo e le comunicazioni con enti e persone esterne ad esso.

\textbf{Comunicazioni interne al gruppo}

 Per la realizzazione di videoconferenze telematiche, verrà utilizzato l'apposito server privato "SWEG11" sulla piattaforma di VoIP Discord.
Per le comunicazioni testuali sono inoltre stati predisposti tre canali testuali interni a tale server:

\begin{itemize}
    \item Risorse: contiene link a materiale utile finalizzato all'autoapprendimento di eventuali argomenti non noti ai membri del gruppo. Include inoltre vari link ai siti dedicati alla presentazione dei capitolati e  alle specifiche indicate dal professore.
    \item Generale: canale testuale realizzato di default dalla piattaforma utilizzata, contiene i link ai files generati per i sondaggi interni al gruppo e la tabella delle disponibilità di ogni membro del gruppo. 
    \item Chat-riunione: contiene i link e i files inviati durante le videoconferenze.
\end{itemize}

Per le comunicazioni rapide ed eventuali discussioni o proposte esterne alle riunioni viene utilizzato l'apposito gruppo privato sulla piattaforma di messaggistica istantanea Telegram.

\textbf{Comunicazioni esterne al gruppo}

/*Da stabilire con l'azienda ed eventualmente il professore in seguito all'assegnazione del capitolato.*/

\textbf{Controllo delle versioni,issue tracking system e memorizzazione remota condivisa}

Per il controllo delle versioni del prodotto sviluppato e la possibilità di memorizzare documenti in una repository remota verrà utilizzato GitHub (si rimanda all'apposita sezione "Classificazione dei documenti" nei processi di supporto).

\subsubsection{Manutenzione dell'infrastruttura}

\myparagraph{}L'infrastruttura deve essere mantenuta, monitorata e modificata se necessario per poter mantenere il requisito di appagamento dei requisiti del processo. Ciò deve dunque includere una definizione del grado di gestione della configurazione.
