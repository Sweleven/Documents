\subsection{Processo di perfezionamento}
Il processo di perfezionamento riguarda l'istituzione, la valutazione,il controllo e il miglioramento del ciclo di vita del software. Il processo segue le seguenti fasi:

\begin{enumerate}
    \item istituzione del processo;
    \item controllo del processo;
    \item miglioramento del processo.
\end{enumerate}

\subsubsection{Istituzione del processo}
Il team deve stabilire una suite di processi organizzativi per tutto il ciclo di vita del software. I processi e le loro applicazioni a casi specifici devono essere documentate nei documenti pubblicati dall'organizzazione. Inoltre va stabilito un sistema controllo sul processo per lo sviluppo, il monitoraggio ed eventuali miglioramenti.

\subsubsection{Controllo del processo}

\myparagraph{Valutazione del processo} Essa deve essere stabilita, sviluppata, documentata e applicata. I risultati delle valutazioni devono essere memorizzati.

In particolare, nel progetto corrente verrà assegnato un grado di rischio a ciascuna epic basato su tre criteri presentati secondo il seguente schema (critico, medio, favorevole):

\begin{itemize}
    \item quantità di personale preposto allo svolgimento di tale incarico (1, 2, >2);
    \item tempo a disposizione per la conclusione di tale incarico ($\leq$7 giorni, tra 7 giorni e 14 giorni, $\geq$14 giorni);
    \item complessità stabilita dal team nelle videoriunioni (alta, media, bassa).
\end{itemize}

Il grado di rischio delle Epic prevede 5 livelli:
\begin{enumerate}
    \item[\textbf{5.}] Highest: tutti e tre i valori sono critici;
    \item[\textbf{4.}] High: al massimo due dei tre criteri presentano valori critici;
    \item[\textbf{3.}] Medium: al massimo un criterio presenta valori critici;
    \item[\textbf{2.}] Low: non sono previsti valori critici;
    \item[\textbf{1.}] Lowest: non sono presenti valori critici e almeno uno dei tre parametri di valutazione del rischio presenta il valore favorevole.
\end{enumerate}

Si consiglia di pianificare le attività di progetto in modo tale da mantenere un valore medio di rischio per ogni task. La codifica di rischio nelle epic va aggiornata in accordo con i criteri precedentemente presentati. Qualora venisse raggiunto un grado di rischio superiore al medio si prospetta un'analisi più accurata della situazione con annessi eventuali modifiche al piano relativo a tale task.

\myparagraph{Controllo}Il team deve pianificare ed eseguire revisioni periodiche dei processi per assicurarsi che l'adeguatezza ed efficacia dei processi in questione rimanga tale nell'ottica dei risultati dei risultati delle valutazioni.

\subsubsection{Miglioramento del processo}

\myparagraph{Analisi}Il team di sviluppo deve apportare dei miglioramenti ai propri processi qualora si rivelasse necessario secondo i risultati della valutazione del processo. La documentazione relativa al processo deve essere aggiornata in maniera tale da poter evidenziare eventuali miglioramenti nei processi organizzativi.

\myparagraph{Valutazione e miglioramento}I dati memorizzati (valutazioni, dati archiviati e dati tecnici) devono essere conservati e analizzati per poter comprendere eventuali debolezze e punti di forza dei processi svolti. Le analisi verranno utilizzate come feedback per migliorare tali processi ed eventualmente per indicare dei cambiamenti utili all'interno del modo di operare all'interno del medesimo progetto o in progetti successivi.
