\subsection{Processo di apprendimento}
Il processo di apprendimento riguarda include tutte le specifiche per formare e riqualificare il personale. Tale processo si articola nelle tre fasi seguenti:

\begin{enumerate}
    \item Implementazione del processo
    \item Sviluppo del materiale per l'apprendimento
    \item Implementazione di un piano per l'apprendimento
\end{enumerate}

\subsubsection{Implementazione del processo}Tale fase prevede le seguenti azioni:

\myparagraph{}\`E consigliato svolgere un'analisi dei requisiti del progetto per stabilire eventuale materiale utile al personale per acquisire o sviluppare le competenze richieste. \`E consigliato stabilire dei livelli di complessità di apprendimento a seconda della categoria di personale da  formare tramite un piano di apprendimento, il quale deve contenere il riferimento alla categoria di personale per cui è sviluppato e le risorse di cui esso necessita (si consiglia di documentare lo sviluppo di tale piano). 

\subsubsection{Sviluppo del materiale per l'apprendimento}

\myparagraph{}Si consiglia lo sviluppo di un manuale per l'apprendimento

\subsubsection{Implementazione di un piano per l'apprendimento}

\myparagraph{}Il piano di insegnamento deve essere implementato al fine di formare il personale. Le sessioni di insegnamento devono essere memorizzate.

\myparagraph{}\`E fortemente consigliato assicurare un buon quantitativo di personale competente per le attività pianificate in tempo rapido.

In particolare il gruppo a cui queste norme sono riferite utilizzerà un approccio di apprendimento di tipo individuale e autonomo. Per l'apprendimento autonomo si consiglia:
\begin{itemize}
    \item  https://www.sololearn.com/
    \item eventuali guide e siti ufficiali dell'oggetto di apprendimento
\end{itemize}