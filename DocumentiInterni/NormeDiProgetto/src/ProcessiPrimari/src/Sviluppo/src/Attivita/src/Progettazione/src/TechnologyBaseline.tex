La technology baseline descrive ed eventualmente motiva le decisioni prese dal gruppo \Gruppo{} in merito ai seguenti temi:

\begin{itemize}
    \item \textbf{Tecnologie utilizzate}: descrizione dettagliata delle tecnologie utilizzate, indicando linguaggi,  \glo{framework} e \glo{librerie} utilizzati;
    \item \textbf{Diagrammi dei casi d'uso }: per individuare i requisiti funzionali che descrivono le interazioni tra l'utente e il sistema;
    \item \textbf{Tracciamento delle componenti}: associazione tra requisiti e componenti che li soddisfano;
    \item \textbf{Test di integrazione}: descrizione dei test eseguiti per verificare che i componenti interagiscano in maniera corretta tra di loro per soddisfare i requisiti.
\end{itemize}

La technology baseline deve concretizzarsi nella realizzazione di un \emph{\glo{PoC}} che ha la funzione di mostrare la capacità di autoapprendimento e di applicazione delle tecnologie il cui utilizzo è stato concordato da tutti i membri del gruppo. 
