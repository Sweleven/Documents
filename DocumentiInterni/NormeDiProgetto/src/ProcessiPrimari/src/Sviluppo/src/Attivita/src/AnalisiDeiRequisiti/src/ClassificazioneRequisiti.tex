Tutti i requisiti individuati dagli analisti devono essere univocamente identificati sendondo il seguente schema identificativo:

\begin{center}
    \textbf{R[Priorit\`{a}]-[Tipologia]-[Identificativo]}
\end{center}

Dove:

\begin{itemize}
    \item \textbf{R}: indica che si tratta di un requisito;
    \item \textbf{Priorit\`{a}}: assume, a seconda del grado di priorit\`{a} i valori:
    \begin{itemize}
        \item \textbf{O}: obbligatorio, cio\`{e} strettamente necessario;
        \item \textbf{D}: desiderabile, cio\`{e} non strettamente necessario;
        \item \textbf{F}: facoltativo, cio\`{e} contrattabile in corso d'opera;
    \end{itemize}
    \item \textbf{Tipologia}: assume, a seconda del tipo di requisito, i valori:
    \begin{itemize}
        \item \textbf{F}: funzionale;
        \item \textbf{P}: prestazionale;
        \item \textbf{Q}: qualitativo;
    \end{itemize}
    \item \textbf{Identificativo}: numero progressivo per contraddistinguere il requisito, in forma gerarchica padre-figlio secondo la seguente struttura:
    \begin{center}
        \textbf{[CodicePadre].[CodiceFiglio]}
    \end{center}
\end{itemize}