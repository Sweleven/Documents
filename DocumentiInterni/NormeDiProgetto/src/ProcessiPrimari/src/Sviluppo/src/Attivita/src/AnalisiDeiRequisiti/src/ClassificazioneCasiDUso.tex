Tutti i casi d'uso individuati dagli analisti devono essere univocamente identificati secondo il seguente schema identificativo:

\begin{center}
    \textbf{UC[CodiceCaso]}
\end{center}

Dove:

\begin{itemize}
    \item \textbf{UC}: indica che si tratta di un caso d'uso;
    \item \textbf{CodiceCaso}: numero progressivo per contraddistinguere il caso d'uso, in forma gerarchica padre-figlio secondo la seguente struttura:
    \begin{center}
        \textbf{[CodicePadre].[CodiceFiglio]}
    \end{center}
\end{itemize}

I casi d'uso devono essere opportunamente corredati da:

\begin{itemize}
    \item \textbf{Titolo}: un titolo descrittivo;
    \item \textbf{Descrizione}: una breve descrizione;
    \item \textbf{Attori primari}: indicazione degli attori principali coinvolti;
    \item \textbf{Attori secondari}: indicazione degli attori secondari coinvolti;
    \item \textbf{Scopo}: scopo del caso d'uso;
    \item \textbf{Precondizioni}: condizioni che devono essere verificate affinch\'{e} gli eventi descritti si verifichino;
    \item \textbf{Postcondizioni}: condizioni che devono essere verificarsi alla conclusione degli eventi descritti; 
    \item \textbf{Flusso principale}: flusso principale degli eventi, in forma di elenco numerato, con eventuale riferimento ad ulteriori casi d'uso;
    \item \textbf{Scenario alternativo}: indicazione di uno scenario alternativo \emph{(se presente)};
    \item \textbf{Estensioni}: indicazione delle estensioni \emph{(se presenti)};
    \item \textbf{Inclusioni}: indicazione delle inclusioni \emph{(se presenti)}.
\end{itemize}