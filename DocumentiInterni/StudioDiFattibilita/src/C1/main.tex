
\subsection{Infomazioni generali}
Il capitolato in questione si chiama "BlockCOVID: supporto digitale al contrasto della pandemia", il proponente è l'azienda Imola Informatica ed i committenti sono il Prof. Tullio Vardanega ed il Prof. Riccardo Cardin.

\subsection{Descrizione del capitolato}
Nel contesto della continua diffusione della pandemia di COVID-19, l'obiettivo di questo capitolato riguarda il tracciamento delle postazioni nell'ambito di un laboratorio informatico. In particolare questo progetto richiede la creazione di una piattaforma che sia in grado di gestire e tracciare efficacemente l'utilizzo delle postazoni di lavoro, così come la pulizia delle stesse.

\subsection{Prerequisiti}
\begin{itemize}
\item competenza nello sviluppo di applicazionei mobile;
\item competenza nella programmazione e strutturazione di applicazioni server-side;
\item competenza nella costruzione e configurazione di comunicazioni di rete;
\item conoscenza adeguata delle tecnologie e librerie proposte;
\item conoscenza riguardo il funzionamento di strumenti di riconoscimento fisico, quali \glo{RFID}.
\end{itemize}

\subsection{Tecnologie coinvolte}
\begin{itemize}
\item \glo{RFID}: tecnologia a radiofrequenza per l'identificazione o trasmissione di informazioni;
\item \glo{Java}, \glo{Python} o \glo{Node.js} per lo sviluppo del server back-end;
\item protocolli asincroni per le comunicazioni app mobile-server;
\item sistema \glo{Blockchain} per salvare con opponibilità a terzi i dati di sanificazione;
\item piattaforme \glo{IAAS} o \glo{PAAS}, come \glo{Kubernetes}, \glo{Openshift} o \glo{Rancher}, per il rilascio delle componenti del server e la gestione della scalabilità orizzontale;
\item \glo{Docker} per la \glo{containerization};
\item \glo{API Rest} oppure \glo{gRPC} attraverso le quali è possibile utilizzare l'applicativo.
\end{itemize}

\subsection{Vincoli}
\begin{itemize}
\item identificazione del soggetto e dello spazio fisico di lavoro tramite tag \glo{RFID};
\item la tracciatura deve avvenire in tempo reale ed essere memorizzata, immutabile e certificata;
\item realizzazione di un server centralizzato, corredato di una \glo{UI} con una procedura di autenticazione ed attraverso il quale deve essere possibile:
\begin{itemize}
\item gestire più stanze, postazioni, personale ed utenti, con possibilità di creare, modificare ed eliminare gli stessi;
\item sapere in ogni momento se la stanza o postazione è occupata, prenotata, da pulire o pulita (la visualizzazione deve seguire uno schema che mostri la stanza con le postazioni di diverso colore a seconda del relativo stato);
\item monitorare il numero di dipendenti presenti in tutte le postazioni e nella stanza nel suo complesso;
\item bloccare le prenotazioni per una determinata stanza;
\item prenotare una postazione con granularità di 1 ora;
\item effettuare ricerche sugli accessi e sulle postazioni occupate da uno specifico dipendente.
\end{itemize}
\item sviluppare un'applicazione mobile per dispositivi con sistema \glo{Android} oppure \glo{iOS} che deve permettere di:
\begin{itemize}
\item recuperare una lista delle postazioni libere e acceso alle informazioni delle stesse;
\item prenotare una postazione;
\item segnalare la propria presenza in tempo reale (nella postazione, appoggiando il telefono sul tag \glo{RFID}), così come la pulizia di una postazione o dell'intera stanza;
\item ricevere  un  elenco  delle  stanze  che  necessitano l’igienizzazione;
\item recuperare uno storico delle postazioni occupate e igienizzate.
\end{itemize}
\item le autorizzazioni di accesso si devono distinguere in due macro-tipologie di soggetti: Amministratore di sistema e Utente;
\item poichè il lettore \glo{RFID} consuma molta batteria al dispositivo è richiesto di trovare il tempo sufficiente a garantire il giusto bilanciamento tra consumo batteria e scansioni, il tutto con test e report delle scelte fatte;
\item avere le componenti applicative corredate da test unitari e d’integrazione. Inoltre, è richiesto che il sistema venga testato nella sua interezza tramite test \glo{end-to-end}, con copertura all'80 percento, correlati da report;
\item realizzazione di un'adeguata documentazione.
\end{itemize}

L’azienda proponente valuta inoltre positivamente i seguenti requisiti opzionali:
\begin{itemize}    
\item possibilità per il server di esportare un rapporto tabellare delle ore trascorse da un utente e/o delle pulizie, per ogni postazione o stanza;
\item possibilità per il server di mostrare le prenotazioni in una vista a calendario;
\item possibilità per l'applicazione di ricevere il nome di una postazione igienizzata e libera in una determinata stanza; 
\item cifrare tutte le comunicazioni fra applicazione e server in modo tale da garantire la validità delle informazioni;
\item fornire un’analisi rispetto al carico massimo supportato in numero di utenti e di quale sarebbe il servizio cloud più adatto per supportarlo analizzando prezzo, stabilità  del servizio ed assistenza.
\end{itemize}

\subsection{Aspetti positivi}
\begin{itemize}
\item le tecnologie coinvolte sono nuove e stimolanti ed inoltre la loro conoscenza offre diverse opportunità in ambito professionale;
\item alcune soluzioni suggerite offrono un interessante grado di sfida;
\item l'azienda è stata piuttosto chiara e precisa sugli obiettivi ed i vincoli del progetto;
\item il proponente offre consulenza e supporto tecnico per la realizzazione;
\item il contrasto della diffusione della pandemia è un tema d'importanza rilevante in questo periodo;
\item molti membri del gruppo si sono dimostrati incuriositi da questo capitolato.

\end{itemize}

\subsection{Aspetti critici}
\begin{itemize}
\item seppur accattivanti le tecnologie coinvolte sono molte ed anche relativamente recenti e complesse: richiedono tempo per lo studio ed apprendimento;
\item il grado di sfida proposto è discretamente alto;
\item il capitolato ha suscitato l'interesse di molti gruppi partecipanti, potrebbe quindi essere difficile aggiudicarsi l'appalto.
\end{itemize}

\subsection{Conclusioni}
Questo capitolato ha attirato subito l'attenzione di molti membri del gruppo. Seppure non di semplice svolgimento, le tecnologie che coinvolge sono stimolanti e di grande interesse perchè molto usate al giorno d'oggi ed offrono un bagaglio di conoscenze molto utili ed apprezzate in campo professionale. Pertanto questo progetto è stato selezionato come prima scelta da parte del gruppo.
