\subsection{Informazioni generali}
Il capitolato in esame è intitolato "PORTACS: piattaforma di controllo mobilità autonoma", proposto dall'azienda Sanmarco Informatica mentre i committenti sono il Prof. Tullio Vardanega ed il Prof. Riccardo Cardin.
\subsection{Descrizione del capitolato}
L'azienda propone la creazione di un motore di elaborazione in tempo reale, che gestisca delle entità guidandole attraverso percorsi evitando collisioni ed eventualmente trovando i percorsi più convenienti.
Il sistema centrale deve indicare ad ogni entità la prossima mossa per raggiungere il \glo{POI}, tenendo conto della mappa fornita con tutte le sue caratteristiche (limiti di corsie, sensi unici..).
\subsection{Tecnologie Coinvolte}
\begin{itemize}
    \item Git;
    \item container software;
    \item diagrammi UML.
\end{itemize}
\subsection{Vincoli}
\begin{itemize}
    \item le unità devono evitare le collisioni;
    \item le unità devono continuamente inviare la posizione, direzione e stato al sistema centrale;
    \item deve essere presente una UI dove vengono mostrate le mosse suggerite per ogni unità.
\end{itemize}
\subsection{Aspetti positivi}
\begin{itemize}
    \item formazione di competenze in ambito di real-time monitoring and analysis;
    \item il capitolato introduce alle problematiche del mondo della logistica, ambito molto informatizzato;
    \item non è richiesta l'implementazione di algoritmi di ricerca operativa, che complicherebbero di molto lo sviluppo dell'algoritmo.
\end{itemize}
\subsection{Aspetti critici}
\begin{itemize}
    \item l'intero sistema deve essere simulato, portando un grande lavoro aggiuntivo per simulare la raccolta dati di sensori;
    \item non sono state indicate tecnologie consigliate;
    \item non sono stati indicati vincoli che possano guidare all'implementazione.
\end{itemize}
\subsection{Conclusioni}
Il capitolato introduce molto bene la problematica da risolvere, ma non spiega le tecnologie da utilizzare e come eseguire l'implementazione.
Non ha catturato l'attenzione del gruppo dato il fatto che tutto il sistema deve essere simulato, portando poca concretezza al progetto.
È inoltre poco correlata l'implementazione per controllare auto a guida automona, camerieri robot, e muletti autonomi, non rendendo chiara la modalità di pensiero che si deve avere durante lo svolgimento del capitolato.