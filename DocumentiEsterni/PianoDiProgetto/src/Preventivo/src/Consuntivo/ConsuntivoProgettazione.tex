\subsubsection{Bilancio investimento fase di progettazione}
Il bilancio della fase di progettazione è positivo,il gruppo ha impiegato meno ore di quelle preventivate inzialmente,ma sono stati riscontrati diversi problemi.\\
La seguente tabella illustra la differenza oraria ed economica rilevata a posteriori.
{
\rowcolors{2}{\evenRowColor}{\oddRowColor}
\renewcommand{\arraystretch}{2}
\begin{longtable}[h]{ C{2.5cm} C{2cm} C{1.8cm} C{2.2cm} C{1.5cm} C{2.3cm}}
\caption{Tabella del costo complessivo per ruolo}\\
\rowcolor{\primaryColor}

\textcolor{\secondaryColor}{\textbf{Ruolo}} & 
\textcolor{\secondaryColor}{\textbf{Ore preventivate}} & 
\textcolor{\secondaryColor}{\textbf{Variazione oraria}} & 
\textcolor{\secondaryColor}{\textbf{Costo preventivato (in \euro{})}} & 
\textcolor{\secondaryColor}{\textbf{Costo effettivo (in \euro{})}} & 
\textcolor{\secondaryColor}{\textbf{Variazione di costo (in \euro{})}}\\	
	
Responsabile    &  21 & 0 & 630 & 630 &  0 \\
Amministratore  &  20 & 0 & 400 & 400 & 0 \\
Analista        & 24 & +16 & 600 & 1000 & +400 \\
Progettista     &  49 & 9 & 1078 & 880 & -198 \\
Programmatore   &   58 & -12 & 870 &  690 & -180 \\
Verificatore    &  59 &  -10 & 885 & 735 & -150 \\
\textbf{Totale} & 231 & 20 & 4575 & 4400 & -128 \\	

\end{longtable}
}

\subsubsection{Conclusioni}
Come riportato dalla tabella, il bilancio risulta essere positivo per i seguenti motivi:
\begin{itemize}
	\item \textbf{Responsabile}: {essendo il team abbastanza coeso la maggior parte del tempo è stato dedicato al PdP;}
	\item \textbf{Amministratore}: {la collaborazione tra membri del team in questo ruolo è risultata soddisfaciente;}
	\item \textbf{Analista}: {Dopo l'aggiudicazione del capitolato il team si è reso conto che la documentazione presente nell'AdR presentava diverse lacune, ciò ha portato a un incremento orario per tale ruolo;}
	\item \textbf{Progettista}: {Grazie all'esperienza dei membri e alla chiarezza dei requisiti si è riuscito a mitigare il numero di ore neccessarie a progettare il PoC, ammortizzando il costo orario dell'analista;}
	\item \textbf{Programmatore}: {Grazie all'esperienza dei membri e alla chiarezza dei requisiti si è riuscito a mitigare il numero di ore neccessarie allo sviluppo del PoC, ammortizzando il costo orario dell'analista;}
	\item \textbf{Verificatore}: {Avendo più esperienza con le tecnologie richieste(alla documentazione in particolare), il numero di ore è inferiore a quelle preventivate.}
\end{itemize}
La suddivisione a stadi di una fase permette di migliorare la produttività del team incrementalmente, visto che viene effettuato un check intermedio sostanziale alla fine di uno stadio:
\begin{itemize}
	\item Dai \textbf{verificatori}, che validano i deliverables relativi a quello stadio;
	\item E dal \textbf{responsabile}, che verifica la presenza di eventuali problematiche e si attiva nel risolverle.
\end{itemize}
L'esperienza e la coesione maturata saranno cruciale nel cercare di mitigare il numero di ore richieste,visti i scostamenti orari in particolare dell'analista rispetto al preventivo iniziale, che hanno portato ad una riduzione oraria complessiva di ciascun membro.
Visti i problemi riscontrati nel bilancio della fase di progettazione, il costo totale del preventivo a finire rimarrà invariato, per mitigare eventuali problemi di pianificazione oraria(come quello riscontrato in tale fase).
