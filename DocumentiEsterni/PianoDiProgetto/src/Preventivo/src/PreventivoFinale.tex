\subsection{Preventivo finale} 
Nel preventivo riportiamo la spesa totale che il committente dovrà affrontare, derivata dal totale delle ore rendicontate e preventivate nelle fasi di Progettazione Architetturale, Progettazione di Dettaglio e Codifica, Validazione e Collaudo.

\subsubsection{Divisione oraria complessiva} 
La seguente tabella rappresenta la distribuzione oraria dei ruoli per ogni componente del gruppo:
{
	\rowcolors{2}{\evenRowColor}{\oddRowColor}
\renewcommand{\arraystretch}{2}
\begin{longtable}[h!] { C{4cm} C{1cm} C{1cm} C{1cm} C{1cm} C{1cm} C{1cm} C{3cm}}
\caption{Tabella della divisione oraria complessiva}	\\
\rowcolor{\primaryColor}

\textcolor{\secondaryColor}{\textbf{Membro del gruppo}} & 
\textcolor{\secondaryColor}{\textbf{RE}} & 
\textcolor{\secondaryColor}{\textbf{AM}} & 
\textcolor{\secondaryColor}{\textbf{AN}} & 
\textcolor{\secondaryColor}{\textbf{PT}} & 
\textcolor{\secondaryColor}{\textbf{PR}} & 
\textcolor{\secondaryColor}{\textbf{VE}} & 
\textcolor{\secondaryColor}{\textbf{Ore complessive}}\\	
\endhead

\AD{}                     &  - &  - &  - & - & - & - & - \\
\AT{}                     &  - &  - &  - & - & - & - & - \\
\AW{}                     &  - &  - &  - & - & - & - & - \\
\EC{}                     &  - &  - &  - & - & - & - & - \\
\EM{}                     &  - &  - &  - & - & - & - & - \\
\FP{}                     &  - &  - &  - & - & - & - & - \\
\GG{}                     &  - &  - &  - & - & - & - & - \\
\textbf{Ore totali ruolo} & X & X & X & - & - & X & X \\
\end{longtable}
}

\subsubsection{Costo complessivo per ruolo}
Nella seguente tabella viene illustrato il monte ore risultante per ogni ruolo con il costo ad esso associato:
{
\rowcolors{2}{\evenRowColor}{\oddRowColor}
\renewcommand{\arraystretch}{2}
\begin{longtable}{ C{3cm} C{2cm} C{4cm}}
\caption{Tabella del costo complessivo per ruolo}\\
\rowcolor{\primaryColor}

\textcolor{\secondaryColor}{\textbf{Ruolo}} & 
\textcolor{\secondaryColor}{\textbf{Totale ore}} & 
\textcolor{\secondaryColor}{\textbf{Costo ruolo (in \euro{})}}\\	
\endhead
        
Responsabile   &  x & x \\
Amministratore &  x & x \\
Analista       &  x & x \\
Progettista    &  x & x \\
Programmatore  &  x & x \\
Verificatore   &  x & x \\
        	
\end{longtable}
}

% La quantità di ore totali per ciascun ruolo (rendicontate e non) viene rappresentata nel seguente areogramma:
% \begin{center}
% 	\begin{tikzpicture}
% 		\pie[rotate = 180, color={blue, yellow, red, green, grigetto, orange}] {
% 			7/Responsabile, % 75/1056 circa 7%
% 			10/Amministratore, % 107/1056 circa 10%
% 			14/Analista, % 152/1056 circa 14%
% 			17/Progettista, % 181/1056 circa 17%
% 			22/Programmatore, % 227/1056 circa 22%
% 			30/Verificatore % 314/1056 circa 30%
% 		}
% 	\end{tikzpicture}
% \end{center}

Nel seguente areogramma viene rappresentata la distribuzione dei costi in percentuale sulla spesa totale da affrontare:
\begin{center}
	\begin{tikzpicture}
		\pie[rotate = 180, color={blue, yellow, red, green, pink, orange}] {
			7/Responsabile, % 56/816 circa 7%
			8/Amministratore, % 69/816 circa 8%
			5/Analista, % 38/816 circa 5%
			22/Progettista, % 181/816 circa 22%
			28/Programmatore, % 227/816 circa 28%
			30/Verificatore % 245/816 circa 30%
		}
	\end{tikzpicture}
\end{center}

\subsubsection{Costo complessivo}
Nella seguente tabella vengono riportati i costi complessivi delle varie fasi e infine l'importo proposto da \Gruppo{} per la realizzazione del progetto \NomeProgetto{}:\\
{
\rowcolors{2}{\evenRowColor}{\oddRowColor}
\renewcommand{\arraystretch}{2}
\begin{longtable}{ C{5cm} C{5cm}}
\caption{Tabella del costo complessivo}\\
\rowcolor{\primaryColor}

\textcolor{\secondaryColor}{\textbf{Fase}} &
\textcolor{\secondaryColor}{\textbf{Costo fase (in \euro{})}}\\	
\endhead
		
Progettazione Architetturale          &  x \\
Progettazione di Dettaglio e Codifica &  x \\
Validazione e Collaudo                &  x \\
\textbf{Totale}                       &  x \\

\end{longtable}
}