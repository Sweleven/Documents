\newpage
\subsection{Preventivo fase di Progettazione}

\subsubsection{Divisione oraria}
La seguente tabella rappresenta la distribuzione oraria dei ruoli per ogni componente del gruppo:
\rowcolors{2}{\evenRowColor}{\oddRowColor}
\renewcommand{\arraystretch}{2}
\begin{longtable}[h!] { C{4cm} C{1cm} C{1cm} C{1cm} C{1cm} C{1cm} C{1cm} C{3cm}}
\caption{Tabella della divisione oraria della Progettazione}\\
\rowcolor{\primaryColor}

\textcolor{\secondaryColor}{\textbf{Membro del gruppo}} & 
\textcolor{\secondaryColor}{\textbf{RE}} & 
\textcolor{\secondaryColor}{\textbf{AM}} & 
\textcolor{\secondaryColor}{\textbf{AN}} & 
\textcolor{\secondaryColor}{\textbf{PT}} & 
\textcolor{\secondaryColor}{\textbf{PR}} & 
\textcolor{\secondaryColor}{\textbf{VE}} & 
\textcolor{\secondaryColor}{\textbf{Ore complessive}}\\	
\endhead
        
\AW{}                     & 10  & - & - & 8 & 8 & 5 & 32 \\
\AT{}                     & -  & 10 & - & 7 & - & 14 & 32 \\
\AD{}                     & -  & 10 & - & 7 & 9 & 9 & 33 \\
\EC{}                     & 11  & - & 4 & - & 9 & 8 & 32 \\
\EM{}                     & -  & - & 5 & 7 & 11 & 9 & 32 \\
\FP{}                     & -  & - & 6 & 10 & 10 & 7 & 33 \\
\GG{}                     & -  & - & 5 & 10 & 11 & 7 & 33 \\
\textbf{Ore totali ruolo} & 21 & 20 & 20 & 49 & 58 & 59 & 227 \\

		
\end{longtable}
La suddivisione delle ore preventivate viene rappresentata anche sottoforma del seguente istogramma:
\begin{center}
	\pgfplotsset{width=17cm, height=8.5cm}
	\begin{tikzpicture}
		\begin{axis}[
			ybar stacked,
			bar width=20pt,
			legend style={
				at={(0.5,-0.15)},
				anchor=north,
				legend columns=-1
			},
			symbolic x coords={Abdelwahad, Alessio, Andrea, Edoardo, Elvis, Filippo, Giovanni},
			xtick=data
		]
			\legend{Responsabile, Amministratore, Analista, Progettista, Programmatore, Verificatore}
			% Responsabile
			\addplot [ybar, fill=blue] coordinates {\ColonnaIstogramma{10}{0}{0}{11}{0}{0}{0}};
			% Amministratore
			\addplot [ybar, fill=yellow] coordinates {\ColonnaIstogramma{0}{10}{10}{0}{0}{0}{0}};
			% Analista
			\addplot [ybar, fill=red] coordinates {\ColonnaIstogramma{0}{0}{0}{4}{5}{6}{5}};
			% Progettista
			\addplot [ybar, fill=green] coordinates {\ColonnaIstogramma{8}{7}{7}{0}{7}{10}{10}};
			% Programmatore
			\addplot [ybar, fill=pink] coordinates {\ColonnaIstogramma{8}{0}{9}{9}{11}{10}{11}};
			% Verificatore
			\addplot [ybar, fill=orange] coordinates {\ColonnaIstogramma{5}{14}{9}{8}{9}{7}{7}};
		\end{axis}
	\end{tikzpicture}
\end{center}

\clearpage
\subsubsection{Costo risultante}
La seguente tabella rappresenta per ogni ruolo le ore totali preventivate ed il corrispondente costo in euro:
{
\rowcolors{2}{\evenRowColor}{\oddRowColor}
\renewcommand{\arraystretch}{2}
\begin{longtable}{ C{3cm} C{2cm} C{4cm}}
\caption{Tabella del costo risultante della Progettazione}\\
\rowcolor{\primaryColor}

\textcolor{\secondaryColor}{\textbf{Ruolo}} & 
\textcolor{\secondaryColor}{\textbf{Totale ore}} & 
\textcolor{\secondaryColor}{\textbf{Costo ruolo (in \euro{})}}\\	
\endhead
        
Responsabile    &  21 &  630 \\
Amministratore  &  20 &  400 \\
Analista        &  20 &  500 \\
Progettista     &  49 &  1078 \\
Programmatore   &  58 &  870 \\
Verificatore    &  59 &  885 \\
\textbf{Totale} &  227 & 4363 \\	
        	
\end{longtable}
}

\vskip 30pt %spazio verticale
La quantità di ore totali preventivate per ciascun ruolo viene rappresentata nel seguente areogramma:
\begin{center}
	\begin{tikzpicture}
		\pie[rotate = 180, color={blue, yellow, red, green, pink, orange}] {
			7/Responsabile,
			9/Amministratore,
			9/Analista,
			22/Progettista,
			27/Programmatore,
			26/Verificatore
		}
	\end{tikzpicture}
\end{center}