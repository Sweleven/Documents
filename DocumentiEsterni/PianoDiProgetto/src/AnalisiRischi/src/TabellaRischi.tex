Di seguito viene riportata in forma tabellare un elenco di tutti i potenziali rischi rilevati comprensivi dei criteri sopracitati. Per ognuno di essi viene inoltre rappresentata una breve descrizione e la strategia da adottare per mitigarlo nel migliore dei modi.
{
        \rowcolors{2}{\evenRowColor}{\oddRowColor}
        \renewcommand{\arraystretch}{1.5}
        
        \begin{longtable}{|p{0.8cm}|p{3cm}|p{1cm}|p{1cm}|p{3cm}|p{3.2cm}|}
        \caption{Tabella dei rischi}\\
        \rowcolor{\primaryColor}
        \textcolor{\secondaryColor}{\textbf{Cod.}} & 
        \textcolor{\secondaryColor}{\textbf{Descrizione}} & 
        \textcolor{\secondaryColor}{\textbf{Prob.}} & 
        \textcolor{\secondaryColor}{\textbf{Grav.}} & 
        \textcolor{\secondaryColor}{\textbf{Rilevamento}} &
        \textcolor{\secondaryColor}{\textbf{Risoluzione}}\\
       
        RTa & 
        Gli strumenti o tecnologie richieste sono completamente o in parte sconosciute ad uno o più membri del gruppo. &
        alta & 
        alta &
        La persona interessatà si accorgerà di non conoscere ciò che viene richiesto nel relativo lavoro. &
        Documentarsi autonomamente, eventuale supporto dei compagni più esperti, suddivisione dei compiti tale da risultare meno onerosa.\\
        \hline
        RTb &
        L'ambiente di lavoro è male ottimizzato per lo sviluppo e lo svolgimento dei compiti. &
        media &
        bassa &
        Comunicare tra i membri del gruppo per informarsi a vicenda sulle metodologie solitamente usate per determinate procedure. &
        I membri già agili in determinate situazioni daranno consigli utili ai meno esperti.\\
        \hline
        RTc &
        Impossibilità di utilizzare alcuni strumenti necessari al progetto, dovuta a malfunzionamenti o mancato possesso. &
        bassa &
        media &
        Una volta che si ha chiaro cosa sia necessario, informare tempestivamente il gruppo dell'eventuale mancanza. &
        Munirsi al più presto del necessario. Qualora non fosse possibile altri membri del gruppo prenderanno in carico le relative mansioni, affidando all'interessato quelle che non hanno necessità dell'elemento in questione. \\
        \hline
        RTd &
        Mancata comprensione di alcuni \glo{requisiti} richiesti dal progetto. &
        media &
        alta &
        Comunicare quanto prima al gruppo qualunque cosa non sia stata ben capita dalla persona in questione. &
        Gli altri membri che hanno chiara la questione si impegnano a spiegare ed aiutare nella comprensione. Se necessario si provvederà a contattare il proponente. \\
        \hline
        RTe &
        Tempo a disposizione scarso per lo svolgimento di alcuni compiti onerosi. &
        alta &
        alta &
        In base all'esperienza cercare di stimare il tempo necessario effettivo per un determinato compito. &
        Si cerca di ottimizzare distribuendo le diverse parti del lavoro tra i membri più esperti o adatti a svolgerle. \\
        \hline
        RPa &
        Impegni personali regolari. &
        alta &
        bassa &
        E' stato creato un documento condiviso contenente una tabella in cui ogni membro del gruppo segna i giorni ed orari in cui garantisce la propria disponibilità. &
        Ci si impegna ad organizzare il lavoro e specialmente gli incontri concordantemente agli impegni di tutti. \\
        \hline
        RPb &
        Indisposizione improvvisa di un membro del gruppo (dovuta ad impegni o malattia) che impossibilita lo svolgimento di alcuni compiti. &
        bassa &
        alta &
        Comunicare tempestivamente al gruppo di essere impossibilitati. &
        A seconda dell'importanza del compito assegnato si valuterà se concedere più tempo oppure suddividerlo tra i restanti membri. \\
        \hline
        ROa &
        Incomprensioni o difficili rapporti tra membri del gruppo. &
        media &
        alta &
        Qualora ci fossero dei problemi relazionali, comunicarli per quanto possibile ai membri del gruppo. &
        Comunicare trasparentemente e con rispetto il proprio punto di vista; cercare di venirsi incontro trovando delle soluzioni comuni. \\
        \hline
        ROb &
        Mancata osservanza di alcune regole pre-concordate dal gruppo. &
        media &
        alta &
        Sono stati adottati e predisposti diversi strumenti che permettono un continuo ed aggiornato monitoraggio dell'attività di ogni membro nelle diverse fasi del lavoro. &
        Verrà fatto presente quanto prima all'interessato, invitando ad osservare le regole più fedelmente. \\
        \hline
        ROc &
        Tempi assegnati sovrastimati o sottostimati, per la realizzazione di alcuni compiti. &
        alta &
        media &
        Nel momento in cui ci si accorge del problema, comunicarlo al gruppo per verificare se altri membri sono della stessa opinione. &
        Prendere atto della cosa per programmare in maniera più efficace i futuri compiti da svolgere. \\
        \hline
        ROd &
        Mancato assolvimento o ritardo dei compiti assegnati. &
        bassa &
        alta &
        All'avvicinarsi dello scadere dei tempi stabiliti, il responsabile si informerà sullo stato dei lavori. &
        Nei casi più urgenti i restanti membri del gruppo valuteranno se suddividersi il lavoro da svolgere. \\
        \hline
    
        \end{longtable}
    }