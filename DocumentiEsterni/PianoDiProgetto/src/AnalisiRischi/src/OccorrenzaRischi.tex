Di seguito viene stilata una tabella parziale, soggetta ad aggiornamenti progressivi, in cui si elencano le occorrenze dei rischi effettivamente verificatisi finora. Vengono riportati:
\begin{itemize}
    \item il codice identificativo del rischio;
    \item lo stadio, o gli stadi, in cui si è verificato (vedasi il capitolo 4, sulla pianificazione);
    \item il grado di occorrenza (in questo caso indicativo di quanto spesso si sia realmente verificato);
    \item una breve descrizione specifica dell'evenienza e di come sia stata effettivamente mitigata.
\end{itemize}
{
\rowcolors{2}{\evenRowColor}{\oddRowColor}
        \renewcommand{\arraystretch}{1.5}
        \centering
        \begin{longtable}{|p{0.8cm}|p{1.2cm}|p{2cm}|p{8.8cm}|}
        \caption{Tabella di occorrenza dei rischi}\\
        \rowcolor{\primaryColor}
        \textcolor{\secondaryColor}{\textbf{Cod.}} & 
        \textcolor{\secondaryColor}{\textbf{Stadio}} & 
        \textcolor{\secondaryColor}{\textbf{Occorrenza}} & 
        \textcolor{\secondaryColor}{\textbf{Descrizione}}\\
       
        RTa & 
        1 & 
        media &
        Parte dei membri del gruppo si sono ritrovati per la prima volta ad usare strumenti quali \glo{GitHub} e \glo{LaTex}. Sono stati predisposti dei template ed impostato un repository di base da chi era già pratico di tali strumenti; ognuno ha inoltre provveduto a documentarsi ed esercitarsi autonomamente sull'utilizzo; i membri già avezzi si sono resi da subito disponibili per suggerimenti e chiarimenti.\\
        \hline
        RTb & 
        2 & 
        bassa &
        Non tutti i membri hanno fin da subito utilizzato strumenti di supporto consigliati nelle \NdP{}, ad esempio \glo{GitKraken}, utile per una veloce sincronizzazione con \glo{GitHub} da \glo{repository} locale. Chi già conosceva questi strumenti ne ha incitato e spiegato l'utilizzo, permettendo così una velocizzazione del processo di lavoro.\\
        \hline
        RTd & 
        1 & 
        media &
        Inizialmente c'è stata una certa confusione riguardo il formato ed i contenuti della documentazione da consegnare. Ad ogni riunione svolta si è cercato tra tutti i membri di sollevare e risolvere insieme ogni eventuale dubbio, in alcuni casi con l'aiuto di risorse online, standard \glo{ISO} e le indicazioni fornite dal committente.\\
        \hline
        RPa & 
        1, 2 & 
        bassa &
        In alcuni rari casi non è stato comunque possibile fissare una data per gli incontri senza andare in contrasto con gli impegni regolari di almeno un membro del gruppo. La riunione è stata svolta ugualmente, previo unanime consenso, con la promessa di un aggiornamento da parte dell'assente.\\
        \hline
        ROb & 
        1 & 
        alta &
        Strumenti di monitoraggio importanti pre-concordati, quali ad esempio \glo{Jira}, non sono stati fin da subito usati regolarmente da tutti i membri del gruppo. E' stato fatto presente incitando all'osservanza delle regole e tutti i membri si sono impegnati a seguirle con maggiore accortezza.\\
        \hline
        ROd & 
        1 & 
        bassa &
        In qualche caso la redazone di alcuni documenti si è protratta oltre il tempo prestabilito per ultimare gli stessi, ritardando così i tempi per la loro validazione. Fortunatamente i ritardi non sono stati eccessivi e si è riusciti ugualmente a completare i lavori entro i tempi stabiliti, senza particolari complicazioni. E' stato in ogni caso fatto presente tra i membri del gruppo.\\
        \hline
    \end{longtable}
}