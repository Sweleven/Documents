Gli aspetti su cui ci si è maggiormente concentrati sono descritti nei punti seguenti.

\subsubsection{Tipo di rischio}
    Per identificare al meglio ogni potenziale rischio viene assegnata ad ognuno di essi una determinata sigla che indica, come da titolo, il tipo di rischio in cui ci si può imbattere e viene suddiviso nelle seguenti categorie:
    \begin{itemize}
        \item \textbf{RTx}: rischio tecnologico, riguarda i problemi prettamente tecnici che possono subentrare;
        \item \textbf{RPx}: rischio personale, relativo a questioni private esterne che possono riguardare uno o più membri del gruppo;
        \item \textbf{ROx}: rischio organizzativo, comprendente problemi di tempistiche o eventuali disguidi all'interno del gruppo;
        \item \textbf{RRx}: rischio di requisiti, comprende problemi legati ai requisiti individuati nell'\AdR{};
        \item \textbf{RSx}: rischio di stima, relativo a problemi sul calcolo costo/orario.
    \end{itemize}
    Dove il carattere \textbf{x} indica per ogni tipologia una lettera minuscola assegnata, in ordine alfabetico crescente, con lo scopo di dare un'identificazione univoca al rischio corrispondente.
    
\subsubsection{Probabilità}
    Indica la probabilità che il rischio si presenti e può essere alta, media o bassa. Utile perchè permette di porre diversa attenzione alle problematiche identificate con una certa probabilità, ponendo particolare riguardo a quelle con possibile occorrenza più alta.

\subsubsection{Gravità}
    Specifica quanto il rispettivo rischio può impattare sullo svolgimento del lavoro:
    \begin{itemize}
        \item \textbf{bassa}: il problema può essere affrontato con relativa calma perchè non urgente o di lieve importanza;
        \item \textbf{media}: il problema non è urgente ma abbastanza importante, quindi andrebbe affrontato non appena possibile;
        \item \textbf{alta}: il problema è potenzialmente bloccante per l'intero progetto e necessita quindi di immediata attenzione e risoluzione.
    \end{itemize}
    
\subsubsection{Identificazione del problema}
    Per velocizzare l'intero \glo{processo}, è importante stabilire delle eventuali strategie anche per la rilevazione di ogni possibile problema conosciuto.