

        \subsubsection{Periodo 1 fase di sviluppo}
        Questo periodo è relativo in gran parte allo sviluppo del codice. In questa \glo{fase} principale di sviluppo ci si concentrerà non solo sugli aspetti prettamente funzionali, ma anche, e soprattuto, sui \glo{requisiti} di tipo \glo{prestazionale}. L'obiettivo principale è dunque raggiungere una versione più aggiornata, funzionante, rifinita e sufficientemente ottimizzata del prodotto.
        \begin{itemize}
                \item \textbf{intervallo: } 17 marzo 2021 - 31 marzo 2021;
                \item  \textbf{Figure professionali coinvolte}
                \begin{itemize}
                    \item responsabile;
                    \item amministratore;
                    \item progettista;
                    \item programmatore;
                    \item verificatore.
                \end{itemize}
                \item \textbf{Obiettivi}
                \begin{itemize}
                    \item verifica rischi occorsi e/o potenziali nella fase ed attivare l'eventuale risoluzione o prevenenzione
                    \item definizione di tutti gli incrementi di sviluppo obbligatori da soddisfare;
                    \item correzione della documentazione secondo le direttive del committente;
                    \item implementazione incrementi appena definiti;
                    \item aggiornamento della documentazione.
                \end{itemize}
        \end{itemize}

    %    \subsubsection{Periodo 2 fase di sviluppo}
     %   A questo punto si presuppone di avere un prodotto sufficientemente funzionante ma ancora fondamentalmente incompleto. Questo periodo corrisponde ad una prima parte della \glo{fase} di \glo{validazione} ed il raggiungimento dei suoi obiettivi corrisponde alla terza \glo{milestone} principale del progetto ovvero la Revisione di Qualifica. Lo scopo di fondo è ultimare l'implementazione e concentrarsi principalmente sugli aspetti qualitativi del prodotto, in modo da consegnare infine un \glo{deliverable} conforme a quanto richiesto dal terzo incontro di revisione programmato.
      %  \begin{itemize}
       %         \item \textbf{intervallo: } 1 aprile 2021 - 15 aprile 2021;
        %        \item  \textbf{Figure professionali coinvolte}
         %       \begin{itemize}
                    %\item responsabile;
                   % \item amministratore;
                  %  \item progettista;
                 %   \item programmatore;
                %    \item verificatore.
               % \end{itemize}
              %  \item \textbf{Obiettivi}
             %   \begin{itemize}
            %        \item implementazione delle componenti restanti aggiuntive ed eventuale soddisfacimento dei \glo{requisiti} facoltativi;
           %         \item controllo, implementazione e correzione degli aspetti qualitativi del prodotto;
          %          \item mostrare un prodotto funzionante e con tutti i requisiti obbligatori soddisfatti al proponente per ricevere eventuali feedback;
         %           \begin{itemize}
        %                \item tali aspetti sono da riferirsi allo studio fatto nel documento \PdQ{}. %(potrei sbagliarmi)
       %             \end{itemize}
      %              \item copertura minima dei test all'80\% con relativo resoconto;
     %               \item \glo{verifica} del codice e delle funzionalità del prodotto;
    %                \item aggiornamento, verifica e correzione generale dell'intera documentazione redatta fino a questo momento;
   %                 \item consegna del \glo{deliverable} in vista dell'incontro con il committente per la \RQ{}.
  %              \end{itemize}
 %       \end{itemize}
        