Questo stadio è relativo in gran parte allo sviluppo del codice e la sua conclusione corrisponde di fatto alla seconda \glo{milestone} principale che il gruppo \Gruppo{} intende rispettare. In questa \glo{fase} principale di codifica ci si concentrerà non solo sugli aspetti prettamente funzionali, ma anche, e soprattuto, sui \glo{requisiti} di tipo \glo{prestazionale}. L'obiettivo principale è dunque raggiungere una versione più aggiornata, funzionante, rifinita e sufficientemente ottimizzata del prodotto, conformemente a quanto richiesto dal secondo incontro di revisione programmato.
        
        \subsubsection{Periodo di svolgimento}
        15 febbraio - 1 marzo 2021;
        
        \subsubsection{Figure professionali coinvolte}
            \begin{itemize}
                \item responsabile;
                \item amministratore;
                \item progettista;
                \item programmatore;
                \item verificatore.
            \end{itemize}

        \subsubsection{Obiettivi}
        \begin{itemize}
            \item consolidamento delle componenti obbligatorie;
            \item inclusione delle componenti e requisiti desiderabili;
            \item rifinitura e miglioramento delle \glo{interfacce grafiche};
            \item \glo{verifica} del codice e delle funzionalità del prodotto;
            \item aggiornamento, verifica e correzione generale dell'intera documentazione redatta fino a questo momento;
            \item consegna del \glo{deliverable} necessario, in vista dell'incontro con il committente per la \RP{}.
        \end{itemize}