A questo punto si presuppone di avere un prodotto sufficientemente funzionante ma ancora fondamentalmente incompleto. Questo periodo corrisponde ad una prima parte della \glo{fase} di \glo{validazione} ed il raggiungimento dei suoi obiettivi corrisponde alla terza \glo{milestone} principale del progetto ovvero la Revisione di Qualifica. Lo scopo di fondo è ultimare l'implementazione e concentrarsi principalmente sugli aspetti qualitativi del prodotto, in modo da consegnare infine un \glo{deliverable} conforme a quanto richiesto dal terzo incontro di revisione programmato.
        
        \subsubsection{Periodo di svolgimento}
        \textbf{intervallo obbiettivo:}1 aprile 2021 - 15 aprile 2021;
        
        \subsubsection{Figure professionali coinvolte}
            \begin{itemize}
                \item responsabile;
                \item amministratore;
                \item progettista;
                \item programmatore;
                \item verificatore.
            \end{itemize}

        \subsubsection{Obiettivi}
        \begin{itemize}
            \item implementazione delle componenti restanti aggiuntive ed eventuale soddisfacimento dei \glo{requisiti} facoltativi;
            \item controllo, implementazione e correzione degli aspetti qualitativi del prodotto;
            \item mostrare un prodotto funzionante e con tutti i requisiti obbligatori soddisfatti al proponente per ricevere eventuali feedback;
            \begin{itemize}
                \item tali aspetti sono da riferirsi allo studio fatto nel documento \PdQ{}. %(potrei sbagliarmi)
            \end{itemize}
            \item copertura minima dei test all'80\% con relativo resoconto;
            \item \glo{verifica} del codice e delle funzionalità del prodotto;
            \item aggiornamento, verifica e correzione generale dell'intera documentazione redatta fino a questo momento;
            \item consegna del \glo{deliverable} in vista dell'incontro con il committente per la \RQ{}.
        \end{itemize}