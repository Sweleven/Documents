Svolte le attività preliminari, il gruppo si concentra più in dettaglio sull'analisi del capitolato scelto e redige una prima stesura della documentazione esterna. L'obiettivo principale di questa \glo{fase} è produrre la documentazione richiesta per il primo \glo{incontro formale} di revisione programmato. La positiva conclusione di questo stadio corrisponde di fatto alla prima \glo{milestone} ufficiale del gruppo.
        
        \subsubsection{Periodo di svolgimento}
        4 dicembre 2020 - 11 gennaio 2021;
        
        \subsubsection{Figure professionali coinvolte}
            \begin{itemize}
                \item responsabile;
                \item amministratore;
                \item analista;
                \item verificatore.
            \end{itemize}
        
        \subsubsection{Obiettivi}
            \begin{itemize}
                \item \glo{verifica} e correzione delle \NdP{} e dello \SdF{};
                \item redazione \AdR{}:
                \begin{itemize}
                    \item studio e definizione dei casi d'uso;
                    \item individuazione, classificazione ed analisi dei \glo{requisiti} di progetto;
                    \item definizione delle modalità di \glo{tracciamento}.
                \end{itemize}
                \item redazione \PdQ{};
                \begin{itemize}
                    \item studio e definizione dei criteri qualitativi del prodotto e dei processi;
                    \item definizione dei criteri e delle modalità di verifica, di autovalutazione ed esecuzione dei test.
                \end{itemize}
                \item redazione \PdP{};
                \begin{itemize}
                    \item analisi dei rischi;
                    \item scelta e definizione di un modello di sviluppo e pianificazione del lavoro;
                    \item definizione dell'organigramma generale del gruppo;
                    \item calcolo del preventivo iniziale relativo allo svolgimento del lavoro;
                \end{itemize}
                \item aggiornamento, verifica e correzione generale dell'intera documentazione redatta fino a questo momento;
                \item consegna del \glo{deliverable} necessario, in vista dell'incontro con il committente per la \RR{}.
            \end{itemize}