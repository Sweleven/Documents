Da qui in poi si entra nella parte più specifica di costruzione del sistema. Per avviare lo sviluppo questo stadio ultima la progettazione al dettaglio ed inzia quella di codifica. Viene quindi prodotta una prima stesura del codice con lo scopo di mettere insieme un iniziale prototipo stabile del sistema, avente le minime funzionalità essenziali e basilari. La sua conclusione corrisponde di fatto alla seconda \glo{milestone} principale che il gruppo \Gruppo{} intende rispettare, ovvero la Revisione di Progettazione, in vista dell'incontro di revisione programmato.
        
        \subsubsection{Periodo di svolgimento}
        1 febbraio 2021 - 1 marzo 2021;
        
        \subsubsection{Figure professionali coinvolte}
            \begin{itemize}
                \item responsabile;
                \item amministratore;
                \item progettista;
                \item programmatore;
                \item verificatore.
            \end{itemize}

        \subsubsection{Obiettivi}    
        \begin{itemize}
            \item definizione completa delle strutture tramite diagrammi \glo{UML};
            \item implementazione delle componenti essenziali e obbligatorie;
            \begin{itemize}
                \item tali componenti verranno specificate più in dettaglio nei futuri aggiornamenti del presente documento, una volta stabiliti gli incrementi necessari.
            \end{itemize}
            \item progettazione di base delle \glo{interfacce} utente;
            \item \glo{verifica} del codice e delle funzionalità del prodotto;
            \item aggiornamento, verifica e correzione generale dell'intera documentazione redatta fino a questo momento;
            \item consegna del \glo{deliverable} necessario, in vista dell'incontro con il committente per la \RP{}.
        \end{itemize}