Da qui in poi si entra nella parte più specifica di costruzione del sistema. Questo periodo ultima la fase di progettazione e si avvia verso la fase di sviluppo. Viene quindi prodotta una prima stesura del codice con lo scopo di mettere insieme un iniziale prototipo stabile del sistema, avente le minime funzionalità essenziali e basilari. La sua conclusione corrisponde di fatto alla seconda \glo{milestone} principale che il gruppo \Gruppo{} intende rispettare, ovvero la Revisione di Progettazione, in vista dell'incontro di revisione programmato.
        
        \subsubsection{Periodo di svolgimento}
        9 febbraio 2021 - 10 marzo 2021;
        
        \subsubsection{Figure professionali coinvolte}
            \begin{itemize}
                \item responsabile;
                \item amministratore;
                \item progettista;
                \item programmatore;
                \item verificatore.
            \end{itemize}

        \subsubsection{Obiettivi}    
        \begin{itemize}
            \item definizione completa delle strutture tramite diagrammi \glo{UML};
            \item scelta effettiva tecnologie rischieste

            \item progettazione concettuale dell'architettura del sistema(technology baseline);
            \item implementazione di parte delle componenti essenziali e obbligatorie che verrà usata come baseline per lo sviluppo verò e proprio,si rimanda alla\hypersetup{
                linkcolor=blue
            }
            \hyperlink{TabellaIncrementi}{tabella degli incrementi} in appendice;
            \hypersetup{
                linkcolor=black
            };
            \item aggiornamento, verifica e correzione generale dell'intera documentazione redatta fino a questo momento;
            \item consegna del \glo{deliverable} necessario, in vista dell'incontro con il committente per la \RP{}.
        \end{itemize}