       \subsubsection{Periodo 1 - Fase di progettazione}
        Una volta approvati i \glo{requisiti} di ingresso da parte del team in seguito all'aggiudicazione ufficale del capitolato, il gruppo procede verso la \glo{fase} progettuale. In questo primo periodo viene effettuata un'analisi critica del lavoro svolto. Lo scopo principale prevede la definizione degli aspetti preliminari associati allo sviluppo dell'applicazione. Inoltre si procede con la valutazione e la definizione della \glo{technology baseline}. Vengono così poste le basi per intraprendere una prima progettazione del prodotto.
        \begin{itemize}
                \item \textbf{intervallo: } 25 gennaio 2021 - 8 febbraio 2021;
                \item  \textbf{Figure professionali coinvolte}
                \begin{itemize}
                    \item responsabile;
                    \item amministratore;
                    \item progettista;
                    \item analista;
                    \item verificatore.
                \end{itemize}
                \item \textbf{Obiettivi}
                \begin{itemize}
                    \item valutazione critica di eventuali rischi riscontrati o con alta probabilità che si riscontrino;
                    \item correzione della documentazione;
                    \item ristrutturazione e correzione dell'\AdR;
                    \item valutazione delle componenti e tecnologie richieste:
                    \begin{itemize}
                        \item valutazione della fattibilità effettiva dell'utilizzo della tecnologia \glo{RFID};
                        \item contattare il proponente per alcuni chiarimenti in merito a \glo{blockchain} e \glo{NFC};
                    \end{itemize}

                    \item aggiornamento della documentazione.
                    
                \end{itemize}
        \end{itemize}


            \subsubsection{Periodo 2 - Fase di progettazione}
        
            Vengono studiate le tecnologie proposte e la loro applicabilità al fine del futuro sviluppo del prodotto, analizzandone le relative interazioni.
    
            \begin{itemize}
                \item \textbf{intervallo: }9 febbraio 2021 - 23 febbraio 2021;
            
            \item  \textbf{Figure professionali coinvolte}
                \begin{itemize}
                    \item responsabile;
                    \item amministratore;
                    \item progettista;
                    \item verificatore.
                \end{itemize}
    
                \item \textbf{Obiettivi}  
                            \begin{itemize}
                                \item scelta effettiva delle tecnologie richieste per implementare i requisiti obbligatori e critici;
                                \item studio delle tecnologie scelte coinvolte nel \glo{PoC};
                                \item definizione primi incrementi di sviluppo secondo i criteri di necessità, tipologia e quindi priorità dei requisiti.
                            \end{itemize}
                \end{itemize}          
            
                \subsubsection{Periodo 3 - Fase di progettazione}
        
                Da qui in poi si entra nella parte più specifica di costruzione del sistema. Viene quindi prodotta una prima stesura del codice con lo scopo di costruire un \glo{Proof of Concept} che utilizzi le tecnologie scelte.
        
                \begin{itemize}
                    \item \textbf{intervallo: }24 febbraio 2021 - 10 marzo 2021;
                
                \item  \textbf{Figure professionali coinvolte}
                    \begin{itemize}
                        \item responsabile;
                        \item amministratore;
                        \item progettista;
                        \item programmatore;
                        \item verificatore.
                    \end{itemize}
        
                    \item \textbf{Obiettivi}

                                \begin{itemize}
                                    \item implementazione degli incrementi definiti nel periodo 2 si veda la\hypersetup{
                                        linkcolor=blue
                                    }
                                    \hyperlink{TabellaIncrementi}{tabella degli incrementi} in appendice;
                                    \hypersetup{
                                        linkcolor=black
                                    }
                                    \item implementazione parziale componenti a supporto dell'architettura del sistema e di futuri incrementi:
                                            \begin{itemize}
                                                \item setup \glo{Kubernetes};
                                                \item \glo{backend} gestione postazioni;
                                                \item autenticazione delle \glo{API esposte};
                                            \end{itemize}
                                    \item aggiornamento, verifica e correzione generale dell'intera documentazione redatta fino a questo momento;
                                    \item consegna del \glo{deliverable} necessario, in vista dell'incontro con il committente per la \RP{}.
                                \end{itemize}
                    \end{itemize}
            
