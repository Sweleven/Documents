Una volta approvati i \glo{requisiti} di ingresso, in seguito all'aggiudicazione ufficale del capitolato, il gruppo procede verso la \glo{fase} progettuale. Appoggiandosi all'\AdR{}, in questo stadio lo scopo principale è definire e preparare tutti gli aspetti preliminari allo sviluppo vero e proprio, avviando così la progettazione architetturale del prodotto.
        
        \subsubsection{Periodo di svolgimento}
        18 gennaio 2021 - 31 gennaio 2021;
        
        \subsubsection{Figure professionali coinvolte}
            \begin{itemize}
                \item responsabile;
                \item amministratore;
                \item progettista;
                \item verificatore.
            \end{itemize}

        \subsubsection{Obiettivi}
        \begin{itemize}
            \item correzione della documentazione secondo le direttive del committente;
            \item strutturazione generale del sistema:
            \begin{itemize}
                \item organizzazione di tutti i suoi elementi;
                \item pianificazione di massima di un ordine di sviluppo secondo i criteri di necessità, tipologia e quindi priorità dei requisiti.
            \end{itemize}
            \item valutazione delle componenti e tecnologie richieste:
            \begin{itemize}
                \item valutazione della fattibilità effettiva dell'utilizzo di tecnologia \glo{RFID};
                \item scelta dei linguaggi e degli strumenti da utilizzare per lo sviluppo;
            \end{itemize}
            \item aggiornamento della documentazione.
            
        \end{itemize}