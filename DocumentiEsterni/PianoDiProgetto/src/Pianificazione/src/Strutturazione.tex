Di seguito verrà riportata la struttura generale in cui verrà organizzato l'intero \glo{processo} di lavoro. In questo caso sono stati specificati in totale sette \glo{stadi} raggruppati essenzialmente in quattro \glo{fasi}. Per ogni stadio vengono individuati quelli che attualmente sono gli obiettivi che il gruppo \Gruppo{} si prefigge di raggiungere entro la data indicata di fine dello stesso. Nei successivi paragrafi verrà descritto ogni stadio più nel dettaglio.\\
Qui sotto viene esposto sinteticamente lo schema generale, dove ogni stadio è concettualmente collocato all'interno della rispettiva fase:

\begin{itemize}
    \item \textbf{Fase di Analisi}:
    \begin{itemize}
        \item Stadio 1 - Organizzazione preliminare;
        \item Stadio 2 - Soddisfacimento requisiti d'ingresso.
    \end{itemize}

    \item \textbf{Fase di Progettazione Architetturale}:
    \begin{itemize}
        \item Stadio 3 - Strutturazione delle componenti;
        \item Stadio 4 - Concretizzazione del \glo{Proof of Concept};
    \end{itemize}

    \item \textbf{Fase di Dettaglio e Codifica}:
    \begin{itemize}  
        \item Stadio 5 - Sviluppo e rifinitura;
        \item Stadio 6 - Qualifica e testing.
    \end{itemize}

    \item \textbf{Fase di Validazione e Collaudo}:
    \begin{itemize}        
        \item Stadio 7 - Finalizzazione.
    \end{itemize}
\end{itemize}

