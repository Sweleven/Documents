Nella pianificazione vengono definiti tutti i passi necessari alla conclusione del presente progetto. \\
Di seguito viene qui riportata la struttura generale in cui verrà organizzato l'intero lavoro. In questo caso sono state specificate sette fasi e per ognuna di esse sono stati individuati gli obiettivi che il gruppo \Gruppo{} si prefigge di aver raggiunto entro la data di fine della fase stessa. Nel corso delle varie fasi il gruppo non mancherà di restare, ogni qualvolta necessario, in contatto con l'azienda proponente \proponente{}, per poter così chiarire dubbi o accordarsi su eventuali modifiche da poter apportare ad alcuni requisiti.\\
Si ricorda inoltre che secondo i modello scelto, lo sviluppo in generale avverrà in modo incrementale, perciò, oltre ad una preventiva organizzazione delle parti, si procederà sempre all'implementazione per aggiunte e/o modifiche successive, secondo gli incrementi stabiliti e cercando ogni volta di garantire un sistema funzionante.

\subsection{Fase 1 - Organizzazione preliminare}\label{sec:Fase1}
\input{src/Pianificazione/src/Fase1.tex}

\subsection{Fase 2 - Soddisfacimento requisiti di ingresso}\label{sec:Fase2}
Svolte le attività preliminari, il gruppo si concentra più in dettaglio nel capitolato scelto e redige una prima stesura della documentazione esterna, preparandosi conformemente a quanto richiesto dal primo incontro di revisione programmato.
        
        \subsubsection{Periodo di svolgimento}
        4 dicembre 2020 - 11 gennaio 2021;
        
        \subsubsection{Figure professionali coinvolte}
            \begin{itemize}
                \item responsabile;
                \item amministratore;
                \item analista;
                \item verificatore.
            \end{itemize}
        
        \subsubsection{Obiettivi}
            \begin{itemize}
                \item verifica e correzione delle \NdP{} e dello \SdF{};
                \item scelta del logo del gruppo;
                \item redazione \AdR{}:
                \begin{itemize}
                    \item studio e definizione dei casi d'uso;
                    \item individuazione, classificazione e codifica dei requisiti di progetto;
                    \item definizione delle modalità di tracciamento.
                \end{itemize}
                \item redazione \PdQ{};
                \begin{itemize}
                    \item studio e definizione dei criteri qualitativi del prodotto e dei processi;
                    \item definizione dei criteri e delle modalità di verifica, di autovalutazione ed esecuzione dei test.
                \end{itemize}
                \item redazione \PdP{};
                \begin{itemize}
                    \item analisi dei rischi;
                    \item scelta e definizione di un modello di sviluppo e pianificazione del lavoro;
                    \item definizione dell'organigramma generale del gruppo;
                    \item calcolo del preventivo iniziale relativo allo svolgimento del lavoro;
                \end{itemize}
                \item aggiornamento, verifica e correzione generale dell'intera documentazione redatta fino a questo momento;
                \item consegna di tutto il materiale necessario, in vista dell'incontro con il committente per la Revisione dei Requisiti.
            \end{itemize}
    
\subsection{Fase 3 - Consolidamento delle componenti}\label{sec:Fase3}
Una volta approvati i requisiti di ingresso, in seguito all'aggiudicazione ufficale del capitolato, il gruppo procede verso una fase intermedia. Appoggiandosi all'\AdR{}, questa fase ha lo scopo principale di definire e preparare tutti gli aspetti preliminari allo sviluppo vero e proprio ed avviare la progettazione del prodotto.
        
        \subsubsection{Periodo di svolgimento}
        18 gennaio 2021 - 27 gennaio 2021;
        
        \subsubsection{Figure professionali coinvolte}
            \begin{itemize}
                \item responsabile;
                \item amministratore;
                \item progettista;
                \item verificatore.
            \end{itemize}

        \subsubsection{Obiettivi}
        \begin{itemize}
            \item correzione della documentazione secondo le direttive del committente;
            \item strutturazione generale del sistema:
            \begin{itemize}
                \item organizzazione di tutti i suoi elementi;
                \item pianificazione di massima di un ordine di sviluppo secondo i criteri di necessità, tipologia e quindi priorità dei requisiti
            \end{itemize}
            \item valutazione delle componenti e tecnologie richieste:
            \begin{itemize}
                \item valutazione della fattibilità effettiva dell'utilizzo di tecnologia \glo{RFID};
                \item scelta dei linguaggi e degli strumenti da utilizzare per lo sviluppo;
            \end{itemize}
            \item aggiornamento coerente e continuativo della documentazione.
        \end{itemize}
    
\subsection{Fase 4 - Progettazione e sviluppo}\label{sec:Fase4}
Da qui in poi si entra nelle fasi più pratiche di costruzione del sisema. In questa fase ci sarà una progettazione più approfondita ed una prima stesura del codice con lo scopo di mettere insieme un primo prototipo funzionante del prodotto, avente le minime funzionalità essenziali e basilari. 
        
        \subsubsection{Periodo di svolgimento}
        28 gennaio 2021 - 14 febbraio 2021;
        
        \subsubsection{Figure professionali coinvolte}
            \begin{itemize}
                \item responsabile;
                \item amministratore;
                \item progettista;
                \item programmatore;
                \item verificatore.
            \end{itemize}

        \subsubsection{Obiettivi}    
        \begin{itemize}
            \item definizione completa delle strutture tramite diagrammi \glo{UML};
            \item studio ed apprendimento degli strumenti e tecnologie necessarie;
            \item implementazione delle componenti essenziali e obbligatorie;
            \item progettazione di base delle interfacce utente;
            \item verifica del codice e delle funzionalità del prodotto;
            \item aggiornamento continuativo e rigoroso della documentazione.
        \end{itemize}
    
\subsection{Fase 5 - Sviluppo e rifinitura }\label{sec:Fase5}
Questa ulteriore fase relativa in gran parte allo sviluppo del codice è di fatto la seconda \glo{milestone} principale del gruppo. In essa ci si concentrerà non solo sugli aspetti prettamente funzionali, ma anche, e soprattuto, sui requisiti di tipo prestazionale. L'obiettivo principale è dunque raggiungere una versione più aggiornata, funzionante, rifinita e sufficientemente ottimizzata del prodotto, preparando il \glo{deliverable} conformemente a quanto richiesto dal secondo incontro di revisione programmato.
        
        \subsubsection{Periodo di svolgimento}
        15 febbraio - 1 marzo 2021;
        
        \subsubsection{Figure professionali coinvolte}
            \begin{itemize}
                \item responsabile;
                \item amministratore;
                \item progettista;
                \item programmatore;
                \item verificatore.
            \end{itemize}

        \subsubsection{Obiettivi}
        \begin{itemize}
            \item consolidamento delle componenti obbligatorie;
            \item implementazione delle componenti e soddisfacimento dei requisiti desiderabili;
            \item rifinitura e miglioramento delle parti e delle interfacce;
            \item verifica del codice e delle funzionalità del prodotto;
            \item aggiornamento, verifica e correzione generale dell'intera documentazione redatta fino a questo momento;
            \item consegna di tutto il materiale necessario, in vista dell'incontro con il committente per la Revisione di Progettazione.
        \end{itemize}
    
\subsection{Fase 6 - Qualifica e testing}\label{sec:Fase6}
A questo punto si presuppone di avere un prodotto sufficientemente funzionante ma ancora fondamentalmente incompleto. In questa fase l'obiettivo è ultimare la parte di implementazione per poi concentrarsi principalmente sugli aspetti qualitativi del prodotto, in modo da avere tutto il necessario conformemente a quanto richiesto dal terzo incontro di revisione programmato.
        
        \subsubsection{Periodo di svolgimento}
        8 marzo 2021 - 2 aprile 2021;
        
        \subsubsection{Figure professionali coinvolte}
            \begin{itemize}
                \item responsabile;
                \item amministratore;
                \item progettista;
                \item programmatore;
                \item verificatore.
            \end{itemize}

        \subsubsection{Obiettivi}
        \begin{itemize}
            \item correzione della documentazione secondo le direttive del committente;
            \item implementazione delle componenti restanti aggiuntive ed eventuale soddisfacimento dei requisiti facoltativi;
            \item controllo, implementazione e correzione degli aspetti qualitativi del prodotto;
            \item effettuazione dei test con copertura minima all'80\% e stesura di un relativo resoconto;
            \item unificazione dei consuntivi di periodo nella creazione di un consuntivo generale;
            \item verifica del codice e delle funzionalità del prodotto;
            \item aggiornamento, verifica e correzione generale dell'intera documentazione redatta fino a questo momento;
            \item consegna di tutto il materiale necessario, in vista dell'incontro con il committente per la Revisione di Qualifica.
        \end{itemize}
    
\subsection{Fase 7 - Validazione e collaudo}\label{sec:Fase7}
La fase finale prevede il controllo e la rifinitura generale di tutto il lavoro. L'obiettivo è perciò concludere il progetto nella sua interezza, assicurandosi di aver soddisfatto con successo i requisiti richiesti sia dal proponente che dal committente e consegnando quindi il prodotto finito in modo conforme a quanto richiesto dal quarto ed ultimo incontro di revisione programmato.
        
        \subsubsection{Periodo di svolgimento}
        9 aprile 2021 - 3 maggio 2021;
        
        \subsubsection{Figure professionali coinvolte}
            \begin{itemize}
                \item responsabile;
                \item amministratore;
                \item progettista;
                \item programmatore;
                \item analista;
                \item verificatore.
            \end{itemize}

        \subsubsection{Obiettivi}
        \begin{itemize}
            \item correzione della documentazione secondo le direttive del committente;
            \item aggiunta e/o finalizzazione degli aspetti rimasti in sospeso o eventualmente incompleti dalle fasi precedenti;
            \item verifica finale del codice e delle funzionalità del prodotto;
            \item controllo complessivo di ogni parte e collaudo interno del prodotto;
            \item aggiornamento, verifica e correzione generale dell'intera documentazione redatta fino a questo momento;
            \item consegna di tutto il materiale necessario, in vista dell'incontro con il committente per la Revisione di Accettazione.
        \end{itemize}