La scelta di tale modello porta i seguenti pro e contro:\\
\subsubsection{\textbf{Pro}}
\begin{itemize}
    \item facilita la generazione di software funzionante in modo rapido e precoce durante il ciclo di vita del software.
    \item riduce i costi di consegna iniziale;
    \item È più facile testare ed eseguire il debug durante un'iterazione piccola;
    \item le \glo{risorse} con le skills necessarie non sono disponibili
    \item più facile gestire il rischio perché i pezzi rischiosi vengono identificati e gestiti durante l'iterazione
    \item in questo modello il proponente può dare un feedback ad ogni build.
\end{itemize}
\subsubsection{\textbf{Contro}}
\begin{itemize}
    \item Ha bisogno di una buona pianificazione e design;
    \item Ha bisogno di una definizione chiara e completa dell'intero sistema prima che possa essere scomposto e costruito in modo incrementale.
\end{itemize}
\subsubsection{\textbf{Conclusione}}
Questa metodologia è risultata quindi ideale perchè:
\begin{itemize}
    \item il livello  di esperienza del gruppo \Gruppo{} è mediamente limitato;
    \item la maggior parte dei requisiti del capitolato C1 sono chiari e ben definiti;
    \item è necessario(potenzialmente) immettere in anticipo un prodotto sul mercato(visto il contesto del capitolato C1);
    \item l'azienda proponente \proponente{},si è dimostrata aperta al dialogo, dando la possibilita di contrattare modifiche ai \glo{requisiti} , qualora giustificati;
\end{itemize}
