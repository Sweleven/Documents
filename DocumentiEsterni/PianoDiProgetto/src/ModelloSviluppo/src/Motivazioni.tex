In un primo contatto informativo con l'azienda proponente \proponente{}, questa si è dimostrata molto aperta al dialogo, finanche alla possibilità di discutere di possibili modifiche ai \glo{requisiti} richiesti, qualora giustificati.\\
Si evince quindi come questo modello risulti abbastanza appropriato, poiché in esso è possibile ogni volta valutare lo stato del lavoro, gli elementi e le funzionalità che devono essere poi implementati. Qualora subentrassero difficoltà, data appunto la disponibilità del proponente e la natura stessa del modello, sarà anche possibile scegliere in itinere le aggiunte o modifiche da apportare. Tutto ciò ha un impatto decisamente meno gravoso sull'intero progetto perché si cerca comunque di garantire per quanto possibile la stabilità del prodotto cosicchè, quando necessario, si potrà sempre ritornare agilmente ad uno stato precedente. \\
Tale approccio risulta inoltre più adatto anche per il livello mediamente limitato di esperienza del gruppo \Gruppo{}, dove la gran parte dei membri si trova ad affrontare per la prima volta un progetto di tali dimensioni.