Il modello incrementale presenta i seguenti pro e contro:\\
\subsubsection{Pro}
\begin{itemize}
    \item facilita la generazione di software funzionante in modo rapido e precoce durante il ciclo di vita del software.
    \item riduce i costi di consegna iniziale;
    \item è più facile testare ed eseguire il debug durante un'iterazione breve;
    \item aiuta anche i collaboratori con meno esperienza e abilità favorendo l'apprendimento graduale;
    \item più facile gestire il rischio poiché le parti più complesse vengono identificate e gestite durante l'iterazione;
    \item in questo modello il proponente può dare un feedback ad ogni build.
\end{itemize}

\subsubsection{Contro}
\begin{itemize}
    \item richiede una buona pianificazione e \glo{design};
    \item necessita di una definizione chiara e completa dell'intero sistema prima che possa essere scomposto e costruito in modo incrementale;
    \item è richiesta molta esperienza nell'applicazione del modello (che il gruppo sta ancora maturando) per la sua completa padronanza.
\end{itemize}

\subsubsection{Conclusione}
Questa metodologia è risultata quindi ideale perché:
\begin{itemize}
    \item è necessario immettere il prima possibile il prodotto sul mercato, considerato il contesto e la possibile concorrenza nello sviluppo di applicazioni simili a quella proposta dal capitolato C1;
    \item la maggior parte dei requisiti presentati nel capitolato C1 sono chiari e ben definiti;
    \item il gruppo ha dimostrato uno spiccato interesse per l'applicazione di tale modello nonostante la scarsa esperienza, poiché l'obiettivo preposto è ambizioso e stimolante;
    \item l'azienda proponente \proponente{}, si è dimostrata aperta al dialogo, dando la possibilità di contrattare eventuali modifiche ai \glo{requisiti}, qualora giustificate.
\end{itemize}
