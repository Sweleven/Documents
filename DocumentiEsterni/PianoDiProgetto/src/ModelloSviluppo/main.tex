\subsection{Scelta del modello}   
    Il gruppo \Gruppo{} ha deciso di pianificare il lavoro seguendo un modello essenzialmente di tipo \textbf {\glo{incrementale}}, poiché permette di sviluppare il prodotto tramite rilasci successivi. \\
    Questi rilasci saranno perciò in questo documento definiti non nel dettaglio, ma verrà eseguita una suddivisione preliminare in fasi progressive.

\subsection{Motivazioni della scelta}
    In un primo contato informativo con l'azienda proponente \proponente{}, questa si è detta molto aperta al dialogo, finanche alla possibilità di discutere di possibili modifiche ai requisiti richiesti, qualora giustificati.\\
    Si evince quindi come questo modello risulta abbastanza appropriato, poichè è possibile ad ogni fase valutare gli elementi e le funzionalità che devono essere poi implementati. Avvicinandosi così in qualche modo anche ad un modello che accarezza il tipo \glo{SCUM}, qualora subentrassero difficoltà e data appunto la disponibilità del proponente, sarà anche possibile, ogni qual volta necessario e motivato, discutere appunto con l'azienda \proponente{} eventuali cambiamenti da poter apportare in ogni passo. \\
    Per lo stesso motivo, la possibilità di valutare per gradi lo stato del lavoro e decidere incrementalmente solo poi in maniera più completa i passi successivi, risulta inoltre più adatta anche per il livello limitato di esperienza del gruppo \Gruppo{}, in cui la gran parte dei membri si trova ad affrontare per la prima volta un progetto di tali dimensioni.

\subsection{Definizione degli incrementi}
    Nei primi incrementi ci si concentrerà quindi nell'identificare e definire i requisiti principali richiesti dal prodotto e garantirne perciò le funzionalità essenziali. Nelle successive fasi si proseguirà dunque nell'aggiunta e nella valutazione graduale delle altre funzionalità richieste, fino a giungere alla realizzazione del prodotto completo. \\
    Nella pianificazione di ogni fase verranno man mano definiti in modo più specifico tutti gli elementi necessari da sviluppare in essa (qui precedentemente identificati in modo più generico), suddividendoli ognuno in altrettanti piccoli compiti per garantire così una definizione più immediata, chiara ed agevolata degli obiettivi e permettere di conseguenza ogni volta un carico di lavoro non esageratamente oneroso. \\
    Per ogni incremento (specialmente nella fase di sviluppo) si cercherà comunque di garantire per quanto possibile la stabilità del prodotto, cosicchè nel caso in cui ci dovessero essere valutazioni negative o modifiche necessarie da fare si potrà sempre ritornare agilmente allo stato immediatamente precedente, senza dover quindi stravolgere il progetto.

\subsection{Organizzazione delle fasi}
    Di seguito viene qui riportata la struttura generale in cui verrà organizzato l'intero lavoro, nel corso del quale si intende mantenere la documentazione relativa quanto più aggiornata possibile. \\
    Il processo si svilupperà in sette fasi principali con relativi obiettivi da raggiungere, i cui aspetti incrementali verrano definiti più in dettaglio nel prossimo capitolo, relativo alla pianificazione:

    \subsubsection{Fase 1 - Organizzazione preliminare}
        \begin{itemize}
            \item scelta degli strumenti ed organizzazione dell'ambiente di lavoro;
            \item definizione delle Norme di Progetto e Studio di Fattibilità;
            \item attività di verifica.
        \end{itemize}
            
    \subsubsection{Fase 2 - Definizione e stesura degli obiettivi di ingresso}
        \begin{itemize}
            \item analisi dei requisiti;
            \item piano di qualifica;
            \item piano di progetto e organigramma;
            \item verifica della documentazione per la Revisione dei Requisiti.
        \end{itemize}
    
    \subsubsection{Fase 3 - Consolidamento delle componenti}
        \begin{itemize}
            \item attività di correzione;
            \item individuazione delle funzionalità essenziali e dei componenti principali;
            \item individuazione delle funzionalità e componenti secondarie;
            \item strutturazione generale del sistema;
            \item valutazione delle componenti e tecnologie richieste.
        \end{itemize}
    
    \subsubsection{Fase 4 - Progettazione e sviluppo}
        \begin{itemize}
            \item definizione completa della struttura tramite diagrammi \glo{UML};
            \item implementazione delle componenti essenziali individuate;
            \item progettazione di base delle interfacce utente;
            \item verifica del codice e delle funzionalità.
        \end{itemize}

    \subsubsection{Fase 5 - Sviluppo e rifinitura }
        \begin{itemize}
            \item implementazione delle componenti secondarie;
            \item rifinitura delle componenti e delle interfacce;
            \item verifica del codice e delle funzionalità;
            \item verifica della documentazione per la Revisione di Progettazione.
        \end{itemize}

    \subsubsection{Fase 6 - Qualifica e testing}
        \begin{itemize}
            \item attività di correzione;
            \item copertura dei test;
            \item stesura resoconto dei test;
            \item consuntivo generale;
            \item controllo complessivo della qualità;
            \item verifica della documentazione per la Revisione di Qualifica.
        \end{itemize}

    \subsubsection{Fase 7 - Revisione generale e collaudo}
        \begin{itemize}
            \item attività di correzione;
            \item controllo complessivo di ogni parte del prodotto;
            \item attività di collaudo;
            \item aggiornamento finale e verifica della documentazione per la Revisione di Accettazione;
        \end{itemize}
