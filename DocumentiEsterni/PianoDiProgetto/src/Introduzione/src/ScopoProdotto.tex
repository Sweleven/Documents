L'obiettivo principale del progetto \NomeProgetto{}, proposto dall'azienda \proponente{}, riguarda il \glo{tracciamento} dell'utilizzo delle postazioni di lavoro in un laboratorio informatico, applicabile ad un contesto sia lavorativo che accademico.
Il motivo è fortemente legato alla diffusione della pandemia di coronavirus che tutto il mondo sta vivendo ed affrontando in questo momento. Per garantire la sicurezza nell'utilizzo di ogni postazione, è richiesta quindi la creazione di un sistema software formato da due componenti principali:
\begin{itemize}
    \item \textbf{Applicazione mobile:} attraverso un sistema di autenticazione deve offrire la possibilità di visualizzare lo stato di una postazione (libera, occupata, igienizzata, non igienizzata), prenotarne una disponibile, registrare in tempo reale l'utilizzo della stessa, così come segnalarne l'eventuale igienizzazione svolta;
    \item \textbf{Server centrale:} dedicato alla gestione generale del sistema, attraverso un'apposita \glo{interfaccia grafica}, deve garantire la possibilità di creare, eliminare, abilitare, disabilitare postazioni di lavoro come anche di intere aule; memorizzare in maniera certificata ed immutabile lo storico di ogni postazione; monitorare in tempo reale lo stato di ogni postazione; gestire gli utenti all'interno del sistema.
\end{itemize}