{
    \rowcolors{2}{\evenRowColor}{\oddRowColor}
    \renewcommand{\arraystretch}{1.5}
    \centering
    \begin{longtable}{ c C{11cm} c }
        \caption{Elenco dei test di accettazione}\\
        \rowcolor{\primaryColor}
        \textcolor{\secondaryColor}{
        \textbf{Codice}}     & \textcolor{\secondaryColor}
        {\textbf{Descrizione}} & \textcolor{\secondaryColor}{\textbf{Esito}} \\


        TA1&
        L'utente autenticato deve poter essere in grado di creare una prenotazione.\newline
        All'utente autenticato viene chiesto di:
        \begin{itemize}
            \item navigare all'apposita sezione di creazione di una prenotazione;
            \item verificare che sia possibile creare una prenotazione specificando data e ora;
            \item verificare che la prenotazione passi ad uno stato di "pending";
            \item verificare che la prenotazione venga conferamta.
            \item [] OPPURE
            \item verificare che la prenotazione venga eliminata per un motivo specificato.
        \end{itemize}&
        NI\\
        TA2&
        L'utente autenticato deve poter essere in grado di modificare una prenotazione.\newline
        All'utente autenticato viene chiesto di:
        \begin{itemize}
            \item navigare all'apposita sezione di modifica di una prenotazione;
            \item verificare che sia possibile modificare una prenotazione specificando una nuova data e ora;
            \item verificare che la prenotazione passi ad uno stato di "pending";
            \item verificare che la prenotazione venga conferamta.
            \item [] OPPURE
            \item verificare che la prenotazione venga rifiutata per un errore specificato.
        \end{itemize}&
        NI\\
        TA3&
        L'utente autenticato deve poter essere in grado di eliminare una prenotazione.\newline
        All'utente autenticato viene chiesto di:
        \begin{itemize}
            \item navigare all'apposita sezione di cancellazione di una prenotazione;
            \item verificare che sia possibile eliminaree una prenotazione effettuata in precedenza per una specifica data e ora;
            \item verificare che la prenotazione venga eliminata.
        \end{itemize}&
        NI\\
        TA4&
        L'utente autenticato deve poter essere in grado di visualizzare le prenotazioni attive nel sistema.\newline
        All'utente autenticato viene chiesto di:
        \begin{itemize}
            \item navigare all'apposita sezione di visualizzazione delle prenotazioni;
            \item verificare che siano elencate correttamente tutte le sue prenotazioni attive.
            \item [] OPPURE
            \item verificare che sia visualizzato un messaggio di errore indicando che non esistono prenotazioni attive.
        \end{itemize}&
        NI\\
        TA5&
        L'amministratore deve poter essere in grado di impedire prenotazioni su una postazione.\newline
        All'amministratore viene chiesto di:
        \begin{itemize}
            \item navigare all'apposita sezione di blocco delle prenotazioni per una postazione;
            \item bloccare la possibilità di prenotare una postazione;
            \item verificare che la postazione non permetta la prenotazione.
        \end{itemize}&
        NI\\
        TA6&
        L'amministratore, l'utente autenticato e l'igienizzatore devono poter essere in grado di vedere e gestire lo stato di tutte le postazioni a cui ha accesso.\newline
        All'amministratore, all'utente autenticato e all'igienizzatore viene chiesto di:
        \begin{itemize}
            \item navigare all'apposita sezione di gestione delle postazioni;
            \item verificare che possa gestire le postazioni a cui ha accesso.
        \end{itemize}&
        NI\\
        TA7&
        L'amministratore, l'utente autenticato e l'igienizzatore devono poter essere in grado di marcare come igienizzata una postazione.\newline
        All'amministratore, all'utente autenticato e all'igienizzatore viene chiesto di:
        \begin{itemize}
            \item navigare all'apposita sezione di gestione della postazione;
            \item verificare che possa marcare come igienizzata la postazione;
            \item verificare il corretto esito della marcatura.
            \item [] OPPURE
            \item verificare l'esito negativo della marcatura dovuto ad un errore specificato.
        \end{itemize}&
        NI\\
        TA8&
        L'amministratore e l'utente autenticato devono poter essere in grado di marcare come sporca una postazione.\newline
        All'amministratore e all'utente autenticato viene chiesto di:
        \begin{itemize}
            \item navigare all'apposita sezione di gestione della postazione;
            \item verificare che possa marcare come sporca la postazione utilizzata;
            \item verificare il corretto esito della marcatura.
            \item [] OPPURE
            \item verificare l'esito negativo della marcatura dovuto ad un errore specificato.
        \end{itemize}&
        NI\\
        TA9&
        L'amministratore e l'utente autenticato devono poter essere in grado di marcare come libera una postazione.\newline
        All'amministratore e all'utente autenticato viene chiesto di:
        \begin{itemize}
            \item navigare all'apposita sezione di gestione della postazione;
            \item verificare che possa marcare come libera la postazione utilizzata;
            \item verificare il corretto esito della marcatura.
            \item [] OPPURE
            \item verificare l'esito negativo della marcatura dovuto ad un errore specificato.
        \end{itemize}&
        NI\\
        TA10&
        L'amministratore e l'utente autenticato devono poter essere in grado di marcare come occupata una postazione.\newline
        All'amministratore e all'utente autenticato viene chiesto di:
        \begin{itemize}
            \item navigare all'apposita sezione di gestione della postazione;
            \item verificare che possa marcare come occupata la postazione utilizzata;
            \item verificare il corretto esito della marcatura.
            \item [] OPPURE
            \item verificare l'esito negativo della marcatura dovuto ad un errore specificato.
        \end{itemize}&
        NI\\
        TA11&
        L'amministratore e l'utente autenticato devono poter essere in grado di confermare l'occupazione di una postazione prenotata.\newline
        All'amministratore e all'utente autenticato viene chiesto di:
        \begin{itemize}
            \item aprire la schermata pop-up di conferma della prenotazione;
            \item verificare l'esito della conferma della prenotazione.
        \end{itemize}&
        NI\\
        TA12&
        L'amministratore e l'igienizzatore devono poter essere in grado di disabilitare una postazione.\newline
        All'amministratore e all'igienizzatore viene chiesto di:
        \begin{itemize}
            \item navigare all'apposita sezione di gestione delle postazioni;
            \item marcare la postazione desiderata come disabilitata;
            \item verificare che la postazione non sia né prenotabile né servibile.
        \end{itemize}&
        NI\\
        TA13&
        L'amministratore e l'igienizzatore devono poter essere in grado di riabilitare una postazione precedentemente disabilitata.\newline
        All'amministratore e all'igienizzatore viene chiesto di:
        \begin{itemize}
            \item navigare all'apposita sezione di gestione delle postazioni;
            \item marcare la postazione disabilitata desiderata come abilitata;
            \item verificare che la postazione sia ora prenotabile e servibile.
        \end{itemize}&
        NI\\
        TA14&
        L'igienizzatore deve poter essere in grado di gestire le stanze o postazioni a cui ha accesso.\newline
        All'igienizzatore viene chiesto di:
        \begin{itemize}
            \item navigare all'apposita sezione di gestione delle postazioni e delle stanze;
            \item marcare la postazione o la stanza desiderata come igienizzata o come sporca;
            \item verificare che la postazione o la stanza sia ora nello stato precedentemente selezionato.
        \end{itemize}&
        NI\\
        
        
    \end{longtable}
}