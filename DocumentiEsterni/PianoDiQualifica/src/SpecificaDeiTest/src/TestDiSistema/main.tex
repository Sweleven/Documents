I testi di sistema hanno lo scopo di testare l’applicativo prodotto nella sua interezza, portando alla luce eventuali problematiche sulle parti del codice sviluppate da altri sviluppatori.
{
    \rowcolors{2}{\evenRowColor}{\oddRowColor}
    \renewcommand{\arraystretch}{1.5}
    \centering
    \begin{longtable}{ c C{11cm} c }
        \caption{Elenco dei test di sistema}\\
        \rowcolor{\primaryColor}
        \textcolor{\secondaryColor}{
        \textbf{Codice}}     & \textcolor{\secondaryColor}
        {\textbf{Descrizione}} & \textcolor{\secondaryColor}{\textbf{Esito}} \\

        TS1&
        L'amministratore deve poter essere in grado di registrare nuovi utenti.\newline
        All'amministratore viene chiesto di:
        \begin{itemize}
            \item poter accedere alla pagina di registrazione;
            \item selezionare il tipo di utente scelto;
            \item inserire i dati rispetto al tipo di utente scelto.
            
        \end{itemize}&
        NI\\

      TS1.1&
        L'amministratore deve poter essere in grado di registrare altri Amministratori.\newline
        All'amministratore viene chiesto di:
        \begin{itemize}
            \item accedere alla pagina di registrazione e specificare il tipo di utente come Amministratore;
            \item inserire il nome del nuovo utente;
            \item inserire il cognome del nuovo utente;
            \item inserire l'indirizzo e-mail del nuovo utente;
            \item inserire una password temporanea.
            
        \end{itemize}&
        NI\\

        TS1.2&
        L'amministratore deve poter essere in grado di registrare utenti come dipendenti.\newline
        All'amministratore viene chiesto di:
        \begin{itemize}
            \item accedere alla pagina di registrazione e specificare il tipo di utente come Dipendente;
            \item inserire il nome del nuovo utente;
            \item inserire il cognome del nuovo utente;
            \item inserire l'indirizzo e-mail del nuovo utente;
            \item inserire una password temporanea.
            
        \end{itemize}&
        NI\\

        TS1.3&
        L'amministratore deve poter essere in grado di registrare utenti come Igienizzatori.\newline
        All'amministratore viene chiesto di:
        \begin{itemize}
            \item accedere alla pagina di registrazione e specificare il tipo di utente come Igienizzatore;
            \item inserire il nome del nuovo utente;
            \item inserire il cognome del nuovo utente;
            \item inserire l'indirizzo e-mail del nuovo utente;
            \item inserire una password temporanea.
            
        \end{itemize}&
        NI\\

        TS2.0&
        Il dipendente deve essere in grado di accedere all'applicazione.\newline
        Al dipendente viene chiesto di:
        \begin{itemize}
            \item accedere alla pagina di Login;
            \item inserire l'indirizzo e-mail;
            \item inserire la password.
        \end{itemize}&
        NI\\

        TS3.0&
        Il dipendente deve essere in grado di scannarizzare il tag RFID.\newline
        Al dipendente viene chiesto di:
        \begin{itemize}
            \item accedere alla pagina dedicata;
            \item selezionare la scannerizzazione;
            \item appoggiare il telefono al tag RFID;
            \item verificare la corretta scannerizzazione del tag RFID;
            \item [] oppure
            \item visualizzazione dell'errore della scannerizzazione non avvenuta correttamente.
        \end{itemize}&
        NI\\

        TS3.1&
        Il dipendente deve essere verificare lo stato di una postazione tramite la scannarizzazione del tag RFID.\newline
        Al dipendente viene chiesto di:
        \begin{itemize}
            \item accedere alla pagina dedicata;
            \item selezionare la scannerizzazione;
            \item appoggiare il telefono al tag RFID;
            \item verificare la corretta scannerizzazione del tag RFID;
            \item [] oppure
            \item visualizzazione dell'errore della scannerizzazione non avvenuta correttamente.
            \item verificare lob stato della postazione.
        \end{itemize}&
        NI\\


        TS4.0&
        Il dipendente deve poter essere in grado di creare una prenotazione.\newline
        Al dipendente viene chiesto di:
        \begin{itemize}
            \item navigare all'apposita sezione di creazione di una prenotazione;
            \item verificare che sia possibile creare una prenotazione specificando data e ora;
            \item verificare che la prenotazione passi ad uno stato di "pending";
            \item verificare che la prenotazione venga confermata.
            \item [] OPPURE
            \item verificare che la prenotazione venga eliminata per un motivo specificato.
        \end{itemize}&
        NI\\

        TS4.1&
        Il dipendente deve essere in grado di modificare una prenotazione.\newline
        Al dipendente viene chiesto di:
        \begin{itemize}
            \item navigare all'apposita sezione di modifica di una prenotazione;
            \item verificare che sia possibile modificare una prenotazione specificando una nuova data e ora;
            \item verificare che la prenotazione passi ad uno stato di "pending";
            \item verificare che la prenotazione venga conferamta.
            \item [] OPPURE
            \item verificare che la prenotazione venga rifiutata per un errore specificato.
        \end{itemize}&
        NI\\
        TS4.2&
        Il dipendente deve essere in grado di eliminare una prenotazione.\newline
        Al dipendente viene chiesto di:
        \begin{itemize}
            \item navigare all'apposita sezione di cancellazione di una prenotazione;
            \item verificare che sia possibile eliminare una prenotazione effettuata in precedenza per una specifica data e ora;
            \item verificare che la prenotazione venga eliminata.
        \end{itemize}&
        NI\\
        TS4.3&
        Il dipendente deve poter in grado di visualizzare le prenotazioni attive nel sistema.\newline
        Al dipendente viene chiesto di:
        \begin{itemize}
            \item navigare all'apposita sezione di visualizzazione delle prenotazioni;
            \item verificare che siano elencate correttamente tutte le sue prenotazioni attive.
            \item [] OPPURE
            \item verificare che sia visualizzato un messaggio di errore indicando che non esistono prenotazioni attive.
        \end{itemize}&
        NI\\

        TS4.4&
        Il dipendente deve poter in grado di visualizzare le prenotazioni attive nel sistema.\newline
        Al dipendente viene chiesto di:
        \begin{itemize}
            \item navigare all'apposita sezione di visualizzazione delle prenotazioni;
            \item verificare che siano elencate correttamente tutte le sue prenotazioni attive.
            \item [] OPPURE
            \item verificare che sia visualizzato un messaggio di errore indicando che non esistono prenotazioni attive.
        \end{itemize}&
        NI\\

        TS4.5&
        Il dipendente deve non essere in grado di prenotare una postazione già prenotata.\newline
        Al dipendente viene chiesto di:
        \begin{itemize}
            \item navigare all'apposita sezione di creazione di una prenotazione;
            \item verificare che sia possibile creare una prenotazione specificando data e ora;
            \item verificare che la prenotazione passi ad uno stato di "pending";
            \item verificare che la prenotazione non venga effettuata.
        \end{itemize}&
        NI\\

        TS4.6&
        Il dipendente deve poter essere in grado di vedere e gestire lo stato di tutte le postazioni a cui ha accesso.\newline
        Al dipendente viene chiesto di:
        \begin{itemize}
            \item navigare all'apposita sezione di gestione delle postazioni;
            \item verificare che possa gestire le postazioni a cui ha accesso.
        \end{itemize}&
        NI\\

        TS5.0&
        L'amministratore deve essere in grado di accedere all'applicazione web.\newline
        All'amministratore viene chiesto di:
        \begin{itemize}
            \item accedere alla pagina di Login;
            \item inserire l'indirizzo e-mail;
            \item inserire la password.
        \end{itemize}&
        NI\\

        TS6.0&
        L'amministratore deve poter essere in grado di creare nuove postazioni.\newline
        All'amministratore viene chiesto di:
        \begin{itemize}
            \item navigare all'apposita sezione e selezionare il tipo di postazione;
            \item inserire i dati richiesti;
            \item confermare l'avvenuta creazione della postazione.
        \end{itemize}&
        NI\\

        TS6.1&
        L'amministratore deve poter essere in grado di creare nuove stanze.\newline
        All'amministratore viene chiesto di:
        \begin{itemize}
            \item navigare all'apposita sezione e selezionare il tipo di postazione come stanza;
            \item inserire il codice della stanza;
            \item verificare che non ci siano messaggi di errore;
            \item confermare la creazione della stanza.
            \item verificare l'avvenuta creazione della stanza.
        \end{itemize}&
        NI\\

        TS6.2&
        L'amministratore deve poter essere in grado di creare nuove postazioni all'interno delle stanze.\newline
        All'amministratore viene chiesto di:
        \begin{itemize}
            \item navigare all'apposita sezione e selezionare la stanza dove è rischiesta la creazione della postazione;
            \item inserire il codice della postazione;
            \item verificare che non ci siano messaggi di errore;
            \item confermare la creazione della postazione;
            \item verificare l'avvenuta creazione della postazione.
        \end{itemize}&
        NI\\

        TS7.0&
        L'amministratore deve poter essere in grado di modificare le postazioni.\newline
        All'amministratore viene chiesto di:
        \begin{itemize}
            \item navigare all'apposita sezione e selezionare il tipo di postazione;
            \item inserire i dati richiesti;
            \item confermare l'avvenuta modifica della postazione.
        \end{itemize}&
        NI\\

        TS7.1&
        L'amministratore deve poter essere in grado di modificare le stanze.\newline
        All'amministratore viene chiesto di:
        \begin{itemize}
            \item navigare all'apposita sezione e selezionare la stanza da modificare;
            \item premere il pulsante modifica;
            \item inserire i nuovi dati;
            \item verificare che non ci siano messaggi di errore per la modifica;
            \item confermare la modifica della stanza;
            \item verificare l'avvenuta modifica della stanza.
        \end{itemize}&
        NI\\

        TS7.2&
        L'amministratore deve poter essere in grado di modificare le postazioni all'interno delle stanze.\newline
        All'amministratore viene chiesto di:
        \begin{itemize}
            \item navigare all'apposita sezione e selezionare la stanza dove è presente la postazione da modificare;
            \item premere il pulsante modifica;
            \item inserire i nuovi dati;
            \item verificare che non ci siano messaggi di errore per la modifica;
            \item confermare la modifica della postazione;
            \item verificare l'avvenuta modifica della postazione.
        \end{itemize}&
        NI\\

        TS8.0&
        L'amministratore deve poter essere in grado di eliminare le postazioni.\newline
        All'amministratore viene chiesto di:
        \begin{itemize}
            \item navigare all'apposita sezione e selezionare il tipo di postazione;
            \item eliminare la postazione;
            \item confermare l'eliminazione della postazione;
            \item verificare l'avvenuta eliminazione della postazione.
        \end{itemize}&
        NI\\

        TS8.1&
        L'amministratore deve poter essere in grado di eliminare le stanze.\newline
        All'amministratore viene chiesto di:
        \begin{itemize}
            \item navigare all'apposita sezione e selezionare la stanza da eliminare;
            \item premere il pulsante elimina;
            \item verificare che non ci siano messaggi di errore per la modifica;
            \item confermare l'eliminazione della stanza;
            \item verificare l'avvenuta eliminazione della stanza.
        \end{itemize}&
        NI\\

        TS8.2&
        L'amministratore deve poter essere in grado di eliminare le postazioni all'interno delle stanze.\newline
        All'amministratore viene chiesto di:
        \begin{itemize}
            \item navigare all'apposita sezione e selezionare la stanza dove è presente la postazione da eliminare;
            \item premere il pulsante elimina;
            \item verificare che non ci siano messaggi di errore per la modifica;
            \item confermare l'eliminazione della postazione;
            \item verificare l'avvenuta eliminazione della postazione.
        \end{itemize}&
        NI\\

        TS9.0&
        L'amministratore deve poter essere in grado di disabilitare le postazioni.\newline
        All'amministratore viene chiesto di:
        \begin{itemize}
            \item navigare all'apposita sezione e selezionare il tipo di postazione;
            \item disabilitare la postazione;
            \item confermare la disabilitazione della postazione;
            \item verificare l'avvenuta disabilitazione della postazione.
        \end{itemize}&
        NI\\

        TS9.1&
        L'amministratore deve poter essere in grado di disabilitare le stanze.\newline
        All'amministratore viene chiesto di:
        \begin{itemize}
            \item navigare all'apposita sezione e selezionare la stanza da disabilitare;
            \item premere il pulsante disabilitare;
            \item verificare che non ci siano messaggi di errore per la modifica;
            \item confermare la disabilitazione della stanza;
            \item verificare l'avvenuta disabilitazione della stanza.
        \end{itemize}&
        NI\\

        TS9.2&
        L'amministratore deve poter essere in grado di disabilitare le postazioni all'interno delle stanze.\newline
        All'amministratore viene chiesto di:
        \begin{itemize}
            \item navigare all'apposita sezione e selezionare la stanza dove è presente la postazione da disabilitare;
            \item premere il pulsante disabilitare;
            \item verificare che non ci siano messaggi di errore per la modifica;
            \item confermare la disabilitazione della postazione;
            \item verificare l'avvenuta disabilitazione della postazione.
        \end{itemize}&
        NI\\

        TS10.0&
        L'amministratore deve poter essere in grado di impedire le prenotazioni.\newline
        All'amministratore viene chiesto di:
        \begin{itemize}
            \item navigare all'apposita sezione di blocco delle prenotazioni;
            \item bloccare la possibilità di prenotare;
            \item verificare che la postazione non permetta la prenotazione.
        \end{itemize}&
        NI\\

        TS10.1&
        L'amministratore deve poter essere in grado di impedire prenotazioni di una stanza.\newline
        All'amministratore viene chiesto di:
        \begin{itemize}
            \item navigare all'apposita sezione di blocco delle prenotazioni della stanza;
            \item bloccare la possibilità di prenotare in quella determinata stanza;
            \item verificare che la stanza non permetta le prenotazione.
        \end{itemize}&
        NI\\

        TS11&
        L'amministratore deve poter essere in grado di impedire prenotazioni di una singola postazione.\newline
        All'amministratore viene chiesto di:
        \begin{itemize}
            \item navigare all'apposita sezione di blocco delle prenotazioni della postazione;
            \item bloccare la possibilità di prenotare quella postazione;
            \item verificare che la postazione non permetta le prenotazione.
        \end{itemize}&
        NI\\

        TS12&
        L'amministratore deve poter essere in grado di vedere e gestire lo stato di tutte le postazioni a cui ha accesso.\newline
        All'amministratore viene chiesto di:
        \begin{itemize}
            \item navigare all'apposita sezione di gestione delle postazioni;
            \item verificare che possa gestire le postazioni a cui ha accesso.
        \end{itemize}&
        NI\\

        TS13&
        L'amministratore deve poter essere in grado di riabilitare una postazione precedentemente disabilitata.\newline
        All'amministratore viene chiesto di:
        \begin{itemize}
            \item navigare all'apposita sezione di gestione delle postazioni;
            \item marcare la postazione disabilitata desiderata come abilitata;
            \item verificare che la postazione sia ora prenotabile e servibile.
        \end{itemize}&
        NI\\

        TS14&
        L'amministratore deve poter essere in grado di monitorare in ogni momento il numero di dipendenti presenti in tutte le postazioni e nella
        stanza nel suo complesso.\newline
        All'amministratore viene chiesto di:
        \begin{itemize}
            \item navigare all'apposita sezione di gestione delle postazioni;
            \item selezionare il monitoraggio in tempo reale;
            \item visualizzare la presenza delle persone all'interno della stanza.
        \end{itemize}&
        NI\\

        TS15&
        L'amministratore deve poter essere in grado di effettuare ricerche sugli accessi e sulle postazioni occupate da uno specifico dipendente.\newline
        All'amministratore viene chiesto di:
        \begin{itemize}
            \item navigare all'apposita sezione di gestione degli utenti;
            \item aprire la barra di ricerca;
            \item inserire i dati dell'utente;
            \item visualizzare lon storico dell'utente.
        \end{itemize}&
        NI\\

        TS16&
        L'igienizzatore deve essere in grado di accedere all'applicazione mobile.\newline
        All'igienizzatore viene chiesto di:
        \begin{itemize}
            \item accedere alla pagina di Login;
            \item inserire l'indirizzo e-mail;
            \item inserire la password.
        \end{itemize}&
        NI\\

        TS17&
        L'igienizzatore deve essere in grado di visualizzare le stanze dove c'è stato almeno un dipendente.\newline
        All'igienizzatore viene chiesto di:
        \begin{itemize}
            \item navigare nell'apposita sezione;
            \item visualizzare lo stato delle stanze;
        \end{itemize}&
        NI\\

        TS18&
        L'igienizzatore deve essere in grado di marcare le stanze dove ha effettuato l'igienizzazione.\newline
        All'igienizzatore viene chiesto di:
        \begin{itemize}
            \item navigare nell'apposita sezione;
            \item selezionare la stanza igienizzata;
            \item confermare l'igienizzazione della stanza;
            \item verificare l'avvenuta marcazione della stanza.
        \end{itemize}&
        NI\\

        TS18.1&
        L'igienizzatore deve essere in grado marcare la stanza come igienizzata tramite la scannarizzare del tag RFID.\newline
        All'igienizzatore viene chiesto di:
        \begin{itemize}
            \item accedere alla pagina dedicata;
            \item selezionare la scannerizzazione;
            \item appoggiare il telefono al tag RFID;
            \item verificare la corretta scannerizzazione del tag RFID;
            \item [] oppure
            \item visualizzazione dell'errore della scannerizzazione non avvenuta correttamente;
            \item confermare l'igienizzazione della stanza;
            \item verificare l'avvenuta conferma dell'igienizzazione della stanza.
        \end{itemize}&
        NI\\

        TS19&
        Il dipendente deve essere in grado di marcare la postazione dove ha utilizzato il kit pulizia come l'igienizzato.\newline
        Al dipendente viene chiesto di:
        \begin{itemize}
            \item navigare nell'apposita sezione;
            \item selezionare la stanza;
            \item selezionare la postazione;
            \item confermare l'igienizzazione della postazione;
            \item verificare l'avvenuta marcazione della postazione.
        \end{itemize}&
        NI\\

        TS19.1&
        Il dipendente deve essere in grado di marcare la postazione dove ha utilizzato il kit pulizia come l'igienizzato tramite tag RFID.\newline
        Al dipendente viene chiesto di:
        \begin{itemize}
            \item accedere alla pagina dedicata;
            \item selezionare la scannerizzazione;
            \item appoggiare il telefono al tag RFID;
            \item verificare la corretta scannerizzazione del tag RFID;
            \item [] oppure
            \item visualizzazione dell'errore della scannerizzazione non avvenuta correttamente;
            \item confermare l'igienizzazione della postazione;
            \item verificare l'avvenuta conferma dell'igienizzazione della postazione.
        \end{itemize}&
        NI\\

        TS20&
        Il sistema deve marcare come sporca una postazione che \`{e} stata occupata.\newline
        Quando viene occupata una postazione il sistema deve:
        \begin{itemize}
            \item marcare la postazione come "sporca";
            \item rendere la postazione non prenotabile fin quando non risulter\`{a} "pulita".
        \end{itemize}&
        NI\\
        TS21&
        Il sistema deve essere in grado di marcare come "libera" una postazione igienizzata.\newline
        \begin{itemize}
            \item La postazione \`{e} prenotabile.
        \end{itemize}&
        NI\\
        
    \end{longtable}
}