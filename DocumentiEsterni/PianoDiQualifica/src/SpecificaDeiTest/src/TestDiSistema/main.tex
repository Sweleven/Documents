I testi di sistema hanno lo scopo di testare l’applicativo prodotto nella sua interezza, portando alla luce eventuali problematiche sulle parti del codice sviluppate da altri sviluppatori.
\newpage
{
    \rowcolors{2}{\evenRowColor}{\oddRowColor}
    \renewcommand{\arraystretch}{1.5}
    \centering
    \begin{longtable}{ c C{11cm} c }
        \caption{Elenco dei test di sistema}\\
        \rowcolor{\primaryColor}
        \textcolor{\secondaryColor}{
        \textbf{Codice}}     & \textcolor{\secondaryColor}
        {\textbf{Descrizione}} & \textcolor{\secondaryColor}{\textbf{Esito}} \\

        TS1&
        \begin{flushleft}
            L'amministratore deve poter essere in grado di registrare nuovi utenti.
            All'amministratore viene chiesto di:
        \end{flushleft}
        
        \begin{itemize}
            \item poter accedere alla pagina di registrazione;
            \item selezionare il tipo di utente scelto;
            \item inserire i dati rispetto al tipo di utente scelto.
            
        \end{itemize}&
        NI\\

      TS1.1&
      \begin{flushleft}
            L'amministratore deve poter essere in grado di registrare altri amministratori.
            All'amministratore viene chiesto di:
      \end{flushleft}
        \begin{itemize}
            \item accedere alla pagina di registrazione e specificare il tipo di utente come amministratore;
            \item inserire l'indirizzo e-mail del nuovo utente.
            
        \end{itemize}&
        NI\\

        TS1.2&
        \begin{flushleft}
            L'amministratore deve poter essere in grado di registrare utenti come dipendenti.
            All'amministratore viene chiesto di:
        \end{flushleft}
        \begin{itemize}
            \item accedere alla pagina di registrazione e specificare il tipo di utente come dipendente;
            \item inserire l'indirizzo e-mail del nuovo utente.
            
        \end{itemize}&
        NI\\

        TS1.3&
        \begin{flushleft}
            L'amministratore deve poter essere in grado di registrare utenti come igienizzatori.
            All'amministratore viene chiesto di:
        \end{flushleft}
        \begin{itemize}
            \item accedere alla pagina di registrazione e specificare il tipo di utente come igienizzatore;
            \item inserire l'indirizzo e-mail del nuovo utente.
            
        \end{itemize}&
        NI\\

        TS2&
        \begin{flushleft}
             Il dipendente deve essere in grado di accedere all'applicazione mobile e autenticarsi con le proprie credenziali.
             Al dipendente viene chiesto di:
        \end{flushleft}
        \begin{itemize}
            \item accedere alla pagina di login;
            \item inserire l'indirizzo e-mail;
            \item inserire la password.
        \end{itemize}&
        NI\\

        TS3&
        \begin{flushleft}
            Il dipendente deve essere in grado di scannarizzare il tag RFID.
            Al dipendente viene chiesto di:
        \end{flushleft}
        \begin{itemize}
            \item accedere alla pagina dedicata;
            \item selezionare la scannerizzazione;
            \item appoggiare il telefono al tag RFID;
            \item verificare la corretta scannerizzazione del tag RFID;
            \item [] oppure
            \item visualizzazione dell'errore della scannerizzazione non avvenuta correttamente.
        \end{itemize}&
        NI\\

        TS3.1&
        \begin{flushleft}
            Il dipendente deve verificare lo stato di una postazione tramite la scannarizzazione del tag RFID.\newline
            Al dipendente viene chiesto di:
        \end{flushleft}
        \begin{itemize}
            \item accedere alla pagina dedicata;
            \item selezionare la scannerizzazione;
            \item appoggiare il telefono al tag RFID;
            \item verificare la corretta scannerizzazione del tag RFID;
            \item [] oppure
            \item visualizzazione dell'errore relativo alla scannerizzazione non avvenuta correttamente.
            \item verificare lo stato della postazione.
        \end{itemize}&
        NI\\


        TS4&
        \begin{flushleft}
            Il dipendente deve poter essere in grado di creare una prenotazione.
            Al dipendente viene chiesto di:
        \end{flushleft}
        \begin{itemize}
            \item navigare nell'apposita sezione di creazione di una prenotazione;
            \item verificare che sia possibile creare una prenotazione specificando data e ora;
            \item verificare che la prenotazione passi ad uno stato di "pending";
            \item verificare che la prenotazione venga confermata.
            \item [] oppure
            \item verificare che la prenotazione venga eliminata per un motivo specificato.
        \end{itemize}&
        NI\\

        TS4.1&
        \begin{flushleft}
            Il dipendente deve essere in grado di modificare una prenotazione.
            Al dipendente viene chiesto di:
        \end{flushleft}
        \begin{itemize}
            \item navigare nell'apposita sezione di modifica di una prenotazione;
            \item verificare che sia possibile modificare una prenotazione specificando una nuova data e ora;
            \item verificare che la prenotazione passi ad uno stato di "pending";
            \item verificare che la prenotazione venga conferamta.
            \item [] OPPURE
            \item verificare che la prenotazione venga rifiutata per un errore specificato.
        \end{itemize}&
        NI\\

        TS4.2&
        \begin{flushleft}
            Il dipendente deve essere in grado di eliminare una prenotazione.
            Al dipendente viene chiesto di:
        \end{flushleft}
        \begin{itemize}
            \item navigare nell'apposita sezione di cancellazione di una prenotazione;
            \item verificare che sia possibile eliminare una prenotazione effettuata in precedenza per una specifica data e ora;
            \item verificare che la prenotazione venga eliminata.
        \end{itemize}&
        NI\\

        TS4.3&
        \begin{flushleft}
            Il dipendente deve poter in grado di visualizzare le prenotazioni da lui effettuate.
            Al dipendente viene chiesto di:
        \end{flushleft}
        \begin{itemize}
            \item navigare nell'apposita sezione di visualizzazione delle prenotazioni;
            \item verificare che siano elencate correttamente tutte le sue prenotazioni.
        \end{itemize}&
        NI\\

        TS4.4&
        \begin{flushleft}
            Il dipendente deve poter in grado di visualizzare le prenotazioni attive nel sistema.
            Al dipendente viene chiesto di:
        \end{flushleft}
        \begin{itemize}
            \item navigare nell'apposita sezione di visualizzazione delle prenotazioni;
            \item verificare che siano elencate correttamente tutte le sue prenotazioni attive.
            \item [] oppure
            \item verificare che sia visualizzato un messaggio di errore indicando che non esistono prenotazioni attive.
        \end{itemize}&
        NI\\

        TS4.5&
        \begin{flushleft}
            Il dipendente non deve essere in grado di prenotare una postazione già prenotata.
            Al dipendente viene chiesto di:
        \end{flushleft}
        \begin{itemize}
            \item navigare nell'apposita sezione di creazione di una prenotazione;
            \item verificare che sia possibile creare una prenotazione specificando data e ora;
            \item verificare che la prenotazione passi ad uno stato di "pending";
            \item verificare che la prenotazione non venga effettuata.
        \end{itemize}&
        NI\\

        TS4.6&
        \begin{flushleft}
            L'amministratore deve poter essere in grado di vedere e gestire lo stato di tutte le postazioni.
            All'amministratore viene chiesto di:
        \end{flushleft}
        \begin{itemize}
            \item navigare nell'apposita sezione di gestione delle postazioni;
            \item verificare che possa gestire le postazioni a cui ha accesso e verificarne lo stato.
        \end{itemize}&
        NI\\

        TS5&
        \begin{flushleft}
            L'amministratore deve essere in grado di accedere all'applicazione web.
            All'amministratore viene chiesto di:
        \end{flushleft}
        \begin{itemize}
            \item accedere alla pagina di login;
            \item inserire l'indirizzo e-mail;
            \item inserire la password.
        \end{itemize}&
        NI\\

        TS6&
        \begin{flushleft}
            L'amministratore deve poter essere in grado di creare nuove postazioni.
            All'amministratore viene chiesto di:
        \end{flushleft}
        \begin{itemize}
            \item navigare nell'apposita sezione e selezionare il tipo di postazione;
            \item inserire i dati richiesti;
            \item confermare l'avvenuta creazione della postazione.
        \end{itemize}&
        NI\\

        TS6.1&
        \begin{flushleft}
            L'amministratore deve poter essere in grado di creare nuove stanze.
            All'amministratore viene chiesto di:
        \end{flushleft}
        \begin{itemize}
            \item navigare nell'apposita sezione e selezionare la funzionalità di creazione di una nuova stanza;
            \item inserire il codice della stanza;
            \item verificare che non ci siano messaggi di errore;
            \item confermare la creazione della stanza.
            \item verificare l'avvenuta creazione della stanza.
        \end{itemize}&
        NI\\

        TS7&
        \begin{flushleft}
            L'amministratore deve poter essere in grado di modificare le postazioni.
            All'amministratore viene chiesto di:
        \end{flushleft}
        \begin{itemize}
            \item navigare nell'apposita sezione e selezionare il tipo di postazione;
            \item inserire i dati richiesti;
            \item confermare l'avvenuta modifica della postazione.
        \end{itemize}&
        NI\\

        TS7.1&
        \begin{flushleft}
            L'amministratore deve poter essere in grado di modificare le stanze.
            All'amministratore viene chiesto di:
        \end{flushleft}
        \begin{itemize}
            \item navigare nell'apposita sezione e selezionare la stanza da modificare;
            \item premere il pulsante modifica;
            \item inserire i nuovi dati;
            \item verificare che non ci siano messaggi di errore per la modifica;
            \item confermare la modifica della stanza;
            \item verificare l'avvenuta modifica della stanza.
        \end{itemize}&
        NI\\

        TS8&
        \begin{flushleft}
            L'amministratore deve poter essere in grado di eliminare le postazioni.
            All'amministratore viene chiesto di:
        \end{flushleft}
        \begin{itemize}
            \item navigare nell'apposita sezione e selezionare il tipo di postazione;
            \item eliminare la postazione;
            \item confermare l'eliminazione della postazione;
            \item verificare l'avvenuta eliminazione della postazione.
        \end{itemize}&
        NI\\

        TS8.1&
        \begin{flushleft}
            L'amministratore deve poter essere in grado di eliminare le stanze.
            All'amministratore viene chiesto di:
        \end{flushleft}
        \begin{itemize}
            \item navigare nell'apposita sezione e selezionare la stanza da eliminare;
            \item premere il pulsante elimina;
            \item verificare che non ci siano messaggi di errore per la modifica;
            \item confermare l'eliminazione della stanza;
            \item verificare l'avvenuta eliminazione della stanza.
        \end{itemize}&
        NI\\

        TS9&
        \begin{flushleft}
            L'amministratore deve poter essere in grado di disabilitare le postazioni.
            All'amministratore viene chiesto di:
        \end{flushleft}
        \begin{itemize}
            \item navigare nell'apposita sezione e selezionare il tipo di postazione;
            \item disabilitare la postazione;
            \item confermare la disabilitazione della postazione;
            \item verificare l'avvenuta disabilitazione della postazione;
            \item verificare che la postazione non sia prenotabile.
        \end{itemize}&
        NI\\

        TS9.1&
        \begin{flushleft}
            L'amministratore deve poter essere in grado di disabilitare le stanze.
            All'amministratore viene chiesto di:
        \end{flushleft}
        \begin{itemize}
            \item navigare nell'apposita sezione e selezionare la stanza da disabilitare;
            \item premere il relativo pulsante di disabilitazione della stanza;
            \item verificare che non ci siano messaggi di errore per la modifica;
            \item confermare la disabilitazione della stanza;
            \item verificare l'avvenuta disabilitazione della stanza;
            \item verificare che le postazioni interne alla stanza non siano prenotabili.
        \end{itemize}&
        NI\\


        TS10&
        \begin{flushleft}
            L'amministratore deve poter essere in grado di riabilitare una postazione precedentemente disabilitata.
            All'amministratore viene chiesto di:
        \end{flushleft}
        \begin{itemize}
            \item navigare nell'apposita sezione di gestione delle postazioni;
            \item marcare la postazione precedentemente disabilitata come abilitata;
            \item verificare che la postazione sia ora prenotabile e servibile.
        \end{itemize}&
        NI\\

        TS11&
        \begin{flushleft}
            L'amministratore deve poter essere in grado di monitorare in ogni momento il numero di dipendenti presenti in tutte le postazioni e nella stanza nel suo complesso.
            All'amministratore viene chiesto di:
        \end{flushleft}
        \begin{itemize}
            \item navigare nell'apposita sezione di gestione delle postazioni;
            \item selezionare il monitoraggio in tempo reale;
            \item visualizzare la presenza delle persone all'interno della stanza.
        \end{itemize}&
        NI\\

        TS12&
        \begin{flushleft}
            L'amministratore deve poter essere in grado di effettuare ricerche sugli accessi e sulle postazioni occupate da uno specifico dipendente.
            All'amministratore viene chiesto di:
        \end{flushleft}
        \begin{itemize}
            \item navigare nell'apposita sezione di gestione degli utenti;
            \item aprire la barra di ricerca;
            \item inserire i dati dell'utente;
            \item visualizzare lo storico dell'utente.
        \end{itemize}&
        NI\\

        TS13&
        \begin{flushleft}
            L'igienizzatore deve essere in grado di accedere all'applicazione mobile.
            All'igienizzatore viene chiesto di:
        \end{flushleft}
        \begin{itemize}
            \item accedere alla pagina di login;
            \item inserire l'indirizzo e-mail;
            \item inserire la password.
        \end{itemize}&
        NI\\

        TS14&
        \begin{flushleft}
            L'igienizzatore deve essere in grado di visualizzare le stanze da pulire.
            All'igienizzatore viene chiesto di:
        \end{flushleft}
        \begin{itemize}
            \item navigare nell'apposita sezione;
            \item visualizzare lo stato delle stanze;
        \end{itemize}&
        NI\\

        TS15&
        \begin{flushleft}
            L'igienizzatore deve essere in grado di marcare le stanze dove ha effettuato l'igienizzazione.
            All'igienizzatore viene chiesto di:
        \end{flushleft}
        \begin{itemize}
            \item navigare nell'apposita sezione;
            \item selezionare la stanza igienizzata;
            \item confermare l'igienizzazione della stanza;
            \item verificare l'avvenuta marcatura della stanza come igienizzata;
            \item verificare che tutte le postazioni interne alla stanza risultino igienizzate.
        \end{itemize}&
        NI\\

        TS15.1&
        \begin{flushleft}
            L'igienizzatore deve essere in grado marcare la stanza come igienizzata tramite la scannerizzazione del tag RFID.
            All'igienizzatore viene chiesto di:
        \end{flushleft}
        \begin{itemize}
            \item accedere alla pagina dedicata;
            \item selezionare la funzionalità di scannerizzazione;
            \item appoggiare il telefono al tag RFID;
            \item verificare la corretta scannerizzazione del tag RFID;
            \item confermare l'igienizzazione della stanza;
            \item verificare l'avvenuta conferma dell'igienizzazione della stanza.
        \end{itemize}&
        NI\\

        TS16&
        \begin{flushleft}
            Il dipendente deve essere in grado di marcare la postazione come igienizzata in seguito all'utilizzo dell'apposito kit di pulizia.
            Al dipendente viene chiesto di:
        \end{flushleft}
        \begin{itemize}
            \item navigare nell'apposita sezione;
            \item selezionare la stanza;
            \item selezionare la postazione;
            \item confermare l'igienizzazione della postazione;
            \item verificare l'avvenuta marcatura della postazione come igienizzata.
        \end{itemize}&
        NI\\

        TS16.1&
        \begin{flushleft}
            Il dipendente deve essere in grado di marcare la postazione dove ha utilizzato il kit pulizia come igienizzata tramite tag RFID.
            Al dipendente viene chiesto di:
        \end{flushleft}
        \begin{itemize}
            \item accedere alla pagina dedicata;
            \item selezionare la funzionalità di scannerizzazione;
            \item appoggiare il telefono al tag RFID;
            \item verificare la corretta scannerizzazione del tag RFID;
            \item confermare l'igienizzazione della postazione;
            \item verificare l'avvenuta conferma dell'igienizzazione della postazione.
        \end{itemize}&
        NI\\

        TS17&
        \begin{flushleft}
            Il sistema deve marcare come sporca una postazione che \`{e} stata occupata.
            Quando viene occupata una postazione il sistema deve:
        \end{flushleft}
        \begin{itemize}
            \item marcare la postazione come "sporca";
            \item rendere la postazione prenotabile ma da igienizzare.
        \end{itemize}&
        NI\\

     
        
    \end{longtable}
}