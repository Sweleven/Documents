Per far emergere tutte le problematiche sorte fino ad ora e poter procedere 
ad una loro risoluzione efficiente, di seguito viene presentata la valutazione 
fatta dai membri del gruppo {\Gruppo} circa il lavoro svolto durante l’attività appena conclusa. \\
I problemi analizzati riguardano:

\begin{itemize}
	\item \textbf{Organizzazione}: problemi relativi all’organizzazione e alla comunicazione interna del gruppo;
	\item \textbf{Ruoli}: problemi relativi al corretto svolgimento di un ruolo;
	\item \textbf{Strumenti di lavoro}: problemi relativi all’impiego degli strumenti di lavoro scelti.
\end{itemize}

Non essendo presente una figura esterna che possa fornire una valutazione oggettiva, pertrattare con cognizione 
di causa ogni punto sopra descritto, fondamentale è l’autovalutazione di ciascun membro del gruppo. Nonostante possa 
sembrare un sistema poco efficace il suo impiego ha permesso al gruppo di migliorare progressivamente la qualità del lavoro. 
Questa sezione attualmente risulta incompleta, verrà aggiornata con l’avanzamento del la-voro riportando nuove problematiche, 
ogni qual volta esse dovessero verificarsi. Di seguito sono esposte le difficoltà relative all’organizzazione, ai ruoli e agli strumenti 
adoperati per lo svolgimento del lavoro. Per fornire valutazioni facilmente leggibili e consultabili esse sono organizzate mediante tabelle 
la cui struttura è stabilita in \textit{Norme di Progetto v1.0.0}. 

\subsection{Valutazioni sull'organizzazione}
\begin{table}[H]
    \caption{Tabella delle problematiche relative all'organizzazione}
    \rowcolors{2}{\evenRowColor}{\oddRowColor}
\renewcommand{\arraystretch}{1.5}
\begin{longtable}{  >{\centering}p{0.15\textwidth} >{}p{0.30\textwidth}
    >{\centering}p{0.1075\textwidth} >{}p{0.30\textwidth}}
    \rowcolor{\primaryColor}
    \textcolor{\secondaryColor}{
    \centering\textbf{Problema}}     & \textcolor{\secondaryColor}{\centering\textbf{Descrizione}}    & \textcolor{\secondaryColor}
    {\centering\textbf{Gravità}} & \textcolor{\secondaryColor}{\centering\textbf{Soluzione}}\\
   
    Incontro con il gruppo  
    & Non ci sono stati grossi problemi nell’organizzazione degli incontri. 
    & 1  
    & Si è deciso di utilizzare un
    calendario condiviso per scegliere il giorno in cui tutto il
    team potesse essere presente.{} \\
    Incontro con il proponente
    & Poiché il capitolato è svolto in un periodo di emergenza, 
    gli incontri sono stati fatti via zoom. 
    & 1
    & L’azienda proponente non ha avuto molte difficoltà a proporci 
    diversi orari per i nostri incontri affinché l’intero gruppo fosse presente.{} \\
    \end{longtable}
\end{table}


\subsection{Valutazione sui ruoli}
    \rowcolors{2}{\evenRowColor}{\oddRowColor}
\renewcommand{\arraystretch}{1.5}
\begin{longtable}{  >{\centering}p{0.15\textwidth} >{}p{0.30\textwidth}
    >{\centering}p{0.1075\textwidth} >{}p{0.30\textwidth}}
    \caption{Tabella delle problematiche relative ai ruoli}	\\
    \rowcolor{\primaryColor}
    \textcolor{\secondaryColor}{
    \centering\textbf{Problema}}     & \textcolor{\secondaryColor}{\centering\textbf{Descrizione}}    & \textcolor{\secondaryColor}
    {\centering\textbf{Gravità}} & \textcolor{\secondaryColor}{\centering\textbf{Soluzione}}\\
   
    Rivestire il ruolo di \textit{Responsabile}  
    &  A causa dell'inesperienza, chi ha lavorato come Responsabile ha avuto discrete difficoltà nella suddivisione bilanciata delle ore tra i membri provocando diverse ridistribuzioni delle ore.
    & 2  
    & Il gruppo ha deciso di dedicare del tempo per analizzare meglio la mole di lavoro e compiere così una distribuzione delle ore più accurata per evitare eventuali ritardi nelle consegne. {} \\
    Rivestire il ruolo di \textit{Analista}
    & Difficoltà nell'individuazione dei casi d'uso che ha causato una sottostima del carico di lavoro degli analisti.
    & 3
    & Aumentare il carico di lavoro e il numero degli analisti. \\
    \end{longtable}


\subsection{Valutazioni sugli strumenti di lavoro}
\begin{table}[H]
    \caption{Tabella delle problematiche relative agli strumenti di lavoro}
    \rowcolors{2}{\evenRowColor}{\oddRowColor}
\renewcommand{\arraystretch}{1.5}
\begin{longtable}{  >{\centering}p{0.15\textwidth} >{}p{0.30\textwidth}
    >{\centering}p{0.1075\textwidth} >{}p{0.30\textwidth}}
    \rowcolor{\primaryColor}
    \textcolor{\secondaryColor}{
    \centering\textbf{Problema}}     & \textcolor{\secondaryColor}{\centering\textbf{Descrizione}}    & \textcolor{\secondaryColor}
    {\centering\textbf{Gravità}} & \textcolor{\secondaryColor}{\centering\textbf{Soluzione}}\\
   
    \LaTeX{} 
    & Si sono riscontrati alcuni problemi durante la creazione e compilazione dei documenti 
    dovuto al poco utilizzo di questo strumento dal gruppo.
    & 1  
    & Per risolvere questo problema, i componenti più esperti del gruppo hanno 
    aiutato i colleghi meno esperti affinché il lavoro procedesse senza troppi ritardi. {} \\
    GitHub
    & Alcuni componenti del gruppo hanno avuto diverse difficoltà nell’utilizzo 
    dello strumento Git non avendo molta esperienza con esso creando in alcune occasioni conflitti tra file. 
    & 2
    & Il gruppo ha deciso di avere un unico ramo main che contiene la versione stabile del proprio 
    lavoro e branch dedicati su cui lavorare. {} \\
    \end{longtable}
\end{table}
