AL fine di garantire uno standard qualitativo adeguato, sono stati scelti alcuni processi tra quelli proposti nello standard ISO/IEC/IEEE 12207:1995,
successivamente sono stati semplificati o adattati secondo le esigenze.

\subsection{Processi di Sviluppo}
    Questo processo contiene tutte le attività tipiche dello sviluppo software.
    \subsubsection{Analisi dei requisiti}
        Attraverso le attivit\`{a} di analisi, le quali comprendono lo studio del documento di presentazione del capitolato, gli incontri svolti con il proponente e gli incontri interni, vengono individuati i requisiti e, quindi, i casi d'uso.   

        \paragraph{Obiettivi}
            \begin{itemize}
                \item individuare i casi d'uso e i requisiti;
                \item tracciare il loro cambiamento nel tempo.
            \end{itemize}
        \paragraph{Strategia}
            \begin{itemize}
                \item considerare lo scopo del progetto e le richieste degli stakeholder;
                \item esprimere ciò in forma di requisiti, classificati in obbligatori, desiderabili e opzionali;
                \item valutare il corpo dei requisiti e negoziare cambiamenti se necessario;
                \item disporre il tracciamento dei requisiti del sistema.
            \end{itemize}
        \paragraph{Metriche}
            \begin{enumerate}
                \item \textbf{PROS} (Percentuale di requisiti obbligatori soddisfatti):
                \begin{itemize}
                    \item valore preferibile: $100\%$;
                    \item valore accettabile: $100\%$.
                \end{itemize}
            \end{enumerate}
    \subsubsection{Progettazione di dettaglio}
        La progettazione di dettaglio prevede la scomposizione in componenti pi\`{u} piccole delle macro-componenti di cui il modello architetturale è composto, le quali devono essere:
        \begin{itemize}
            \item facilmente comprensibili;
            \item strettamente correlate ai requisiti funzionali;
            \item implementabili da un singolo programmatore.
        \end{itemize}
        \paragraph{Obiettivi}
            \begin{itemize}
                \item tradurre i requisiti in moduli;
                \item favorire il lavoro dei programmatori assegnando compiti individuali associati a singoli moduli;
                \item mantenere il tracciamento tra requisiti e componenti.
            \end{itemize}
        \paragraph{Strategia}
            \begin{itemize}
                \item scomporre le componenti architetturali in componenti piccole;
                \item rispettare le best practices durante la codifica;
                \item tenere sotto controllo i valori delle metriche e nel caso effettuare le correzioni necessarie.
            \end{itemize}
        \paragraph{Metriche}
            \begin{enumerate}
                \item \textbf{CBO} (Accoppiamento tra le classi di oggetti):
                \begin{itemize}
                    \item valore preferibile: $0 \leq CBO\leq 1$;
                    \item valore accettabile: $0 \leq CBO\leq 6$.
                \end{itemize}
            \end{enumerate}
    \subsubsection{Codifica}
        La codifica prevede la scrittura del codice sorgente basandosi sull'output della progettazione di dettaglio.
        \paragraph{Obiettivi}
            \begin{itemize}
                \item assicurare la qualità del codice sorgente.
            \end{itemize}
        \paragraph{Strategia}
            \begin{itemize}
                \item eseguire l'analisi statica del codice per monitorare le metriche;
                \item rispettare le norme di codifica.
            \end{itemize}
        \paragraph{Metriche}
            \begin{enumerate}
                \item numero di bug:
                \begin{itemize}
                    \item valore preferibile: $0$;
                    \item valore accettabile: $0$.
                \end{itemize}
                \item numero di vulnerabilità:
                \begin{itemize}
                    \item valore preferibile: $0$;
                    \item valore accettabile: $0$.
                \end{itemize}
                \item percentuale di copertura del codice dai test:
                \begin{itemize}
                    \item valore preferibile: $100\%$;
                    \item valore accettabile: $\geq 80\%$.
                \end{itemize}
                \item percentuale di copertura dei rami dai test:
                \begin{itemize}
                    \item valore preferibile: $100\%$;
                    \item valore accettabile: $\geq 80\%$.
                \end{itemize}
                \item percentuale di codice ripetuto:
                \begin{itemize}
                    \item valore preferibile: $0\%$;
                    \item valore accettabile: $\leq 1\%$.
                \end{itemize}
            \end{enumerate}
\subsection{Processi di Supporto}
    \subsubsection{Pianificazione}
        La pianificazione è un attività rilevante della gestione del progetto. Essa permette di gestire le risorse (tempo, costo, ruoli) monitorandole e ripartendole in base all'andamento del progetto.
        Questa attività è descritta nel Piano di Progetto.
        \paragraph{Obiettivi}
            \begin{itemize}
                \item stabilire i piani e gli obiettivi del progetto;
                \item stabilire i ruoli all'interno del progetto;
                \item stabilire l'allocazione del tempo e dei costi.
            \end{itemize}
        \paragraph{Strategia}
            \begin{itemize}
                \item pianificare le attività;
                \item mantenere aggiornata la pianificazione;
                \item utilizzare la pianificazione prodotta come riferimento per l'andamento del progetto.
            \end{itemize}
        \paragraph{Metriche}
            \begin{enumerate}
                \item \textbf{BAC} (Budget at Completion).
                \item \textbf{EAC} (Estimate At Completion):
                \begin{itemize}
                    \item  misurazione: numero intero;
				    \item  valore preferibile: $ EAC \leq preventivo$;
				    \item  valore accettabile: $ preventivo -5\% \leq EAC \leq preventivo + 5\%$.
                \end{itemize}
                \item \textbf{VAC} (Variance At Completion):
                \begin{itemize}
                    \item  misurazione: percentuale: $\frac{preventivo - EAC}{100}$;
                    \item  valore preferibile: $\geq 0$;
                    \item  valore accettabile: $\geq 0$.
                \end{itemize}
                \item \textbf{AC} (Actual Cost):
                \begin{itemize}
                    \item valore preferibile: $0 \leq AC < PV$;
                    \item valore accettabile: $0 \leq AC \leq budget\ totale$.
                \end{itemize}
                \item \textbf{EV} (Earned Value):
                \begin{itemize}
                    \item valore preferibile: $\geq 0$;
                    \item valore accettabile: $\geq 0$.
                \end{itemize}
                \item \textbf{PV} (Planned Value):
                \begin{itemize}
                    \item valore preferibile: $\geq 0$;
                    \item valore accettabile: $\geq 0$.
                \end{itemize}
                \item \textbf{SV} (Schedule Variance):
                \begin{itemize}
                    \item valore preferibile: $\geq 0$;
                    \item valore accettabile: $0$.
                \end{itemize}
                \item \textbf{CV} (Cost Variance):
                \begin{itemize}
                    \item valore preferibile: $0\%$;
                    \item valore accettabile: $0\% \leq CV \leq 5\%$.
                \end{itemize}
            \end{enumerate}
    \subsubsection{Verifica}
        Questo processo consiste nella ricerca e correzione di errori nei processi e prodotti del progetto.
        \paragraph{Obiettivi}
            \begin{itemize}
                \item individiare e correggere gli errori;
                \item assicurarsi che il prodotto soddisfi i requisiti.
            \end{itemize}
        \paragraph{Strategia}
            \begin{itemize}
                \item stabilire le tecniche e gli strumenti di verifica;
                \item utilizzare questi strumenti e tecniche per individuare gli errori.
            \end{itemize}
        \paragraph{Metriche}
            \begin{enumerate}
                \item \textbf{SC} (Statement Coverage):
                \begin{itemize}
                    \item valore preferibile: $100\%$;
                    \item valore accettabile: $\geq 80\%$.
                \end{itemize}
                \item \textbf{BC} (Branch Coverage):
                \begin{itemize}
                    \item valore preferibile: $100\%$;
                    \item valore accettabile: $\geq 80\%$.
                \end{itemize}
                \item \textbf{MCDC} (Modified Condition/Decision Coverage):
                \begin{itemize}
                    \item valore preferibile: $100\%$;
                    \item valore accettabile: $\geq 80\%$.
                \end{itemize}
                \item \textbf{DSCC} (Documentation service code coverage):
                \begin{itemize}
                    \item valore preferibile: $100\%$;
                    \item valore accettabile: $\geq 80\%$.
                \end{itemize}
            \end{enumerate}
    \subsubsection{Documentazione}
        Questo processo consiste nella stesura e rilascio di documenti a supporto di tutte le attività di progetto.
        \paragraph{Obiettivi}
            I documenti prodotti devono essere:
            \begin{itemize}
                \item specifici;
                \item completi;
                \item non ambigui;
                \item modulari;
                \item disponibili esternamente.
            \end{itemize}
        \paragraph{Strategia} i documenti devono essere:
            \begin{itemize}
                \item scritti in un linguaggio modulare, come \LaTeX;
                \item prodotti in modo collaborativo;
                \item ospitati in una repository pubblica;
                \item corredati di un glossario;
            \end{itemize}
        \paragraph{Metriche}
        \begin{enumerate}
            \item \textbf{IG} (Indice di Gulpease):
            \begin{itemize}
                \item valore preferibile: $80< IG < 100$;
                \item valore accettabile: $60< IG < 100$.
            \end{itemize}
            \item \textbf{Correttezza ortografica}
            \begin{itemize}
                \item valore preferibile: $100\%$;
                \item valore accettabile: $100\%$.
            \end{itemize}
        \end{enumerate}
\subsection{Processi Organizzativi}
    \subsubsection{Gestione della Qualità}
        \paragraph{Obiettivi}
            Garantire il raggiungimento da parte dei processi e dei prodotti degli standard di qualità richiesti.
        \paragraph{Strategia}
            \begin{itemize}
                \item pianificare le attività di gestione del sistema qualità;
                \item stabilire un sistema di indicatori per la valutazione dei processi e prodotti;
                \item monitorare costantemente i risultati prodotti dal punto sopra, eventualmente ripianificando il primo punto.
            \end{itemize}
        \paragraph{Metriche}
        \begin{enumerate}
        \item \textbf{PMS} (Percentuale di metriche soddisfatte):
        \begin{itemize}
            \item valore preferibile: $\geq 80\%$;
            \item valore accettabile: $\geq 60\%$.
        \end{itemize}
    \end{enumerate}