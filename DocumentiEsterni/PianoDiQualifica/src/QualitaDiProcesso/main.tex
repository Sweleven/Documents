Per ricercare qualità nello svolgimento del progetto si adoperano dei processi. Inizialmente tali processi sono stati scelti tra quelli proposti nello standard ISO/IEC/IEEE 12207:1995,
successivamente sono stati semplificati o adattati secondo le esigenze.
Il risultato sono i processi esposti a seguito.
\subsection{Processi di Sviluppo}
    Questo processo contiene tutte le attività tipiche dello sviluppo software, alcune di esse sono molto importanti per cui vengono approfondite: si tratta dell'analisi dei requisiti,
    la progettazione dell'architettura, la progettazione di dettaglio e la codifica.
    \subsubsection{Analisi dei requisiti}
        Durante l'analisi le informazioni ottenute dalle varie fonti sono trasformate in forma di casi d'uso e requisiti.
        Questa forma fornisce una descrizione dettagliata del sistema e definisce il funzionamento e le caratteristiche di ogni sua parte.
        \paragraph{Obiettivi}
            \begin{itemize}
                \item formulare la definizione di casi d'uso e requisiti;
                \item ottenere la loro approvazione;
                \item tracciare il loro cambiamento nel tempo.
            \end{itemize}
        \paragraph{Strategia}
            \begin{itemize}
                \item considerare lo scopo del progetto e le richieste degli stakeholder;
                \item esprimere ciò in forma di requisiti, classificati in obbligatori, desiderabili e opzionali;
                \item valutare il corpo dei requisiti e negoziare cambiamenti se necessario;
                \item ottenere la loro approvazione da parte del proponente;
                \item disporre del tracciamento dei requisiti del sistema.
            \end{itemize}
        \paragraph{Metriche}
            \begin{enumerate}
                \item \textbf{PROS} (Percentuale di requisiti obbligatori soddisfatti):
                \begin{itemize}
                    \item $PROS = \frac{requisiti\ obbligatori\ soddisfatti}{requisiti\ obbligatori\ totali}$ ;
                    \item valore preferibile: $100\%$;
                    \item valore accettabile: $100\%$.
                \end{itemize}
            \end{enumerate}
    \subsubsection{Progettazione dell'architettura}
        Progettare l'architettura significa tradurre i requisiti in un modello architetturale del sistema.
        Il modello così ottenuto è ad alto livello di dettaglio: esso dà una visione del sistema come composizione di parti, utili a capire il funzionamento del sistema, ma a grossa granularità quindi non ancora realizzabili nella pratica.\newline
        Nell'attività vanno considerati in particolare il tipo di software da produrre, le caratteristiche desiderate e i suoi requisiti non funzionali per scegliere l'architettura più adatta.
        \paragraph{Obiettivi}
            \begin{itemize}
                \item agevolare la realizzazione fornendo un'architettura adeguata;
                \item comprendere meglio il sistema scomponendolo in parti;
                \item progettare adoperando stili architetturali e design pattern.
            \end{itemize}
        \paragraph{Strategia}
            \begin{itemize}
                \item valutare il tipo di software da produrre, le caratteristiche desiderate, i casi d'uso e i requisiti da soddisfare;
                \item valutare i modelli architetturali secondo il punto precedente e scegliere il più adeguato;
                \item individuare nel modello i macro-componenti del sistema e le loro relazioni.
            \end{itemize}
        \paragraph{Metriche}
            \begin{enumerate}
                \item \textbf{SFIN} (Structural Fan-In):
                \begin{itemize}
                    \item valore preferibile: $\geq 1$;
                    \item valore accettabile: $\geq 0$.
                \end{itemize}
                \item \textbf{SFOUT} (Structural Fan-Out):
                \begin{itemize}
                    \item valore preferibile: $=0$;
                    \item valore accettabile: $\leq 6$.
                \end{itemize}
            \end{enumerate}
    \subsubsection{Progettazione di dettaglio}
        La progettazione di dettaglio prevede la scomposizione delle macro-componenti di cui il modello architetturale è fatto in componenti più piccole, che siano:
        \begin{itemize}
            \item facilmente comprensibili;
            \item strettamente collegate ai requisiti funzionali;
            \item implementabili da un singolo programmatore.
        \end{itemize}
        \paragraph{Obiettivi}
            \begin{itemize}
                \item tradurre i requisiti in moduli;
                \item favorire il lavoro dei programmatori assegnando compiti individuali, e relativi a singoli moduli;
                \item mantenere il tracciamento tra requisiti e componenti.
            \end{itemize}
        \paragraph{Strategia}
            \begin{itemize}
                \item scomporre le componenti architetturali in componenti piccole;
                \item rispettare le best practices durante la codifica;
                \item tenere sotto controllo i valori delle metriche e nel caso effettuare le correzioni necessarie.
            \end{itemize}
        \paragraph{Metriche}
            \begin{enumerate}
                \item \textbf{CBO} (Accoppiamento tra le classi di oggetti):
                \begin{itemize}
                    \item valore preferibile: $0 \leq CBO\leq 1$;
                    \item valore accettabile: $0 \leq CBO\leq 6$.
                \end{itemize}
            \end{enumerate}
    \subsubsection{Codifica}
        La codifica prevede la scrittura del codice sorgente basandosi sull'output della progettazione di dettaglio
        \paragraph{Obiettivi}
            \begin{itemize}
                \item assicurare la qualità del codice sorgente.
            \end{itemize}
        \paragraph{Strategia}
            \begin{itemize}
                \item eseguire l'analisi statica del codice per monitorare le metriche;
                \item rispettare le norme di codifica.
            \end{itemize}
        \paragraph{Metriche}
            \begin{enumerate}
                \item numero di bug:
                \begin{itemize}
                    \item valore preferibile: $0$;
                    \item valore accettabile: $0$.
                \end{itemize}
                \item numero di vulnerabilità:
                \begin{itemize}
                    \item valore preferibile: $0$;
                    \item valore accettabile: $0$.
                \end{itemize}
                \item percentuale di copertura del codice dai test:
                \begin{itemize}
                    \item valore preferibile: $100\%$;
                    \item valore accettabile: $\geq 80\%$.
                \end{itemize}
                \item percentuale di copertura dei rami dai test:
                \begin{itemize}
                    \item valore preferibile: $100\%$;
                    \item valore accettabile: $\geq 80\%$.
                \end{itemize}
                \item percentuale di codice ripetuto:
                \begin{itemize}
                    \item valore preferibile: $0\%$;
                    \item valore accettabile: $\leq 1\%$.
                \end{itemize}
            \end{enumerate}
\subsection{Processi di Supporto}
    \subsubsection{Pianificazione}
        La pianificazione è un attività rilevante della gestione del progetto. Essa permette di gestire le risorse (tempo, costo, ruoli) monitorandole e ripartendole in base all'andamento del progetto.
        Questa attività è descritta nel Piano di Progetto.
        \paragraph{Obiettivi}
            \begin{itemize}
                \item stabilire i piani e gli obiettivi del progetto;
                \item stabilire i ruoli all'interno del progetto;
                \item stabilire l'allocazione del tempo e dei costi.
            \end{itemize}
        \paragraph{Strategia}
            \begin{itemize}
                \item pianificare le attività;
                \item mantenere aggiornata la pianificazione;
                \item utilizzare la pianificazione prodotta come riferimento per l'andamento del progetto.
            \end{itemize}
        \paragraph{Metriche}
            \begin{enumerate}
                \item \textbf{BAC} (Budget at Completion):
                \begin{itemize}
                    \item valore preferibile: uguale al preventivo;
                    \item valore accettabile: preventivo $\pm 5\%$.
                \end{itemize}
                \item \textbf{AC} (Actual Cost):
                \begin{itemize}
                    \item valore preferibile: $0 \leq AC < PV$;
                    \item valore accettabile: $0 \leq AC \leq budget\ totale$.
                \end{itemize}
                \item \textbf{EV} (Earned Value):
                \begin{itemize}
                    \item valore preferibile: $\geq 0$;
                    \item valore accettabile: $\geq 0$.
                \end{itemize}
                \item \textbf{PV} (Planned Value):
                \begin{itemize}
                    \item valore preferibile: $\geq 0$;
                    \item valore accettabile: $\geq 0$.
                \end{itemize}
                \item \textbf{SV} (Schedule Variance):
                \begin{itemize}
                    \item valore preferibile: $\geq 0$;
                    \item valore accettabile: $0$.
                \end{itemize}
                \item \textbf{CV} (Cost Variance):
                \begin{itemize}
                    \item valore preferibile: $< 0$;
                    \item valore accettabile: $\geq 0$.
                \end{itemize}
            \end{enumerate}
    \subsubsection{Verifica}
        Questo processo consiste nella ricerca e correzione di errori nei processi e prodotti del progetto.
        \paragraph{Obiettivi}
            \begin{itemize}
                \item individiare e correggere gli errori;
                \item assicurarsi che il prodotto soddisfi i requisiti.
            \end{itemize}
        \paragraph{Strategia}
            \begin{itemize}
                \item stabilire le tecniche e gli strumenti di verifica;
                \item utilizzare questi strumenti e tecniche per cercare gli errori.
            \end{itemize}
        \paragraph{Metriche}
            \begin{enumerate}
                \item \textbf{SC} (Statement Coverage):
                \begin{itemize}
                    \item $SC = \frac{numero\ di\ righe\ eseguite}{numero\ di\ righe\ totali}$;
                    \item valore preferibile: $100\%$;
                    \item valore accettabile: $\geq 80\%$.
                \end{itemize}
                \item \textbf{BC} (Branche Coverage):
                \begin{itemize}
                    \item $SC = \frac{numero\ di\ rami\ percorsi}{numero\ di\ rami\ totali}$;
                    \item valore preferibile: $100\%$;
                    \item valore accettabile: $\geq 80\%$.
                \end{itemize}
                \item \textbf{MCDC} (Modified Condition/Decision Coverage):
                \begin{itemize}
                    \item $SC = \frac{numero\ di\ decisioni\ percorse}{numero\ di\ decisioni\ totali}$;
                    \item valore preferibile: $100\%$;
                    \item valore accettabile: $\geq 80\%$.
                \end{itemize}
            \end{enumerate}
    \subsubsection{Documentazione}
        Questo processo consiste nella stesura e rilascio di documenti a supporto di tutte le attività di progetto.
        \paragraph{Obiettivi}
            I documenti prodotti devono essere:
            \begin{itemize}
                \item specifici;
                \item completi;
                \item non ambigui;
                \item modulari;
                \item disponibili esternamente.
            \end{itemize}
        \paragraph{Strategia} i documenti devono essere:
            \begin{itemize}
                \item scritti in un linguaggio modulare, come \LaTeX;
                \item prododotti in modo collaborativo;
                \item ospitati in una repository pubblica, anche per favorire il punto precedente;
                \item sostenuti da un glossario;
                \item sostenuti dalle Norme di Progetto.
            \end{itemize}
        \paragraph{Metriche}
        \begin{enumerate}
            \item \textbf{IG} (Indice di Gulpease):
            \begin{itemize}
                \item $IG = 89+\frac{300*numero\ di\ frasi-10*numero\ di\ lettere}{numero\ di\ parole}$;
                \item valore preferibile: $80< IG < 100$;
                \item valore accettabile: $60< IG < 100$.
            \end{itemize}
            \item \textbf{Correttezza ortografica}
            \begin{itemize}
                \item valore preferibile: $0$;
                \item valore accettabile: $0$.
            \end{itemize}
        \end{enumerate}
\subsection{Processi Organizzativi}
    \subsubsection{Gestione della Qualità}
        \paragraph{Obiettivi}
            Garantire il raggiungimento da parte dei processi e dei prodotti degli standard di qualità richiesti.
        \paragraph{Strategia}
            \begin{itemize}
                \item pianificare le attività di gestione del sistema qualità;
                \item stabilire un sistema di indicatori per la valutazione dei processi e prododotti;
                \item monitorare costantemente i risultati prodotti dal punto sopra, eventualmente ripianificando il primo punto.
            \end{itemize}
        \paragraph{Metriche}
        \begin{enumerate}
        \item \textbf{PMS} (Percentuale di metriche soddisfatte):
        \begin{itemize}
            \item $PMS = \frac{numero\ di\ metriche\ soddisfatte}{numero\ di\ metriche\ totali}$;
            \item valore preferibile: $\geq 80\%$;
            \item valore accettabile: $\geq 60\%$.
        \end{itemize}
    \end{enumerate}