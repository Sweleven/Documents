\paragraph{Comprensibilità delle funzioni offerte}
indica la percentuale di operazioni comprese in modo immediato dall’utente senza la consultazione del manuale.
	\begin{itemize}
	    \item misurazione: si calcola con la seguente formula: \\
		\centerline{ CFO =  \(\frac{N\textsubscript{FC}}{N\textsubscript{FO}} \)$ \cdot 100$ }
		dove N\textsubscript{FC} indica il numero di funzionalità comprese in modo immediato dall’utente durante l'attivita di testing del 
        prodotto e N\textsubscript{FO} indica il numero di funzionalità offerte dal sistema;
		\item valore preferibile: 100\%;
		\item valore accettabile: $\geq$ 80\%.
	\end{itemize}

    \paragraph{Facilità di apprendimento delle funzionalità}
    indica il tempo medio impiegato dall’utente per imparare ad usare correttamente una data funzionalità. 
    Si misura tramite un indicatore numerico che indica i minuti impiegati da un utente per apprendere il funzionamento di una certa funzionalità.
	\begin{itemize}
		\item valore preferibile: $\leq$ 10 minuti;
		\item valore accettabile: $\leq$ 20 minuti.
	\end{itemize}

    \paragraph{Tolleranza agli errori}
    indica il livello di tolleranza del prodotto relativamente agli errori commessi dagli 
    utenti. Si tratta di condizioni di errore frequente che il software può rilevare, correggere automaticamente e segnalare con un opportuno messaggio.
	\begin{itemize}
	    \item misurazione: si calcola con la seguente formula: \\
		\centerline{ TE =  \(\frac{N\textsubscript{MC}}{N\textsubscript{MT}} \)$ \cdot 100$ }
		dove N\textsubscript{MC} indica il numero di messaggi che risultano chiari, completi e corretti 
        e N\textsubscript{MT} indica il numero totale di messaggi previsti;
		\item valore preferibile: 100\%;
		\item valore accettabile: $\geq$ 80\%.
	\end{itemize}

