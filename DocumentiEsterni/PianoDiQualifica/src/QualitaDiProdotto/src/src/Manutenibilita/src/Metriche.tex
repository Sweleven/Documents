\paragraph{Semplicità delle funzioni}
La facilità di un metodo può essere rappresentata dal numero di parametri per metodo: meno parametri ha una funzione più è semplice e intuitiva.
	\begin{itemize}
		\item misurazione: numero di parametri per metodo;
		\item valore preferibile $\leq$ 3;
		\item valore accettabile $\leq$ 6.
	\end{itemize}

\paragraph{Facilità di comprensione}
La facilità con cui è possibile comprendere cosa fa il codice può rappresentata dal numero di linee di commento nel codice.
	\begin{itemize}
	    \item misurazione: Si può calcolare con la seguente formula: \\
		\centerline{R = \(\frac{N\textsubscript{LCOM}}{N\textsubscript{LCOD}} \) }
			dove N\textsubscript{LCOM} indica le linee di commento e N\textsubscript{LCOD} indica le linee di codice;
		\item valore preferibile: $\geq$ 0.20;
		\item valore accettabile: $\geq$ 0.10.
	\end{itemize}

\paragraph{Semplicità delle classi}
La facilità di una classe può essere rappresentata dal numero di metodi per classe: una classe con pochi metodi ha uno scopo ben preciso e facilmente comprensibile.
	\begin{itemize}
		\item misurazione: numero di metodi per classe;
		\item valore preferibile $\leq$ 8;
		\item valore accettabile $\leq$ 15.
	\end{itemize}

\paragraph{Structural Fan-In}
Numero di componenti che utilizzano un dato modulo. Un alto valore indica un alto riuso della
componente.
    \begin{itemize}
         \item valore preferibile: $\geq 1$;
        \item valore accettabile: $\geq 0$.
     \end{itemize}
\paragraph{Structural Fan-Out}
Numero di componenti che vengono utilizzate dalla componente in esame. Un alto valore indica
un alto accoppiamento della componente.
    \begin{itemize}
        \item valore preferibile: $=0$;
        \item valore accettabile: $\leq 6$.
	\end{itemize}
    