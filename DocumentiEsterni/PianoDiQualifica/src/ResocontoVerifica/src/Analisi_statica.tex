L’analisi dei documenti mediante \textit{Walkthrough}\glo{} ha portato all’individuazione di alcuni errori frequenti. 
Una volta redatto il documento, il \textit{Verificatore} ne ha valutato poi la correttezza, nella sua interezza, cercando di individuarvi eventuali errori presenti. 
Se trovati la metodologia adottata è stata la seguente:
\begin{itemize}
    \item correzione degli errori ortografici e sintattici non conformi alle norme tipografiche stabilite nelle \textit{Norme di progetto v1.0.0};
    \item inserimento degli errori più ricorrenti nella \textit{lista di controllo}\glo{}, redatta durante la fase di verifica dei documenti;
    \item applicazione del ciclo \textit{PDCA}\glo{}, per migliorare e velocizzare le verifiche future. 
\end{itemize} 

A questo punto si è impiegata la tecnica dell'\textit{Inspection}\glo{}. 
Grazie alla \textit{lista di controllo}\glo{} si è infatti potuto svolgere un ulteriore esame nei confronti del documento sottoposto a verifica, per scoprire quegli errori che seppur presenti, non fossero stati ancora individuati dalle attività precedenti.
