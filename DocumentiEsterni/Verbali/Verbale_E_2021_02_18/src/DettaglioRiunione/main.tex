Il gruppo Sweleven ha richiesto questo incontro per rispondere alle domande sorte durante la decisione delle tecnologie da utilizzare, e durante la rivisitazione dei casi d'uso.

Di seguito vengono riportati i principali punti di discussione con il Dott. Patera:
\begin{enumerate}
    \item \textbf{uso del solo tag NFC per tracciare il dipendente:} il problema riscontrato è stato quello di stabilire quado un dipendente avesse definitivamente lasciato la postazione e la stanza, senza averla igienizzata e prima della scadenza della sua prenotazione. La prima ipotesi è stata quella di tracciare gli spostamenti del dipendente tramite la tecnologia GPS, cosa che però richiederebbe un uso eccessivo della batteria. Si è quindi deciso di affidarci solamente al NFC, rendendo l'uso del GPS un requisito desiderabile, permettendo al dipendente di lasciare la postazione senza igienizzarla, e avvisando il dipendente dopo dello stato sporco della postazione quando esso scannerizzerà il tag;
    \item \textbf{utilizzo di una postazione senza prenotazione:} un'altra situazione verosimile riscontrata è quando un dipendente voglia usare una postazione ma esso non abbia effettuato la prenotazione. È quindi stato stabilito che un utente posssa usare una postazione non prenotata entro un breve periodo, creando una prenotazione "istantanea" quando si scannerizza il tag associato.;
    \item \textbf{prenotazione di più postazioni in contemporanea:} il consiglio del Dott. Patera è stato quello di rendere questa opzione un requisito desiderabile, in quanto spesso accade che un dipendente debba utilizzare una postazione con degli ospiti esterni all'azienda e loro, non essendo dipendenti, non avrebbero modo di prenotarsi una postazione. È stato anche suggerito di inserire la possibilità di prenotare una stanza intera, in occasione magari di conferenze;
    \item \textbf{cancellazione di una prenotazione se non ci si presenta:} il chiarimento di questo punto ha portato a stabilire un nuovo requisito desiderabile, che indica la cancellazione di una prenotazione se il dioendente che l'ha prenotata non si presenta entro 2 ore dall'inizio dell'ora indicata.
 \end{enumerate}