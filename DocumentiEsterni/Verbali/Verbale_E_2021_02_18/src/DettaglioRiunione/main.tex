Il gruppo Sweleven ha richiesto questo incontro per rispondere alle domande sorte durante la decisione sulle tecnologie da utilizzare e la rivisitazione dei casi d'uso.

Di seguito vengono riportati i principali punti di discussione con il Dott. Patera:
\begin{enumerate}
    \item \textbf{uso del solo tag NFC per tracciare il dipendente:} il problema riscontrato è stato quello di stabilire quando un dipendente avesse definitivamente lasciato la postazione e la stanza, senza averla igienizzata e prima della scadenza della sua prenotazione. La prima ipotesi è stata quella di tracciare gli spostamenti del dipendente tramite la tecnologia GPS, cosa che però richiederebbe un uso eccessivo della batteria del dispositivo dell'utente. Si è quindi deciso di affidarci solamente alla tecnologia \glo{NFC}, rendendo quindi l'utilizzo del \glo{GPS} un requisito desiderabile e permettendo al dipendente di lasciare la postazione senza igienizzarla. Un successivo utilizzatore della postazione non igienizzata sarà quindi avvisato dello stato della postazione al momento della scannerizzazione del tag;
    \item \textbf{utilizzo di una postazione senza prenotazione:} un'altra situazione verosimile riscontrata è quando un dipendente voglia usare una postazione ma esso non abbia effettuato la prenotazione. È quindi stato stabilito che un utente possa usare una postazione non prenotata entro un breve periodo, creando una prenotazione "istantanea" non appena si scannerizzerà il relativo tag associato alla postazione scelta;
    \item \textbf{prenotazione di più postazioni in contemporanea:} sotto consiglio del Dott. Patera, abbiamo reso l'opzione in oggetto come requisito desiderabile, in quanto spesso accade che un dipendente debba utilizzare una postazione con degli ospiti esterni all'azienda ed essi, non essendo dipendenti, non avrebbero modo di prenotarsi una postazione. Ci è stato, inoltre, suggerito di inserire la possibilità di prenotare una stanza intera, in occasione di eventi ad alta numerosità di persone, quali ad esempio conferenze;
    \item \textbf{cancellazione di una prenotazione se l'utente non si presenta:} il chiarimento di questo punto ha portato a stabilire un nuovo requisito desiderabile, che indica la cancellazione di una prenotazione se il dipendente che l'ha prenotata non si presenta entro 2 ore dall'inizio dell'ora indicata.
 \end{enumerate}