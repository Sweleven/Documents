
\begin{itemize}
    \item utilizzo delle librerie \glo{Node.js} e \glo{Web3.js};
    \item valutazione della tecnologia da utilizzare per la gestione dei nodi:
        \begin{itemize}
            \item \glo{Geth}: client pesante e da installare in locale, è stato ritenuto eccessivamente costoso;
            \item \glo{Parity}: presenta funzionalità limitate e non fornisce certezza di validazione, richiede inoltre prestazioni discretamente alte;
            \item \glo{Infura}: in quanto servizio via \glo{cloud} più leggero e di facile utilizzo, è stato ritenuto il più appropriato da adottare.
        \end{itemize}
    \item iniziare a valutare e prendere confidenza con strumenti quali \glo{Docker}, \glo{Docker Compose} e \glo{Minikube};
    \item creare un gruppo \glo{Telegram} con i membri del gruppo ed il Dott. Patera (in seguito all'aperta disponibilità di quest'ultimo), per agevolare eventuali comunicazioni veloci.
\end{itemize}

Non è stata fissata una data precisa per il prossimo incontro. Si prevede in ogni caso che questo verrà probabilmente svolto nell'arco della prossima settimana rispetto alla data corrente.
