\documentclass[a4paper, oneside, openany, dvipsnames, table]{article}
\usepackage[utf8]{inputenc}
\usepackage{../../../Shared/Sweleven}
\newcommand{\Titolo}{Piano di Qualifica}

\newcommand{\Approvatore}{TODO}
\newcommand{\Redattori}{Andrea De Tomasi \newline Abdelwahad Kandoul}
\newcommand{\Verificatori}{Alessio Trevisan \newline Filippo Pinton}

\newcommand{\pathimg}{../../Shared/logo.png}

\newcommand{\Versionedoc}{X.X.X}

\newcommand{\Distribuzione}{Prof. Tullio Vardanega \newline Prof. Riccardo Cardin \newline Imola Informatica S.p.A \newline Sweleven}

\newcommand{\Uso}{Esterno}

\newcommand{\DescrizioneDoc}{TODO}



% Ogni volta che si crea un documento partendo dal template nuovo, bisogn apportare le seguenti modifiche: 
%  Riga 13 comandi.tex per indicare i destinatari del documento
%  Riga 15 comandi.tex per indicare se il documento è interno o esterno 
%  Riga 21 comandi.tex per la descrizione del documento
%  Riga 11 comandi.tex per specificare la versione del documento.
%  Riga 11 comandi.tex per specificare la versione del documento.
%  Riga 3 introduzione.tex per lo scopo del documento
%  Riga 10 introduzione.tex riferimenti del documento.
% quindi cominciare la stesura del documento dei file presenti in sezioni/


\begin{document}
    \copertina{}

    {
    \rowcolors{2}{\evenRowColor}{\oddRowColor}
    \renewcommand{\arraystretch}{1.5}
    \centering
    \begin{longtable}{ c c  C{3.5cm}  C{3.5cm}  c }
        \rowcolor{\primaryColor}
        \textcolor{\secondaryColor}{
        \textbf{Versione}}     & \textcolor{\secondaryColor}{\textbf{Data}}       & \textcolor{\secondaryColor}
        {\textbf{Descrizione}} & \textcolor{\secondaryColor}{\textbf{Nominativo}} & \textcolor{\secondaryColor}{\textbf{Ruolo}}                          \\

        % usare \verificatore per indicare verificatore
        % usare \redattore per indicare redatore
        % usare \responsabile per indicare responsabile

        %		L'ultimo evento deve essere sempre all'inizio della tabella
        1.0.0  & 
        2021-01-10  & 
        Approvazione & 
        Giovanni Grigoletto & 
        \responsabile{} \\

        0.1.1  & 
        2021-01-09 & 
        Aggiunti dettagli finali e correzioni & 
        Elvis Murtezan, Giovanni Grigoletto  & 
        \redattore{} \\

        0.1.0  & 
        2021-01-05 & 
        Verifica & 
        Abdelwahad Kandoul, Edoradro Caregato  & 
        \verificatore{} \\

        0.0.8  & 
        2021-01-03 & 
        Aggiunti grafici & 
        Elvis Murtezan  & 
        \redattore{} \\

        0.0.7  & 
        2021-01-03 & 
        Aggiornato preventivo architettura e preventivo al dettaglio & 
        Elvis Murtezan  & 
        \redattore{} \\

        0.0.6  & 
        2021-01-02 & 
        Stesura preventivo analisi e consuntivo & 
        Elvis Murtezan  & 
        \redattore{} \\

        0.0.5  & 
        2021-01-02 & 
        Aggiunta pianificazione & 
        Giovanni Grigoletto & 
        \redattore{} \\

        0.0.4  & 
        2020-12-28 & 
        Aggiunto organigramma e preventivo parziale & 
        Elvis Murtezan  & 
        \redattore{} \\

        0.0.3  & 
        2020-12-28 & 
        Aggiunto modello di sviluppo & 
        Giovanni Grigoletto & 
        \redattore{} \\
        
        0.0.2  & 
        2020-12-26 & 
        Aggunta introduzione ed analisi dei rischi & 
        Giovanni Grigoletto & 
        \redattore{} \\

        0.0.1  & 
        2020-12-23 & 
        Stesura template iniziale & 
        Giovanni Grigoletto, Elvis Murtezan & 
        \redattore{} \\
    \end{longtable}
}
    
    \newpage
    \tableofcontents
    
    \newpage
    \section{Informazioni generali}\label{sec:informazioni-generali}
    \subsection{Fornitura}\label{sec:Fornitura}
\subsection{Fornitura}\label{sec:Fornitura}
\subsection{Fornitura}\label{sec:Fornitura}
\input{src/ProcessiPrimari/src/Fornitura/main.tex}

\subsection{Sviluppo}\label{sec:Sviluppo}
\input{src/ProcessiPrimari/src/Sviluppo/main.tex}

\subsection{Sviluppo}\label{sec:Sviluppo}
\subsection{Fornitura}\label{sec:Fornitura}
\input{src/ProcessiPrimari/src/Fornitura/main.tex}

\subsection{Sviluppo}\label{sec:Sviluppo}
\input{src/ProcessiPrimari/src/Sviluppo/main.tex}

\subsection{Sviluppo}\label{sec:Sviluppo}
\subsection{Fornitura}\label{sec:Fornitura}
\subsection{Fornitura}\label{sec:Fornitura}
\input{src/ProcessiPrimari/src/Fornitura/main.tex}

\subsection{Sviluppo}\label{sec:Sviluppo}
\input{src/ProcessiPrimari/src/Sviluppo/main.tex}

\subsection{Sviluppo}\label{sec:Sviluppo}
\subsection{Fornitura}\label{sec:Fornitura}
\input{src/ProcessiPrimari/src/Fornitura/main.tex}

\subsection{Sviluppo}\label{sec:Sviluppo}
\input{src/ProcessiPrimari/src/Sviluppo/main.tex}
    
    \newpage
    \section{Verbale della riunione}\label{sec:verbale-della-riunione}
    \subsection{Fornitura}\label{sec:Fornitura}
\subsection{Fornitura}\label{sec:Fornitura}
\subsection{Fornitura}\label{sec:Fornitura}
\input{src/ProcessiPrimari/src/Fornitura/main.tex}

\subsection{Sviluppo}\label{sec:Sviluppo}
\input{src/ProcessiPrimari/src/Sviluppo/main.tex}

\subsection{Sviluppo}\label{sec:Sviluppo}
\subsection{Fornitura}\label{sec:Fornitura}
\input{src/ProcessiPrimari/src/Fornitura/main.tex}

\subsection{Sviluppo}\label{sec:Sviluppo}
\input{src/ProcessiPrimari/src/Sviluppo/main.tex}

\subsection{Sviluppo}\label{sec:Sviluppo}
\subsection{Fornitura}\label{sec:Fornitura}
\subsection{Fornitura}\label{sec:Fornitura}
\input{src/ProcessiPrimari/src/Fornitura/main.tex}

\subsection{Sviluppo}\label{sec:Sviluppo}
\input{src/ProcessiPrimari/src/Sviluppo/main.tex}

\subsection{Sviluppo}\label{sec:Sviluppo}
\subsection{Fornitura}\label{sec:Fornitura}
\input{src/ProcessiPrimari/src/Fornitura/main.tex}

\subsection{Sviluppo}\label{sec:Sviluppo}
\input{src/ProcessiPrimari/src/Sviluppo/main.tex}
    
    \newpage
    \section{Tracciamento delle decisioni}\label{sec:tracciamento-delle-decisioni}
    \subsection{Fornitura}\label{sec:Fornitura}
\subsection{Fornitura}\label{sec:Fornitura}
\subsection{Fornitura}\label{sec:Fornitura}
\input{src/ProcessiPrimari/src/Fornitura/main.tex}

\subsection{Sviluppo}\label{sec:Sviluppo}
\input{src/ProcessiPrimari/src/Sviluppo/main.tex}

\subsection{Sviluppo}\label{sec:Sviluppo}
\subsection{Fornitura}\label{sec:Fornitura}
\input{src/ProcessiPrimari/src/Fornitura/main.tex}

\subsection{Sviluppo}\label{sec:Sviluppo}
\input{src/ProcessiPrimari/src/Sviluppo/main.tex}

\subsection{Sviluppo}\label{sec:Sviluppo}
\subsection{Fornitura}\label{sec:Fornitura}
\subsection{Fornitura}\label{sec:Fornitura}
\input{src/ProcessiPrimari/src/Fornitura/main.tex}

\subsection{Sviluppo}\label{sec:Sviluppo}
\input{src/ProcessiPrimari/src/Sviluppo/main.tex}

\subsection{Sviluppo}\label{sec:Sviluppo}
\subsection{Fornitura}\label{sec:Fornitura}
\input{src/ProcessiPrimari/src/Fornitura/main.tex}

\subsection{Sviluppo}\label{sec:Sviluppo}
\input{src/ProcessiPrimari/src/Sviluppo/main.tex}
\end{document}