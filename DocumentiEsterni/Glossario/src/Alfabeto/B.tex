\section{B}

\TermineGlossario{Backend as a service (BAAS)}

\DefinizioneGlossario{È un modello per fornire agli sviluppatori di applicazioni web o mobile un modo per collegare le loro applicazioni a un \glo{backend} cloud storage e API, esposte da applicazioni \glo{backend}, fornendo allo stesso tempo funzioni quali la gestione degli utenti, le notifiche push, e l'integrazione con servizi di rete sociale.}

\TermineGlossario{Backend (server)}
\DefinizioneGlossario{Sono delle interfacce che hanno come destinatario un programma. Una applicazione \glo{backend} è un programma con il quale l'utente interagisce indirettamente. In una struttura client/server il \glo{backend} è il server.}

\TermineGlossario{Baseline}
\DefinizioneGlossario{È un insieme di configuration item in un dato istante temporale (milestone), verificato, base per l'avanzamento del progetto.}

\TermineGlossario{Beacon (Bluetooth)}
\DefinizioneGlossario{Sono dei trasmettitori hardware (classe di dispositivi Bluetooth a bassa energia LE) che trasmettono il loro identificatore a dispositivi elettronici portatili vicini.}

\TermineGlossario{Behavior-Driven Development (BDD)}
\DefinizioneGlossario{È un processo di sviluppo software agile che incoraggia la collaborazione tra sviluppatori, QA (Quality Assurance) e partecipanti non tecnici o aziendali a un progetto software. Lo sviluppo basato sul comportamento combina le tecniche generali e i principi del TDD (Test-driven development) con idee di progettazione guidata dal dominio e analisi e progettazione orientate agli oggetti per fornire ai team di sviluppo e gestione del software strumenti condivisi e un processo condiviso per collaborare allo sviluppo del software.}

\TermineGlossario{Big Data}
\DefinizioneGlossario{Indica in maniera generica un'estesa raccolta di dati da richiedere tecnologie e metodi analitici specifici per l'estrazione di valore o conoscenza. Il termine è utilizzato in riferimento alla capacità di analizzare  e mettere in relazione un'enorme mole di dati eterogenei, strutturati e non, allo scopo di scoprire i legami tra fenomeni diversi e prevedere quelli futuri.}

\TermineGlossario{Blockchain}
\DefinizioneGlossario{È una struttura dati condivisa e "immutabile", definita come un registro digitale le cui voci sono raggruppate in "blocchi", concatenati in ordine cronologico, e la cui integrità è garantita dall'uso della crittografia.}

\TermineGlossario{Branch}
\DefinizioneGlossario{È un puntatore a un singolo commit e viene utilizzato in Git per l'implementazione di funzionalità tra loro isolate.}

\TermineGlossario{Bug}
\DefinizioneGlossario{È un guasto che porta al malfunzionamento del software, tipicamente dovuto ad un errore nella scrittura del codice sorgente di un programma software scritto da un programmatore.}
\clearpage