\section{O}

\TermineGlossario{OpenAPI}
\DefinizioneGlossario{È un formato di descrizione delle API per le API REST. Un file OpenAPI ti consente di descrivere l'intera API, tra cui: endpoint disponibili e operazioni su ciascun endpoint (GET / utenti, POST / utenti), parametri operativi ingresso e uscita per ogni operazione, metodi di autenticazione, informazioni di contatto, licenza, condizioni d'uso e altre informazioni.
Le specifiche API possono essere scritte in YAML o JSON.}

\TermineGlossario{Openshift}
\DefinizioneGlossario{Costruito a partire da Kubernetes ma sfrutta diversi altri progetti e tecnologie open source (come Source 2 Image, la tecnologia che permette di passare dal codice sorgente di un'applicazione all'applicazione containerizzata, o Jenkins, un popolarissimo server di CI/CD).}

\TermineGlossario{Operabilità}
\DefinizioneGlossario{Capacità di un prodotto software di offrire delle funzioni coerenti con le aspettative dell’utente.}

\TermineGlossario{Orange (Canvas)}
\DefinizioneGlossario{È un toolkit di visualizzazione dei dati open source, machine learning e data mining. È dotato di un front-end di programmazione visiva per l'analisi dei dati esplorativi e la visualizzazione interattiva dei dati. Orange è costituito da un'interfaccia canvas su cui l'utente posiziona i widget e crea un flusso di lavoro di analisi dei dati.}

\TermineGlossario{Ordinamento per data (crescente)}
\DefinizioneGlossario{Considerando una data maggiore di un'altra più recente, si intende che, dato un insieme di date (possibilmente con elementi ripetuti), una data meno recente precede una più recente.}

\TermineGlossario{Ordinamento per data (decrescente)}
\DefinizioneGlossario{Considerando una data maggiore di un'altra più recente si intende che, dato un insieme di date (possibilmente con elementi ripetuti), una data più recente precede una meno recente.}

\clearpage