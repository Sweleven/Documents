\section{U}
\TermineGlossario{UML}
\DefinizioneGlossario{Linguaggio di modellazione visivo comune, ricco sia nella semantica che nella sintassi, per l'architettura, la progettazione e l'implementazione di sistemi software complessi sia dal punto di vista strutturale che comportamentale. L'UML viene applicato anche in altri settore, oltre allo sviluppo di software, come il flusso di processo nella produzione.}

\TermineGlossario{Unicode Transformation Format, 8 bit (UTF-8)}
\DefinizioneGlossario{È una codifica che assegna a ogni carattere Unicode esistente una specifica sequenza di bit, che può essere letta anche come numero binario. Questo significa che UTF-8 assegna un numero binario fisso ad ogni lettera, numero e simbolo di un numero crescente di lingue.}

\TermineGlossario{Unified Modeling Language (UML)}
\DefinizioneGlossario{In ingegneria del software è un linguaggio di modellazione e di specifica basato sul paradigma orientato agli oggetti. Viene usato per descrivere soluzioni analitiche e progettuali in modo sintetico e comprensibile ad un vasto pubblico (standard industriale unificato).}

\TermineGlossario{Uscita}
\DefinizioneGlossario{Per uscita da un luogo di un'azienda si intende l'attività di spostamento fisico in cui un'utente passa da una posizione geografica interna ad un perimetro che delimita un luogo soggetto a tracciamento ad una non soggetta a tracciamento.}

\TermineGlossario{User Interface (UI)}
\DefinizioneGlossario{È un'interfaccia uomo-macchina, ovvero ciò che si frappone tra una macchina e un utente, consentendone l'interazione reciproca.}
\clearpage