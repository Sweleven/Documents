\section{C}
\TermineGlossario{Camel Case}
\DefinizioneGlossario{Formato di scrittura di identificatori (tipicamente di nomi di file o di variabili) in cui le parole che li compongono sono separate dalle loro lettere maiuscole iniziali.}

\TermineGlossario{Cascading Style Sheets (CSS3)}
\DefinizioneGlossario{È un linguaggio usato per definire la formattazione di documenti HTML, XHTML e XML ad esempio i siti web e relative pagine web; permette una programmazione più chiara e facile da utilizzare, sia per gli autori delle pagine stesse sia per gli utenti, garantendo anche il riutilizzo di codice e facilita la manutenzione. Le specifiche CSS3 sono costituite da sezioni separate dette "moduli" e hanno differenti stati di avanzamento e stabilità.}

\TermineGlossario{Ciclo di Deming}
\DefinizioneGlossario{Modello per il miglioramento continuo della qualità in un’ottica a lungo raggio, anche detto PDCA (Plan, Do, Check, Act).}

\TermineGlossario{Clean architecture}
\DefinizioneGlossario{È una filosofia di progettazione software che separa gli elementi di un progetto in livelli ad anello.}

\TermineGlossario{CLI}
\DefinizioneGlossario{Acronimo di Command Line Interface. Si riferisce a un'interfaccia utente che permette di interagire con il sistema operativo e con gli altri programmi mediante la digitazione di comandi testuali. Alcuni esempi di CLI sono cmd o Command Prompt su SO Windows, bash su SO basati su Linux.}


\TermineGlossario{Cloud}
\DefinizioneGlossario{È uno spazio di archiviazione dove può essere accessibile in qualsiasi momento ed in ogni luogo utilizzando semplicemente una qualunque connessione ad Internet.}

\TermineGlossario{Cloudwatch}
\DefinizioneGlossario{Fornisce dati e analisi concrete per monitorare le applicazioni, rispondere ai cambiamenti di prestazioni a livello di sistema, ottimizzare l'utilizzo delle risorse e ottenere una visualizzazione unificata dello stato di integrità operativa.}

\TermineGlossario{Cluster}
\DefinizioneGlossario{È un insieme di computer connessi tra loro tramite una rete telematica con lo scopo di distribuire un'elaborazione molto complessa tra i vari computer, aumentando la potenza di calcolo del sistema e/o garantendo una maggiore disponibilità di servizio.}

\TermineGlossario{Code generation}
\DefinizioneGlossario{Meccanismo dove il compilatore prende in input il codice sorgente e lo converte in codice macchina.}

\TermineGlossario{Code smell} 
\DefinizioneGlossario{Nell'ingegneria del software questa espressione viene usata per indicare una serie di caratteristiche che il codice sorgente può avere e che sono generalmente riconosciute 
come probabili indicazioni di un difetto di programmazione; in altre parole sono debolezze di progettazione che compromettono la qualità del software.}

\TermineGlossario{Commit}
\DefinizioneGlossario{Una serie di modifiche che sono stati esplicitamente convalidate.}

\TermineGlossario{Computation as a service (CaaS)}
\DefinizioneGlossario{Le risorse di elaborazione vengono fornite su richiesta tramite risorse virtuali o fisiche come servizio.}

\TermineGlossario{Continuous integration} 
\DefinizioneGlossario{In ingegneria del software è una pratica che si applica in contesti in cui lo sviluppo del software avviene attraverso un sistema di controllo versione. 
È dimostrabile come incremento, andando verso il risultato voluto ogni passo che si esegue ha un valore aggiuntivo.}

\TermineGlossario{Criptovaluta}
\DefinizioneGlossario{È una rappresentazione digitale di valore basata sulla crittografia.}

\TermineGlossario{Csv}
\DefinizioneGlossario{Formato di file basato su file di testo utilizzato per l'importazione ed esportazione (ad esempio da fogli elettronici o database) di una tabella di dati.}


\clearpage