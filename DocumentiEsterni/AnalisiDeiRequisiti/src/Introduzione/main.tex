\subsection{Scopo del documento}
Questo documento ha lo scopo di presentare un'analisi dei requisiti e dei casi d'uso individuati dagli analisti del gruppo \Gruppo{} per lo sviluppo del progetto \emph{\NomeProgetto}. La seguente analisi \'{e} frutto dello studio del documento di presentazione del progetto e dagli incontri tenuti con il proponente, Imola informatica S.p.A.

\subsection{Obiettivo del prodotto}
La pandemia COVID-19 sta ancora scuotendo la salute pubblica globale con un elevato numero di decessi e
una vasta diffusione geografica. Rispetto ad altri patogeni implicati in recenti pandemie, il SARS-CoV-2 si
distingue per la sua rapida diffusione e le modalità di contagio fra le persone.
In accordo con il Governo, il 14 marzo 2020, sindacati e imprese hanno firmato un protocollo per tutelare la
salute e la sicurezza dei lavoratori dal rischio contagio e garantire la salubrità dell’ambiente di lavoro. In
particolare, le aziende devono assicurare pulizia giornaliera e sanificazione periodica dei locali, degli
ambienti, delle postazioni di lavoro e delle aree comuni e di svago. In caso di contagio di un lavoratore in
occasione di lavoro, il datore di lavoro può essere chiamato a rispondere in sede penale di lesione o omicidio colposo, qualora non avesse messo in atto tutte le misure utili a contrastare tale
avvenimento.

Nello specifico, in questo progetto si prendono in considerazione due casi di studio per individuare possibili
risposte agli obblighi normativi:
\begin{itemize}
    \item tracciamento immutabile e certificato delle presenze in tempo reale alle postazioni di lavoro di un
    laboratorio informatico, contrassegnate tramite dei tag \glo{RFID} (al momento le presenze alle postazioni
    non vengono gestite e tutte le postazioni vengono sanificate indistintamente a fine giornata)
    \item tracciamento immutabile e certificato della pulizia delle postazioni (sia effettuata da azienda
    specializzata, che quella fatta in autonomia da uno studente/dipendente tramite apposito kit di
    pulizia) che devono risultare non utilizzabili se non ancora igienizzate
\end{itemize}

\subsection{Glossario}
Nel documento sono presenti alcuni termini marcati in corsivo con una \emph{G} a pedice; tale notazione sta ad indicare che \'{e} presenta la definizione di tali termini \'{e} presente nel documento \Gv{} allegato.

\subsection{Riferimenti}

\subsubsection{Normativi}
\begin{itemize}
    \item \textbf{\NdPv};
    \item ISO/IEC/IEEE 12207:1995: \url{https://www.math.unipd.it/~tullio/IS-1/2009/Approfondimenti/ISO_12207-1995.pdf};
    \item IEEE Std 830 1998: \url{https://ieeexplore.ieee.org/document/720574}.
\end{itemize}

\subsubsection{Informativi}
\begin{itemize}
    \item presentazione capitolato C1:
    \begin{itemize} 
        \item \url{https://www.math.unipd.it/~tullio/IS-1/2020/Progetto/C1.pdf};
        \item \url{https://www.youtube.com/watch?v=X_KintgpWc8&feature=youtu.be}.
    \end{itemize}
    \item materiale didattico presentato a lezione durante il corso di Ingegneria del Software: \url{https://www.math.unipd.it/~tullio/IS-1/2020/}.
    
    
\end{itemize}