\paragraph{UC-45 Visualizzazione stanze da igienizzare}
\begin{itemize}
    \item \textbf{Attori:} igienizzatore;
    \item \textbf{Descrizione:} un igienizzatore pu\`{o} visualizzare a video l'elenco delle stanze che necessitano di essere igienizzate;
    \item \textbf{Precondizione:} l'igienizzatore ha avviato l'applicazione;
    \item \textbf{Postcondizione:} l'igienizzatore visualizza l'elenco di tutte le stanze da igienizzare censite nel sistema;
    \item \textbf{Scenario principale:}
    \begin{itemize}
        \item l'igienizzatore avvia l'applicazione mobile;
        \item l'applicazione mostra a video un elenco di stanze da igienizzare.
    \end{itemize}
\end{itemize}

    \begin{itemize}
        \item \textbf{Attori primari:} igienizzatore;
        \item \textbf{Descrizione:} l’igienizzatore può visualizzare l'elenco delle stanze (contrassegnate dal nome univoco) da igienizzare;
        \item \textbf{Precondizione:} l'igienizzatore si trova nell’apposita sezione di visualizzazione delle stanze da igienizzare; 
        \item \textbf{Postcondizione:} l'igienizzatore ha ricevuto l'elenco delle stanze da igienizzare;
        \item \textbf{Scenario principale:} 
            \begin{itemize}
                \item il sistema elabora la richiesta;
                \item il sistema restituisce la lista delle stanze da igienizzare;
            \end{itemize}
    \end{itemize}
%TODO: controllare
\paragraph{UC-46 Marcatura stanza come igienizzata}

    \begin{itemize}
        \item \textbf{Attori primari:} igienizzatore;
        \item \textbf{Descrizione:} l’igienizzatore può marcare un'intera stanza come igienizzata, pertanto ogni postazione al suo interno è sanificata e servibile quindi dagli utenti. Tale azione viene certificata tramite la scansione del tag RFID associato alla stanza;
        \item \textbf{Precondizione:} l'igienizzatore naviga nell’apposita sezione di igienizzazione della stanza; 
        \item \textbf{Postcondizione:} l'igienizzatore ha marcato con successo una stanza come igienizzata;
        \item \textbf{Scenario principale:} 
            \begin{itemize}
                \item l'igienizzatore seleziona la stanza da marcare come igienizzata;	
                \item l'igienizzatore contrassegna la stanza come igienizzata scansionando il tag RFID ad essa associata (UC- .1 Scansione del tag RFID per segnalare la pulizia di una stanza);
                \item il sistema elabora correttamente la richiesta contrassegnando tutte le postazioni all'interno della stanza come igienizzate.
            \end{itemize}
    \end{itemize}

\paragraph{UC- .1 Scansione del tag RFID per segnalare la pulizia di una stanza}

    \begin{itemize}
        \item \textbf{Attori primari:} igienizzatore;
        \item \textbf{Descrizione:} l’igienizzatore vuole scansionare il tag RFID della stanza per marcarla come igienizzata;
        \item \textbf{Precondizione:} l'igienizzatore ha selezionato la stanza da marcare come igienizzata; 
        \item \textbf{Postcondizione:} l'igienizzatore ha scansionato con successo il tag RFID;
        \item \textbf{Scenario principale:} 
            \begin{itemize}	
                \item l'igienizzatore appoggia il proprio dispositivo sul tag della stanza igienizzata;
                \item il tag viene scansionato dall'applicazione;
                \item il sistema elabora la richiesta e contrassegna tutte le postazioni all'interno della stanza come igienizzate.
            \end{itemize}
    \end{itemize}
