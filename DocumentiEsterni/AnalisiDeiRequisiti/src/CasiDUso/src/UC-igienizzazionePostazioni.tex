\subsection{Igienizzazione Postazioni}
%TODO cambiare la foto
\begin{figure}[H]
  \centering
    \includegraphics[width=\textwidth]{src/CasiDUso/immagini/UC-igienizzazionePostazioni.png}
  \caption{Igienizzazione postazioni}
\end{figure}

Il presente diagramma vuole riassumere la possibilità di igienizzazione delle postazioni da parte di un utente igienizzatore all'interno dell’applicazione.

\begin{itemize}
\item \textbf{Attori primari:} igienizzatore;
\item \textbf{Descrizione:} l’utente può gestire le stanze e postazioni registrate nell’applicazione a cui ha accesso a livello di permessi, marcando come igienizzate o disabilitando le relative postazioni o intere stanze;
\item \textbf{Precondizione:} l'utente è autenticato, naviga nell’apposita sezione di gestione delle postazioni o stanze;
\item \textbf{Postcondizione:} l’utente ha gestito le proprie postazioni o stanze;
\item \textbf{Scenario principale:} 
	\begin{itemize}
		\item l’utente naviga nell’apposita sezione di gestione delle postazioni o stanze;
		\item l’utente marca come igienizzate le postazioni o stanze a cui ha accesso;
	\end{itemize}
\end{itemize}

\subsection{UC-35 Marca postazione igienizzata}

\begin{itemize}
\item \textbf{Attori primari:} igienizzatore;
\item \textbf{Descrizione:} l'utente può marcare la postazione come igienizzata, quindi servibile e prenotabile dagli utenti;
\item \textbf{Precondizione:} l'utente naviga nell’apposita sezione di gestione della postazione; 
\item \textbf{Postcondizione:} l’utente ha marcato con successo una postazione come igienizzata;
\item \textbf{Scenario principale:} 
	\begin{itemize}
		\item l'utente naviga nell’apposita sezione di gestione della postazione;		
		\item l’utente contrassegna la postazione come igienizzata;
		\item il sistema elabora correttamente la richiesta.
	\end{itemize}
\end{itemize}

%\subsubsection{UC-41 - Annulla postazione igienizzata}

%\begin{itemize}%
%\item \textbf{Attori primari:} igienizzatore;
%\item \textbf{Descrizione:} l’igienizzatore può annullare lo stato di postazione igienizzata precedentemente segnato, impostando la postazione come “non igienizzata”;
%\item \textbf{Precondizione:} l'utente naviga nell’apposita sezione di gestione della postazione; 
%\item \textbf{Postcondizione:} l'utente ha marcato con successo una postazione come non igienizzata;
%\item \textbf{Scenario principale:} 
%	\begin{itemize}
%		\item l’utente naviga nell’apposita sezione di gestione della postazione;		
%		\item l’utente annulla lo stato di postazione igienizzata precedentemente assegnato;
%		\item il sistema elabora correttamente la richiesta.
%		\end{itemize}
%\end{itemize}

\subsection{UC-36 Marca stanza igienizzata}

\begin{itemize}
\item \textbf{Attori primari:} igienizzatore;
\item \textbf{Descrizione:} l’igienizzatore può marcare un'intera stanza come igienizzata, cui ogni postazione al suo interno è sanificata e servibile quindi dagli utenti;
\item \textbf{Precondizione:} l'igienizzatore naviga nell’apposita sezione di gestione della stanza; 
\item \textbf{Postcondizione:} l'igienizzatore ha marcato con successo una stanza come igienizzata;
\item \textbf{Scenario principale:} 
	\begin{itemize}
		\item l'igienizzatore naviga nell’apposita sezione di gestione della stanza;	
		\item l'utente contrassegna la stanza come igienizzata;
		\item il sistema elabora correttamente la richiesta.
		\end{itemize}
\end{itemize}

%\subsubsection{UC-43 - Marca stanza non igienizzata}

%\begin{itemize}
%\item \textbf{Attori primari:} igienizzatore;
%\item \textbf{Descrizione:} l’igienizzatore può annullare lo stato di stanza igienizzata precedentemente segnato, impostando la stanza come “non igienizzata”;
%\item \textbf{Precondizione:} l'utente naviga nell’apposita sezione di gestione della stanza; 
%\item \textbf{Postcondizione:} l'utente ha marcato con successo una stanza come non igienizzata;
%\item \textbf{Scenario principale:} 
%	\begin{itemize}
%		\item l’utente naviga nell’apposita sezione di gestione della stanza;	
%		\item l'utente contrassegna la stanza come non igienizzata;
%		\item il sistema elabora correttamente la richiesta;
%	\end{itemize}
%\end{itemize}
