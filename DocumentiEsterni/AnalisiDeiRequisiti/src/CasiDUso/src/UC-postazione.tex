\subsection{UC-N Aggiunta postazione}
\begin{itemize}
    \item \textbf{attori:} amministratore autenticato;
    \item \textbf{scopo:} aggiungere una postazione alla lista di postazioni disponibili;
    \item \textbf{descrizione:} un amministratore pu\`{o} aggiungere una postazione alla lista di postazioni disponibili eventualmente aggiungendo la postazione ad una stanza;
    \item \textbf{pre-condizioni:} l'amministratore autenticato si trova nella pagina di gestione delle postazioni nell'applicazione web;
    \item \textbf{post-condizioni:} una nuova postazione \`{e} stata aggiunta alla lista di postazioni disponibili;
    \item \textbf{flusso degli eventi principali:}
    \begin{itemize}
        \item l'amministratore inserisce le informazioni riguardanti la postazione da aggiungere;
        \item l'amministratore conferma le informazioni inserite e le salva nel sistema.
    \end{itemize}
\end{itemize}


\subsection{UC-N Disabilitazione postazione}
\begin{itemize}
    \item \textbf{attori:} amministratore autenticato;
    \item \textbf{scopo:} rendere una postazione non pi\`{u} prenotabile;
    \item \textbf{descrizione:} un amministratore pu\`{o} disabilitare una postazione, rendendola non prenotabile. Tutte le prenotazioni per quella postazione vengono cancellate.
    \item \textbf{pre-condizioni:} l'amministratore autenticato si trova nella pagina di dettaglio della postazione nell'applicazione web;
    \item \textbf{post-condizioni:} la postazione \`{e} disabilitata e non pi\`{u} prenotabile, inoltre tutte le prenotazioni future per quella postazione sono annullate;
    \item \textbf{flusso degli eventi principali:}
    \begin{itemize}
        \item l'amministratore disabilita la postazione;
        \item la postazione non \`{e} pi`{u} prenotabile;
        \item le prenotazioni della postazione sono annullate.
    \end{itemize}
\end{itemize}


\subsection{UC-N Rimozione postazione}
\begin{itemize}
    \item \textbf{attori:} amministratore autenticato;
    \item \textbf{scopo:} rimuovere una postazione alla lista di postazioni disponibili;
    \item \textbf{descrizione:} un amministratore pu\`{o} rimuovere una postazione dalla lista di postazioni disponibili.
    \item \textbf{pre-condizioni:} l'amministratore autenticato si trova nella pagina di dettaglio della postazione nell'applicazione web;
    \item \textbf{post-condizioni:} la postazione \`{e} eliminata e non pi\`{u} prenotabile, inoltre vengono eliminate e  tutte le prenotazione per la postazione sono annullate;
    \item \textbf{flusso degli eventi principali:}
    \begin{itemize}
        \item l'amministratore rimuove la stanza;
        \item le postazioni della stanza non sono pi`{u} prenotabili;
        \item le prenotazioni delle postazioni della stanza sono annullate.
    \end{itemize}
\end{itemize}


\subsection{UC-N Visualizzazione storico postazione}
\begin{itemize}
    \item \textbf{attori:} amministratore autenticato;
    \item \textbf{scopo:} visualizzare lo storico delle azioni compiute dagli utenti del sistema su una postazione;
    \item \textbf{descrizione:} un amministratore pu\`{o} visualizzare lo storico degli utilizzi e delle pulizie della postazione con specificazione dell'utente che ha compiuto ogni azione.
    \item \textbf{pre-condizioni:} l'amministratore autenticato si trova nella pagina di dettaglio di una postazione nell'applicazione web;
    \item \textbf{post-condizioni:} l'amministratore visualizza lo storico di tutte le modifiche/prenotazioni/cancellazioni riguardanti la postazione;
    \item \textbf{flusso degli eventi principali:}
    \begin{itemize}
        \item l'amministratore visualizza le informazioni della postazione.
    \end{itemize}
\end{itemize}



\subsection{UC-N Scarica report postazione}
\begin{itemize}
    \item \textbf{attori:} amministratore autenticato;
    \item \textbf{scopo:} scaricare un report dell'utilizzo del sistema;
    \item \textbf{descrizione:} un amministratore pu\`{o} scaricare un report che sommarizza l'utilizzo della postazione.
    \item \textbf{pre-condizioni:} l'amministratore autenticato si trova nella pagina di dettaglio di una postazione nell'applicazione web;
    \item \textbf{post-condizioni:} l'amministratore scarica un report in cui \`{e} possibile consultare lo storico di tutte le modifiche/prenotazioni/cancellazioni riguardanti la postazione;
    \item \textbf{flusso degli eventi principali:}
    \begin{itemize}
        \item l'amministratore visualizza il report contenente le informazioni della postazione.
    \end{itemize}
\end{itemize}