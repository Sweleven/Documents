\subsection{Classificazione}
In seguito all'individuazione dei casi d'uso, all'analisi dei documenti di presentazione del capitolato e agli incontri svolti con il proponente, gli analisti del gruppo \Gruppo{} hanno individuato i requisiti del prodotto che sar\`{a} sviluppato. Tali requisiti sono stati raggruppati in:

\begin{itemize}
    \item \textbf{Requisiti funzionali:} insieme di requisiti che definiscono le azioni fondamentali che devono avvenire in grado di processare uin input e di generare un output;
    \item  \textbf{Requisiti dichiarativi:} insieme di requisiti che rappresentano un vincolo di natura realizzativa, normativa o contrattuale;
    \item \textbf{Requisiti qualitativi:} insieme di requisiti che garantiscono una certa qualit\`{a} al prodotto e che indicano le best practice usate per la realizzazione.
\end{itemize}

Inoltre, ad ogni requisito \`{e} stata assegnata un'importanza:

\begin{itemize}
    \item \textbf{Obbligatori:} requisito irrinunciabile poich\'{e} indispensabile per il corretto funzionamento del prodotto;
    \item \textbf{Desiderabile:} requisito non necessario che rappresenterebbe un notevole valore aggiunto;
    \item \textbf{Facoltativo:} requisito poco utile contrattabile con il proponente.
\end{itemize}



\subsection{Requisiti funzionali}
\rowcolors{2}{\evenRowColor}{\oddRowColor}
\renewcommand{\arraystretch}{1.5}
\begin{longtable}{ c C{9cm} c }
    \rowcolor{\primaryColor}
    \textcolor{\secondaryColor}{
    \textbf{Identificativo}} & \textcolor{\secondaryColor}{\textbf{Descrizione}}                                                            & \textcolor{\secondaryColor}
    {\textbf{Fonte}}                                                                                                                                                      \\

    R-1-F-0                  & Il sistema deve permettere ad un amministratore di accedere al portale web                                   & ...                         \\
    R-1.1-F-0                  & Il sistema deve permettere ad un amministratore di inserire la propria mail con la quale si \`{e} registrato & ...                         \\
    X.X.X                    & XXXX-XX-XX                                                                                                   & ...                         \\
\end{longtable}




\subsection{Requisiti funzionali}
\rowcolors{2}{\evenRowColor}{\oddRowColor}
\renewcommand{\arraystretch}{1.5}
\begin{longtable}{ c C{9cm} c }
    \rowcolor{\primaryColor}
    \textcolor{\secondaryColor}{
    \textbf{Identificativo}} & \textcolor{\secondaryColor}{\textbf{Descrizione}}                                                            & \textcolor{\secondaryColor}
    {\textbf{Fonte}}                                                                                                                                                      \\

    R-1-D-0                  & Il sistema deve permettere ad un amministratore di accedere al portale web                                   & ...                         \\
    R-1.1-D-0                  & Il sistema deve permettere ad un amministratore di inserire la propria mail con la quale si \`{e} registrato & ...                         \\
    X.X.X                    & XXXX-XX-XX                                                                                                   & ...                         \\
\end{longtable}
