
\begin{itemize}
    \item \textbf{utente non registrato:} utente che non possiede un account nel sistema;
    \item \textbf{utente non autenticato:} utente registrato che deve ancora eseguire l'accesso. A seconda delle credenziali possedute può autenticarsi come uno dei ruoli d seguito elencati.

    \item \textbf{amministratore:}
    \begin{itemize}
        \item utente registrato ed autenticato presso l'applicazione web; 
        \item pu\`{o} gestire e monitorare gli utenti, le postazioni e le stanze;
        \item può accedere all'applicazione mobile, dove però figurerà come dipendente.
    \end{itemize}

    \item \textbf{dipendente:}
    \begin{itemize}
        \item utente registrato ed autenticato presso l'applicazione mobile; 
        \item non può accedere all'applicazione web;
        \item può controllare, prenotare ed utilizzare le postazioni di lavoro;
        \item può marcare la singola postazione (ma non l'intera stanza) come pulita una volta eseguita l'igienizzazione.
    \end{itemize}

    \item \textbf{igienizzatore:}
    \begin{itemize}
        \item utente registrato ed autenticato presso l'applicazione mobile;
        \item addetto esclusivamente all'igienizzazione delle postazioni/stanze; 
        \item non può accedere all'applicazione web;
        \item non può effettuare prenotazioni;
        \item può visualizzare le stanze da pulire e marcarle come igienizzate.
    \end{itemize} 
   
    \item \textbf{utente autenticato:}
    \begin{itemize}
        \item identifica un utente che ha eseguito il login presso l'applicazione mobile indistintamente dal ruolo;
        \item usato per indicare le funzionalità comuni sia per dipendenti che per igienizzatori.
\end{itemize}
